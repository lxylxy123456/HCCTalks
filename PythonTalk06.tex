\documentclass
% [aspectratio=169]
{beamer}

\usepackage{CJKutf8}
\usepackage{listings}
\usepackage{tikz}
\usepackage{hyperref}
\usepackage{xcolor}
\usepackage{verbatim}
\usepackage{eso-pic}
\usepackage{courier}
\usepackage{textcomp}

\newcommand{\link}[1]{\href{#1}{#1}}

\newcommand{\insertGraph}[3]{

	\centerline{\includegraphics[scale=#1]{#2}} % 0/0

	\centerline{#3}

}

\newcommand{\HCCLogoSimp}{
	\begin{tikzpicture}[scale=0.1]
		\definecolor{_00ccff}{HTML}{00ccff}
\def\center{(54.9655,-19.629)}
\def\radius{7.924}
\fill [color=_00ccff]
	(55.371,-11.075) -- (57.286,-11.417) -- (55.693,-20.323) -- 
	(53.803,-19.976);
\begin{scope}
	% (47.042,-11.705) rectangle (62.889,-27.553)
	% (49.018,-13.682) rectangle (60.913,-25.576)
%	\clip
%		(59.4605,-11.563) -- (54.9665,-19.629) -- (53.3695,-11.511) -- 
%		(47.041,-11.511) -- (47.041,-27.553) -- (62.890,-27.553) -- 
%		(62.890,-11.511);
%	\fill [color=_00ccff, even odd rule]
%		\center circle [radius=5.9475]
%		(59.4605,-11.563) -- (53.3695,-11.511) -- (54.9665,-19.629)
%		\center circle [radius=\radius]; 
	% https://tex.stackexchange.com/questions/510281/tikz-fill-only-the-a-b-c
	\path \center circle [radius=\radius];
	\clip[overlay]
		(53.3695,-11.511) -- (59.4605,-11.563) -- (54.9665,-19.629) [rev];
	\clip[overlay, eo] \center circle [radius=5.9475, rev];
	\fill[color=_00ccff] \center circle [radius=\radius];
\end{scope}


	\end{tikzpicture}
}

\newcommand{\HCCLogoFull}{
	\begin{tikzpicture}[scale=0.05]
		\definecolor{_00ccff}{HTML}{00ccff}
\def\center{(54.9655,-19.629)}
\def\radius{7.924}
\fill [color=_00ccff]
	(55.371,-11.075) -- (57.286,-11.417) -- (55.693,-20.323) -- 
	(53.803,-19.976);
\begin{scope}
	% (47.042,-11.705) rectangle (62.889,-27.553)
	% (49.018,-13.682) rectangle (60.913,-25.576)
%	\clip
%		(59.4605,-11.563) -- (54.9665,-19.629) -- (53.3695,-11.511) -- 
%		(47.041,-11.511) -- (47.041,-27.553) -- (62.890,-27.553) -- 
%		(62.890,-11.511);
%	\fill [color=_00ccff, even odd rule]
%		\center circle [radius=5.9475]
%		(59.4605,-11.563) -- (53.3695,-11.511) -- (54.9665,-19.629)
%		\center circle [radius=\radius]; 
	% https://tex.stackexchange.com/questions/510281/tikz-fill-only-the-a-b-c
	\path \center circle [radius=\radius];
	\clip[overlay]
		(53.3695,-11.511) -- (59.4605,-11.563) -- (54.9665,-19.629) [rev];
	\clip[overlay, eo] \center circle [radius=5.9475, rev];
	\fill[color=_00ccff] \center circle [radius=\radius];
\end{scope}


		\definecolor{_004455}{HTML}{004455}
\definecolor{_006680}{HTML}{006680}
\definecolor{_0088aa}{HTML}{0088aa}
\definecolor{_00aad4}{HTML}{00aad4}
\fill [color=_00aad4] (49.0,-46.0) rectangle (64.0,-50.5);
\fill [color=_0088aa] (44.5,-35.5) rectangle (49.0,-50.5);
\fill [color=_006680] (44.5,-31.0) rectangle (64.0,-35.5);
\fill [color=_00aad4] (27.5,-46.0) rectangle (42.5,-50.5);
\fill [color=_0088aa] (22.5,-35.5) rectangle (27.5,-50.5);
\fill [color=_006680] (22.5,-31.0) rectangle (42.0,-35.5);
\fill [color=_0088aa] (16.0,-35.5) rectangle (20.5,-51.0);
\fill [color=_006680] (4.5,-31.0) rectangle (20.5,-35.5);
\fill [color=_004455] (0.0,-24.0) rectangle (4.5,-53.5);
% \draw [color=red] (0.0,0.0) rectangle (64,-64);

	\end{tikzpicture}
}

\setbeamercolor{background canvas}{bg=}

\newcommand{\PreFirstFrame}{
	\AddToShipoutPictureFG*{
		\AtPageLowerLeft{
			\put(\LenToUnit{0.05\paperwidth},\LenToUnit{0.1\paperheight}){
				\footnotesize
				这个指引文档在
				\href{https://creativecommons.org/licenses/by-sa/3.0/deed.zh}
				{知识共享 署名-相同方式共享 3.0协议}之条款下提供
			}
			\put(\LenToUnit{0.05\paperwidth},\LenToUnit{0.05\paperheight}){
				\footnotesize
				This guidance is available under the 
				\href{https://creativecommons.org/licenses/by-sa/3.0/}
				{Creative Commons Attribution-ShareAlike License}
			}
			\put(\LenToUnit{0.6\paperwidth},\LenToUnit{0.15\paperheight}){
				\HCCLogoFull
			}
		}
	}
}

\newcommand{\PostFirstFrame}{
	\AddToShipoutPictureBG{
		\AtPageLowerLeft{
			\put(\LenToUnit{0.8\paperwidth},\LenToUnit{0.15\paperheight}){
				\HCCLogoSimp
			}
		}
	}
}

\newcommand{\PreLastFrame}{
	\ClearShipoutPictureBG

	\AddToShipoutPictureFG*{
		\AtPageLowerLeft{
			\put(\LenToUnit{0.6\paperwidth},\LenToUnit{0.15\paperheight}){
				\HCCLogoFull
			}
		}
	}
}

% Note: this C style differs a lot from gedit's
\lstdefinestyle{cstyle}{
	language=c,
	basicstyle=\ttfamily,
	morekeywords={with},
	keywordstyle=\bfseries\color[HTML]{a52a2a},	
	commentstyle=\color[HTML]{0000ff},
	stringstyle=\color[HTML]{ff0bff},
	keywordstyle=[3]\color[HTML]{008a8c},
	alsoletter={0,1,2,3,4,5,6,7,8,9,.},
	morekeywords=[4]{0,1,2,100,999},
	keywordstyle=[4]\color[HTML]{ff0bff},
	upquote=true,
	breaklines=true,
}

\lstdefinestyle{pythonstyle}{
	language=python,
	basicstyle=\ttfamily,
	% frame=single,
	morekeywords={with,yield},
	keywordstyle=\bfseries\color[HTML]{a52a2a},	
	keywordstyle=[2]\color[HTML]{008a8c},
	commentstyle=\color[HTML]{0000ff},
	stringstyle=\color[HTML]{ff0bff},
	keywordstyle=[3]\color[HTML]{008a8c},
	alsoletter={0123456789.},
	morekeywords=[4]{False,True,
		0,1,2,3,4,5,6,7,8,9,10,11,12,13,15,17,19,16,20,24,27,31,32,33,34,35,38,
		45,56,60,64,81,95,97,99,100,123,243,256,400,512,576,729,999,1024,1234,
		1365,1366,2000,2187,2836,2957,3856,3857,5678,6561,9274,100000,1000000,
		0.5,3.14,3.4,
		0x1234,},
	keywordstyle=[4]\color[HTML]{ff0bff},
	upquote=true,
	breaklines=true,
	showstringspaces=false,
}

\lstset{
	tabsize=4,
	columns=fixed,
	extendedchars=false,
}

\newcommand{\inlinePython}{\lstinline[style=pythonstyle]}



\begin{document}
\begin{CJK}{UTF8}{gbsn}

\PreFirstFrame
\begin{frame} [fragile]
	\centerline{\fontsize{42}{42}\selectfont Python Talk 6}
\end{frame}
\PostFirstFrame

\begin{frame} [fragile]
	\frametitle{复习}
	\linespread{1.25}
	\begin{enumerate}
	\item 用\inlinePython{for}和\inlinePython{print}打印
		\begin{itemize}
		\item \inlinePython{1 2 4 8 16 32 64 }
		\end{itemize}
	\item 使用\inlinePython{input}获取输入字符串并判断其是否含\inlinePython{'HCC'}
	\item 这些函数有什么作用?
		\begin{itemize}
		\item \inlinePython{all min any id hash}
		\end{itemize}
	\item 让命令行执行
		\begin{lstlisting}[basicstyle=\ttfamily]
		ping 10.60.0.0 -c 1
		ping 10.60.0.1 -c 1
		ping 10.60.0.2 -c 1
		...
		ping 10.60.0.255 -c 1
		\end{lstlisting}
	\end{enumerate}
\end{frame}

\begin{frame} [fragile]
	\frametitle{排序}
	\linespread{1.25}
	\begin{columns}[T]
		\begin{column}[T]{.5\textwidth}
			\begin{itemize}
			\item 如何排序这个数列?
				\begin{itemize}
				\item \inlinePython{a = [3, 1, 4, 2]}
				\end{itemize}
			\item 答案
				\begin{itemize}
				\item \inlinePython{a.sort()}
				\end{itemize}
			\item 提高:如何反向排序?
			\end{itemize}
		\end{column}
		\begin{column}[T]{.5\textwidth}
			手动实现(选择排序)
			\footnotesize
			\begin{lstlisting}[style=pythonstyle, gobble=12]
			b = []
			for i in range(len(a)) :
				m = 0
				for j in range(len(a)) :
					if a[m] > a[j] :
						m = j
				b.append(a[m])
				del(a[m])
			a = b
			\end{lstlisting}
		\end{column}
	\end{columns}
\end{frame}

\begin{frame} [fragile]
	\frametitle{函数}
	\begin{itemize}
	\item 函数相当于通过参数计算出返回值
	\item 定义
		\begin{lstlisting}[style=pythonstyle, gobble=8, texcl, escapechar=@]
		def @函数名@(@参数1@, @参数2@, ...) :
			@语句@
			return @返回值@
		\end{lstlisting}
	\item 调用
		\begin{lstlisting}[style=pythonstyle, gobble=8, texcl, escapechar=@]
		@函数名@(@参数1@, @参数2@, ...)
		\end{lstlisting}
	\item 简便定义形式(lambda表达式)
		\begin{lstlisting}[style=pythonstyle, gobble=8, texcl, escapechar=@]
		@函数名@ = lambda @参数@... : @返回值@
		\end{lstlisting}
	\end{itemize}
\end{frame}

\begin{frame} [fragile]
	\frametitle{特殊排序}
	\linespread{1.25}
	\begin{itemize}
	\item 如何按照tuple的第二位排序?
		\begin{itemize}
		\item \inlinePython{a = [(1, 2), (4, 9), (6, 7)]}
		\end{itemize}
	\item 方法1
		\begin{itemize}
		\item \inlinePython{a.sort(key = lambda x: x[1])}
		\end{itemize}
	\item 方法2
		\begin{itemize}
		\item 
		\begin{lstlisting}[style=pythonstyle, gobble=8, texcl, escapechar=@]
		def f(a) : 
			return a[1]
		a.sort(key=f)
		\end{lstlisting}
		\end{itemize}
	\end{itemize}
\end{frame}

\begin{frame} [fragile]
	\frametitle{字符串编码问题}
	\linespread{1.25}
	\begin{itemize}
	\item 方法
		\begin{lstlisting}[style=pythonstyle, gobble=8, texcl, escapechar=@]
		a = '@\color{Pink}你好@'
		b = a.encode('@\color{Pink}编码@')	# 将字符串变成二进制码
		c = b.decode('@\color{Pink}编码@')	# 将字节变成字符串
		\end{lstlisting}
	\item 常见编码
		\begin{itemize}
		\item UTF-8 (多语言通用)
		\item GBK (中文)
		\item GB2312 (中文)
		\end{itemize}
	\end{itemize}
\end{frame}

\begin{frame} [fragile]
	\frametitle{进制转换}
	\begin{itemize}
	\item 从十进制转换
		\begin{lstlisting}[style=pythonstyle, gobble=8, texcl]
		bin(1024)		# 二进制
		oct(1024)		# 八进制
		hex(1024)		# 十六进制
		\end{lstlisting}
	\item 转换为十进制
		\begin{lstlisting}[style=pythonstyle, gobble=8, texcl]
		int('10', 2)	# 二进制的 10
		int('0b10', 2)	# 二进制的 10
		int('13', 7)	# 七进制的 13
		\end{lstlisting}
	\item 错误用法
		\begin{lstlisting}[style=pythonstyle, gobble=8, texcl]
		int('0b10')		# 必须注明进制数
		int('0x10', 2)	# 输入和注明进制数不匹配
		\end{lstlisting}
	\end{itemize}
\end{frame}

\begin{frame} [fragile]
	\frametitle{时间}
	\linespread{1.25}
	\begin{itemize}
	\item 关于时间的库
		\begin{lstlisting}[style=pythonstyle, gobble=8, texcl, escapechar=@]
		from datetime import datetime
		a = datetime.now()
		b = datetime(@年@, @月@, @日@, ...)	# 输入整数
		b = datetime(2000, 8, 24)	 # 例
		datetime.fromtimestamp(@时间值@)# 数值转为时间
		a.timestamp()				 # 时间转为数值
		a - b						 # 计算时间差
		a.strftime('%H:%M')			 # 输出为特定格式
		\end{lstlisting}
	\end{itemize}
\end{frame}

\begin{frame} [fragile]
	\frametitle{练习}
	\linespread{1.5}
	\begin{enumerate}
	\item 模拟网络攻击检测
		\begin{itemize}
		\item 假设某个用户上一次访问我方网站的时间已被记录为a
		\item 我们希望判断用户两次访问的间隔是否大于一秒
		\item 如果目前访问网站的时间在a后一秒之前,则输出``攻击''
		\item 否则输出``正常''
		\end{itemize}
	\item 将时间输出为中文
		\begin{itemize}
		\item 例如对于 \inlinePython{a = datetime(2000, 8, 24, 12, 34, 56)}
		\item 输出 ``2000年8月24日12时34分56秒''
		\item 提示:尝试 \inlinePython{a.strftime('%Y %m %d %H:%M:%S')}
		\end{itemize}
	\end{enumerate}
\end{frame}

\PreLastFrame
\begin{frame}
	\centerline{\fontsize{32}{32}\selectfont 感谢参加此次活动}
\end{frame}

\newpage
\end{CJK}
\end{document}

