\documentclass
% [aspectratio=169]
{beamer}

\usepackage{CJKutf8}
\usepackage{listings}
\usepackage{tikz}
\usepackage{hyperref}
\usepackage{xcolor}
\usepackage{verbatim}
\usepackage{eso-pic}
\usepackage{courier}
\usepackage{textcomp}

\newcommand{\link}[1]{\href{#1}{#1}}

\newcommand{\insertGraph}[3]{

	\centerline{\includegraphics[scale=#1]{#2}} % 0/0

	\centerline{#3}

}

\newcommand{\HCCLogoSimp}{
	\begin{tikzpicture}[scale=0.1]
		\definecolor{_00ccff}{HTML}{00ccff}
\def\center{(54.9655,-19.629)}
\def\radius{7.924}
\fill [color=_00ccff]
	(55.371,-11.075) -- (57.286,-11.417) -- (55.693,-20.323) -- 
	(53.803,-19.976);
\begin{scope}
	% (47.042,-11.705) rectangle (62.889,-27.553)
	% (49.018,-13.682) rectangle (60.913,-25.576)
%	\clip
%		(59.4605,-11.563) -- (54.9665,-19.629) -- (53.3695,-11.511) -- 
%		(47.041,-11.511) -- (47.041,-27.553) -- (62.890,-27.553) -- 
%		(62.890,-11.511);
%	\fill [color=_00ccff, even odd rule]
%		\center circle [radius=5.9475]
%		(59.4605,-11.563) -- (53.3695,-11.511) -- (54.9665,-19.629)
%		\center circle [radius=\radius]; 
	% https://tex.stackexchange.com/questions/510281/tikz-fill-only-the-a-b-c
	\path \center circle [radius=\radius];
	\clip[overlay]
		(53.3695,-11.511) -- (59.4605,-11.563) -- (54.9665,-19.629) [rev];
	\clip[overlay, eo] \center circle [radius=5.9475, rev];
	\fill[color=_00ccff] \center circle [radius=\radius];
\end{scope}


	\end{tikzpicture}
}

\newcommand{\HCCLogoFull}{
	\begin{tikzpicture}[scale=0.05]
		\definecolor{_00ccff}{HTML}{00ccff}
\def\center{(54.9655,-19.629)}
\def\radius{7.924}
\fill [color=_00ccff]
	(55.371,-11.075) -- (57.286,-11.417) -- (55.693,-20.323) -- 
	(53.803,-19.976);
\begin{scope}
	% (47.042,-11.705) rectangle (62.889,-27.553)
	% (49.018,-13.682) rectangle (60.913,-25.576)
%	\clip
%		(59.4605,-11.563) -- (54.9665,-19.629) -- (53.3695,-11.511) -- 
%		(47.041,-11.511) -- (47.041,-27.553) -- (62.890,-27.553) -- 
%		(62.890,-11.511);
%	\fill [color=_00ccff, even odd rule]
%		\center circle [radius=5.9475]
%		(59.4605,-11.563) -- (53.3695,-11.511) -- (54.9665,-19.629)
%		\center circle [radius=\radius]; 
	% https://tex.stackexchange.com/questions/510281/tikz-fill-only-the-a-b-c
	\path \center circle [radius=\radius];
	\clip[overlay]
		(53.3695,-11.511) -- (59.4605,-11.563) -- (54.9665,-19.629) [rev];
	\clip[overlay, eo] \center circle [radius=5.9475, rev];
	\fill[color=_00ccff] \center circle [radius=\radius];
\end{scope}


		\definecolor{_004455}{HTML}{004455}
\definecolor{_006680}{HTML}{006680}
\definecolor{_0088aa}{HTML}{0088aa}
\definecolor{_00aad4}{HTML}{00aad4}
\fill [color=_00aad4] (49.0,-46.0) rectangle (64.0,-50.5);
\fill [color=_0088aa] (44.5,-35.5) rectangle (49.0,-50.5);
\fill [color=_006680] (44.5,-31.0) rectangle (64.0,-35.5);
\fill [color=_00aad4] (27.5,-46.0) rectangle (42.5,-50.5);
\fill [color=_0088aa] (22.5,-35.5) rectangle (27.5,-50.5);
\fill [color=_006680] (22.5,-31.0) rectangle (42.0,-35.5);
\fill [color=_0088aa] (16.0,-35.5) rectangle (20.5,-51.0);
\fill [color=_006680] (4.5,-31.0) rectangle (20.5,-35.5);
\fill [color=_004455] (0.0,-24.0) rectangle (4.5,-53.5);
% \draw [color=red] (0.0,0.0) rectangle (64,-64);

	\end{tikzpicture}
}

\setbeamercolor{background canvas}{bg=}

\newcommand{\PreFirstFrame}{
	\AddToShipoutPictureFG*{
		\AtPageLowerLeft{
			\put(\LenToUnit{0.05\paperwidth},\LenToUnit{0.1\paperheight}){
				\footnotesize
				这个指引文档在
				\href{https://creativecommons.org/licenses/by-sa/3.0/deed.zh}
				{知识共享 署名-相同方式共享 3.0协议}之条款下提供
			}
			\put(\LenToUnit{0.05\paperwidth},\LenToUnit{0.05\paperheight}){
				\footnotesize
				This guidance is available under the 
				\href{https://creativecommons.org/licenses/by-sa/3.0/}
				{Creative Commons Attribution-ShareAlike License}
			}
			\put(\LenToUnit{0.6\paperwidth},\LenToUnit{0.15\paperheight}){
				\HCCLogoFull
			}
		}
	}
}

\newcommand{\PostFirstFrame}{
	\AddToShipoutPictureBG{
		\AtPageLowerLeft{
			\put(\LenToUnit{0.8\paperwidth},\LenToUnit{0.15\paperheight}){
				\HCCLogoSimp
			}
		}
	}
}

\newcommand{\PreLastFrame}{
	\ClearShipoutPictureBG

	\AddToShipoutPictureFG*{
		\AtPageLowerLeft{
			\put(\LenToUnit{0.6\paperwidth},\LenToUnit{0.15\paperheight}){
				\HCCLogoFull
			}
		}
	}
}

% Note: this C style differs a lot from gedit's
\lstdefinestyle{cstyle}{
	language=c,
	basicstyle=\ttfamily,
	morekeywords={with},
	keywordstyle=\bfseries\color[HTML]{a52a2a},	
	commentstyle=\color[HTML]{0000ff},
	stringstyle=\color[HTML]{ff0bff},
	keywordstyle=[3]\color[HTML]{008a8c},
	alsoletter={0,1,2,3,4,5,6,7,8,9,.},
	morekeywords=[4]{0,1,2,100,999},
	keywordstyle=[4]\color[HTML]{ff0bff},
	upquote=true,
	breaklines=true,
}

\lstdefinestyle{pythonstyle}{
	language=python,
	basicstyle=\ttfamily,
	% frame=single,
	morekeywords={with,yield},
	keywordstyle=\bfseries\color[HTML]{a52a2a},	
	keywordstyle=[2]\color[HTML]{008a8c},
	commentstyle=\color[HTML]{0000ff},
	stringstyle=\color[HTML]{ff0bff},
	keywordstyle=[3]\color[HTML]{008a8c},
	alsoletter={0123456789.},
	morekeywords=[4]{False,True,
		0,1,2,3,4,5,6,7,8,9,10,11,12,13,15,17,19,16,20,24,27,31,32,33,34,35,38,
		45,56,60,64,81,95,97,99,100,123,243,256,400,512,576,729,999,1024,1234,
		1365,1366,2000,2187,2836,2957,3856,3857,5678,6561,9274,100000,1000000,
		0.5,3.14,3.4,
		0x1234,},
	keywordstyle=[4]\color[HTML]{ff0bff},
	upquote=true,
	breaklines=true,
	showstringspaces=false,
}

\lstset{
	tabsize=4,
	columns=fixed,
	extendedchars=false,
}

\newcommand{\inlinePython}{\lstinline[style=pythonstyle]}



\begin{document}

\PreFirstFrame
\begin{frame} [fragile]
	\centerline{\fontsize{42}{42}\selectfont Python Talk 9}
\end{frame}
\PostFirstFrame

\begin{frame} [fragile]
	\frametitle{函数式编程}
	  \href{https://zh.wikipedia.org/wiki/\%E5\%87\%BD\%E6\%95\%B0\%E5\%BC\%8F\%E7\%BC\%96\%E7\%A8\%8B}{函数式编程}或称函数程序设计,又称泛函编程,是一种编程典范,它将电脑运算视为数学上的函数计算,并且避免使用程序状态以及易变物件。函数程式语言最重要的基础是λ演算(lambda calculus)。而且λ演算     的函数可以接受函数当作输入(引数)和输出(传出值)。

	  比起指令式编程,函数式编程更加强调程序执行的结果而非执行的过程,倡导利用若干简单的执行单元让计算结果不断渐进,逐层推导复杂的运算,而不是设计一个复杂的执行过程。

	  (以上内容来自维基百科)
\end{frame}

\begin{frame} [fragile]
	\frametitle{Python Build-in Functions}
	\linespread{1.5}
	\begin{columns}[T]
		\begin{column}[T]{.333333\textwidth}
			多 $\to$ 一
			\begin{itemize}
				\item \inlinePython{all}
				\item \inlinePython{any}
				\item \inlinePython{len}
				\item \inlinePython{min}
				\item \inlinePython{max}
				\item \inlinePython{sum}
			\end{itemize}
		\end{column}
		\begin{column}[T]{.333333\textwidth}
			多 $\to$ 多
			\begin{itemize}
				\item \inlinePython{enumerate}
				\item \inlinePython{filter}
				\item \inlinePython{map}
				\item \inlinePython{reversed}
				\item \inlinePython{sorted}
				\item \inlinePython{zip}
			\end{itemize}
		\end{column}
		\begin{column}[T]{.333333\textwidth}
			其他
			\begin{itemize}
				\item \inlinePython{range}
				\item \inlinePython{slice}
				\item \inlinePython{yield}
			\end{itemize}
		\end{column}
	\end{columns}
\end{frame}

\begin{frame} [fragile]
	\frametitle{\inlinePython{all}, \inlinePython{any}, \inlinePython{len}}
	\linespread{1.5}
	\begin{lstlisting}[style=pythonstyle, gobble=4, texcl]
	all					# 是否所有元素都为真
	all([1, 2, 3])		# True
	all([1, 2, ''])		# False

	any					# 是否有任何元素为真
	any([0, ''])		# False
	any([0, 1])			# True

	len					# 复杂数据类型的长度
	len([0, 'a', 9])	# 3
	\end{lstlisting}
\end{frame}

\begin{frame} [fragile]
	\frametitle{\inlinePython{max}, \inlinePython{min}, \inlinePython{sum}}
	\linespread{1.5}
	\begin{lstlisting}[style=pythonstyle, gobble=4, texcl]
	max					# 最大值
	max([1, 2, 3])		# 3

	min					# 最小值
	min([1, 2, 3])		# 1

	sum					# 总和
	sum([1, 2, 3])		# 6
	\end{lstlisting}
\end{frame}

\begin{frame} [fragile]
	\frametitle{\inlinePython{sorted}, \inlinePython{reversed}}
	\linespread{1.5}
	\begin{lstlisting}[style=pythonstyle, gobble=4, texcl]
	sorted				# 排序元素
	k = lambda x: x[-1]	# 根据最后一个元素排序
	sorted([(9, 3), (4, ), (2, 5)], key=k)

	reversed			# 反转元素
	reversed([(9, 3), (4, ), (2, 5)])
	\end{lstlisting}
\end{frame}

\begin{frame} [fragile]
	\frametitle{\inlinePython{lambda}表达式}
	\linespread{1.5}
	\begin{lstlisting}[style=pythonstyle, gobble=4, texcl, escapechar=@]
	lambda @参数列表@: @函数返回值@
	pow = lambda x, y: x ** y
	f = lambda x: x ** 2 + 3 * x + 1
	\end{lstlisting}
	\begin{itemize}
	\item 函数 \inlinePython{f} 在数学上相当于 $f(x) = x^2 + 3\,x + 1$
	\item 优点:代码简洁
	\item 缺点:无法进行复杂运算
	\item 提示:不要滥用\inlinePython{lambda}表达式,否则别人会很难理解你的代码
	\end{itemize}
\end{frame}

\begin{frame} [fragile]
	\frametitle{\inlinePython{reversed}实践}
	\linespread{1.25}
	\begin{lstlisting}[style=pythonstyle, gobble=4, texcl]
	>>> reversed([1, 2, 3])
	<list_reverseiterator object at 0x123456789AB>
	>>> # 这时返回的是一个iter的生成器,需要手动转换格式
	>>> # 一般用list函数转换格式
	>>> # 在特殊情况下可以用next
	>>> list(reversed([1, 2, 3]))
	[3, 2, 1]
	>>>
	\end{lstlisting}
\end{frame}

\begin{frame} [fragile]
	\frametitle{\inlinePython{enumerate}}
	\begin{itemize}
	\item \inlinePython{enumerate}插入序号
	\item 输入
		\begin{lstlisting}[style=pythonstyle, gobble=8, texcl]
		['a', 'b', 'c']
		\end{lstlisting}
	\item 输出
		\begin{lstlisting}[style=pythonstyle, gobble=8, texcl]
		[
			(0, 'a'),
			(1, 'b'),
			(2, 'c'),
		]
		\end{lstlisting}
	\item 例:带序号打印列表
		\begin{lstlisting}[style=pythonstyle, gobble=8, texcl]
		for i, j in enumerate(['a', 'b', 'c']) :
			print(i, j)
		\end{lstlisting}
	\end{itemize}
\end{frame}

\begin{frame} [fragile]
	\frametitle{\inlinePython{zip}}
	\begin{itemize}
	\item \inlinePython{zip}可以拼接列表
	\begin{columns}
		\begin{column}[T]{.1\textwidth}
		\end{column}
		\begin{column}[T]{.4\textwidth}
			\begin{itemize}
			\item 输入
			\begin{lstlisting}[style=pythonstyle, gobble=12, texcl]
			[1, 2, 3]
			[6, 7, 8]
			\end{lstlisting}
			\end{itemize}
		\end{column}
		\begin{column}[T]{.4\textwidth}
			\begin{itemize}
			\item 输出
			\begin{lstlisting}[style=pythonstyle, gobble=12, texcl]
			[
				(1, 6),
				(2, 7),
				(3, 8),
			]
			\end{lstlisting}
			\end{itemize}
		\end{column}
		\begin{column}[T]{.1\textwidth}
		\end{column}
	\end{columns}
	\item 例:同时打印两个列表
		\begin{lstlisting}[style=pythonstyle, gobble=8, texcl]
		A = [1, 2]
		B = ['a', 'b']
		for i, j in zip(a, b) :
			print(i)
			print(j)
		\end{lstlisting}
	\end{itemize}
\end{frame}

\begin{frame} [fragile]
	\frametitle{\inlinePython{map}}
	\begin{itemize}
	\item \inlinePython{map}可以将列表的每个元素分别用同一个函数执行
	\begin{columns}
		\begin{column}[T]{.1\textwidth}
		\end{column}
		\begin{column}[T]{.4\textwidth}
			\begin{itemize}
			\item 输入
			\begin{lstlisting}[style=pythonstyle, gobble=12, texcl]
			lambda x: x ** 2
			[1, 2, 3, 4]
			\end{lstlisting}
			\end{itemize}
		\end{column}
		\begin{column}[T]{.4\textwidth}
			\begin{itemize}
			\item 输出
			\begin{lstlisting}[style=pythonstyle, gobble=12, texcl]
			[1, 4, 9, 16]
			\end{lstlisting}
			\end{itemize}
		\end{column}
		\begin{column}[T]{.1\textwidth}
		\end{column}
	\end{columns}
	\item 练习
		\begin{lstlisting}[style=pythonstyle, gobble=8, texcl]
		a = range(100)
		\end{lstlisting}
		使用\inlinePython{map}、\inlinePython{sum}和\inlinePython
		{lambda},用一行程序求\inlinePython{a}的立方和
	\end{itemize}
\end{frame}

\begin{frame} [fragile]
	\frametitle{立方和解法}
	\begin{columns}
		\begin{column}[T]{.42\textwidth}
			非函数式
			\begin{lstlisting}[style=pythonstyle, gobble=12, texcl]
			s = 0
			for i in range(a) :
				s += i ** 3
			print(s)
			\end{lstlisting}
		\end{column}
		\begin{column}[T]{.58\textwidth}
			函数式
			\begin{lstlisting}[style=pythonstyle, gobble=12, texcl]
			sum(map(lambda x: x**3, a))
			\end{lstlisting}

			\

			展开后
			\begin{lstlisting}[style=pythonstyle, gobble=12, texcl]
			sum(
				map(
					lambda x: x**3,
					a
				)
			)
			\end{lstlisting}
		\end{column}
	\end{columns}
\end{frame}

\begin{frame} [fragile]
	\frametitle{\inlinePython{filter}}
	\begin{itemize}
	\item \inlinePython{filter}可筛选元素
	\begin{columns}
		\begin{column}[T]{.1\textwidth}
		\end{column}
		\begin{column}[T]{.4\textwidth}
			\begin{itemize}
			\item 输入
			\begin{lstlisting}[style=pythonstyle, gobble=12, texcl]
			lambda x: x > 4
			[3, 4, 5, 6, 7]
			\end{lstlisting}
			\end{itemize}
		\end{column}
		\begin{column}[T]{.4\textwidth}
			\begin{itemize}
			\item 输出
			\begin{lstlisting}[style=pythonstyle, gobble=12, texcl]
			[5, 6, 7]
			\end{lstlisting}
			\end{itemize}
		\end{column}
		\begin{column}[T]{.1\textwidth}
		\end{column}
	\end{columns}
	\item 练习
		\begin{lstlisting}[style=pythonstyle, gobble=8, texcl]
		a = range(1000)
		\end{lstlisting}
		不重复地打印出\inlinePython{a}中所有三\textbf{或}七的倍数
	\end{itemize}
\end{frame}

\begin{frame} [fragile]
	\frametitle{打印倍数解法}
	\begin{columns}
		\begin{column}[T]{.42\textwidth}
			非函数式
			\begin{lstlisting}[style=pythonstyle, gobble=12, texcl]
			for i in a :
				if i % 3 == 0 :
					print(i)
				if i % 7 == 0 :
					print(i)

			# 尝试找出以上程序的一处错误
			\end{lstlisting}
		\end{column}
		\begin{column}[T]{.58\textwidth}
			函数式
			\begin{lstlisting}[style=pythonstyle, gobble=12, texcl]
			filter(lambda x: x % 3 == 0 or x % 7 == 0, a)
			\end{lstlisting}

			\

			展开后
			\begin{lstlisting}[style=pythonstyle, gobble=12, texcl]
			filter(
				lambda x: x % 3 == 0 or
						  x % 7 == 0,
				a
			)
			\end{lstlisting}
		\end{column}
	\end{columns}
\end{frame}

\begin{frame} [fragile]
	\frametitle{\inlinePython{range}}
	\linespread{1.25}
	\begin{itemize}
	\item \inlinePython{range}可以快速得到一个等差整数数列
	\begin{lstlisting}[style=pythonstyle, gobble=4, texcl]
	>>> range(1, 10, 2)
	range(1, 10, 2)
	>>> list(range(1, 10, 2))
	[1, 3, 5, 7, 9]
	>>> list(range(5, 3, -1))	# 反向
	[5, 4]
	\end{lstlisting}
	\end{itemize}
\end{frame}

\begin{frame} [fragile]
	\frametitle{\inlinePython{slice}}
	\linespread{1.25}
	\begin{itemize}
	\item \inlinePython{slice}即切片,和 \inlinePython{[a:b:c]} 相同
	\begin{lstlisting}[style=pythonstyle, gobble=4, texcl]
	>>> a = [1, 2, 3, 4, 5, 6, 7, 8, 9]
	>>> a[2: 7: 2]
	[3, 5, 7]
	>>> a[slice(2, 7, 2)]
	[3, 5, 7]
	\end{lstlisting}
	\end{itemize}
\end{frame}

\begin{frame} [fragile]
	\frametitle{\inlinePython{yield}}
	\begin{itemize}
	\item \inlinePython{yield}在函数的返回值中添加元素
	\begin{lstlisting}[style=pythonstyle, gobble=4, texcl]
	def f(x):
		for i in range(x, 2 * x) :
			yield(i)

	def g(x):		# 不用yield的写法
		ans = []
		for i in range(x, 2 * x) :
			ans.append(i)
		return ans

	list(f(3))		# [3, 4, 5]
	g(3)			# [3, 4, 5]
	\end{lstlisting}
	\end{itemize}
\end{frame}

\begin{frame} [fragile]
	\frametitle{练习}
	\begin{itemize}
	\item 以下程序可以干什么?
	\begin{lstlisting}[style=pythonstyle, gobble=4, texcl]
	sum(filter(lambda x: x % 3 == 1, map(lambda x: x ** 2, range(20))))
	\end{lstlisting}
	\item 展开后
	\begin{lstlisting}[style=pythonstyle, gobble=4, texcl]
	sum(
		filter(
			lambda x: x % 3 == 1,
			map(
				lambda x : x ** 2,
				range(20)
			)
		)
	)
	\end{lstlisting}
	\end{itemize}
\end{frame}

\begin{frame} [fragile]
	\frametitle{\inlinePython{iter}和\inlinePython{next}}
	\begin{itemize}
	\item \inlinePython{next}可以将复杂数据类型转换为生成器
	\item \inlinePython{iter}可以遍历一个函数式编程得到的生成器
	\item 也可以用\inlinePython{__iter__}和\inlinePython{__next__}
	\end{itemize}
	\begin{columns}
		\begin{column}[T]{.5\textwidth}
			\small
			\begin{lstlisting}[style=pythonstyle, gobble=12, texcl]
			>>> a = [1, 2]
			>>> b = a.__iter__()
			>>> b.__next__()
			1
			>>> b.__next__()
			2
			>>> b.__next__()
			Traceback (most recent call last):
			  File "<stdin>", line 1, in <module>
			StopIteration
			>>>
			\end{lstlisting}
		\end{column}
		\begin{column}[T]{.5\textwidth}
			\small
			\begin{lstlisting}[style=pythonstyle, gobble=12, texcl]
			>>> a = [1, 2]
			>>> b = iter(a)
			>>> next(b)
			1
			>>> next(b)
			2
			>>> next(b)
			Traceback (most recent call last):
			  File "<stdin>", line 1, in <module>
			StopIteration
			>>>
			\end{lstlisting}
		\end{column}
	\end{columns}
\end{frame}

\PreLastFrame
\begin{frame}
	\centerline{\fontsize{32}{32}\selectfont 感谢参加此次活动}
\end{frame}

\newpage
\end{document}

