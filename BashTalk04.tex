\documentclass
% [aspectratio=169]
{beamer}

\usepackage{CJKutf8}
\usepackage{listings}
\usepackage{tikz}
\usepackage{hyperref}
\usepackage{xcolor}
\usepackage{verbatim}
\usepackage{eso-pic}
\usepackage{courier}
\usepackage{textcomp}

\newcommand{\link}[1]{\href{#1}{#1}}

\newcommand{\insertGraph}[3]{

	\centerline{\includegraphics[scale=#1]{#2}} % 0/0

	\centerline{#3}

}

\newcommand{\HCCLogoSimp}{
	\begin{tikzpicture}[scale=0.1]
		\definecolor{_00ccff}{HTML}{00ccff}
\def\center{(54.9655,-19.629)}
\def\radius{7.924}
\fill [color=_00ccff]
	(55.371,-11.075) -- (57.286,-11.417) -- (55.693,-20.323) -- 
	(53.803,-19.976);
\begin{scope}
	% (47.042,-11.705) rectangle (62.889,-27.553)
	% (49.018,-13.682) rectangle (60.913,-25.576)
%	\clip
%		(59.4605,-11.563) -- (54.9665,-19.629) -- (53.3695,-11.511) -- 
%		(47.041,-11.511) -- (47.041,-27.553) -- (62.890,-27.553) -- 
%		(62.890,-11.511);
%	\fill [color=_00ccff, even odd rule]
%		\center circle [radius=5.9475]
%		(59.4605,-11.563) -- (53.3695,-11.511) -- (54.9665,-19.629)
%		\center circle [radius=\radius]; 
	% https://tex.stackexchange.com/questions/510281/tikz-fill-only-the-a-b-c
	\path \center circle [radius=\radius];
	\clip[overlay]
		(53.3695,-11.511) -- (59.4605,-11.563) -- (54.9665,-19.629) [rev];
	\clip[overlay, eo] \center circle [radius=5.9475, rev];
	\fill[color=_00ccff] \center circle [radius=\radius];
\end{scope}


	\end{tikzpicture}
}

\newcommand{\HCCLogoFull}{
	\begin{tikzpicture}[scale=0.05]
		\definecolor{_00ccff}{HTML}{00ccff}
\def\center{(54.9655,-19.629)}
\def\radius{7.924}
\fill [color=_00ccff]
	(55.371,-11.075) -- (57.286,-11.417) -- (55.693,-20.323) -- 
	(53.803,-19.976);
\begin{scope}
	% (47.042,-11.705) rectangle (62.889,-27.553)
	% (49.018,-13.682) rectangle (60.913,-25.576)
%	\clip
%		(59.4605,-11.563) -- (54.9665,-19.629) -- (53.3695,-11.511) -- 
%		(47.041,-11.511) -- (47.041,-27.553) -- (62.890,-27.553) -- 
%		(62.890,-11.511);
%	\fill [color=_00ccff, even odd rule]
%		\center circle [radius=5.9475]
%		(59.4605,-11.563) -- (53.3695,-11.511) -- (54.9665,-19.629)
%		\center circle [radius=\radius]; 
	% https://tex.stackexchange.com/questions/510281/tikz-fill-only-the-a-b-c
	\path \center circle [radius=\radius];
	\clip[overlay]
		(53.3695,-11.511) -- (59.4605,-11.563) -- (54.9665,-19.629) [rev];
	\clip[overlay, eo] \center circle [radius=5.9475, rev];
	\fill[color=_00ccff] \center circle [radius=\radius];
\end{scope}


		\definecolor{_004455}{HTML}{004455}
\definecolor{_006680}{HTML}{006680}
\definecolor{_0088aa}{HTML}{0088aa}
\definecolor{_00aad4}{HTML}{00aad4}
\fill [color=_00aad4] (49.0,-46.0) rectangle (64.0,-50.5);
\fill [color=_0088aa] (44.5,-35.5) rectangle (49.0,-50.5);
\fill [color=_006680] (44.5,-31.0) rectangle (64.0,-35.5);
\fill [color=_00aad4] (27.5,-46.0) rectangle (42.5,-50.5);
\fill [color=_0088aa] (22.5,-35.5) rectangle (27.5,-50.5);
\fill [color=_006680] (22.5,-31.0) rectangle (42.0,-35.5);
\fill [color=_0088aa] (16.0,-35.5) rectangle (20.5,-51.0);
\fill [color=_006680] (4.5,-31.0) rectangle (20.5,-35.5);
\fill [color=_004455] (0.0,-24.0) rectangle (4.5,-53.5);
% \draw [color=red] (0.0,0.0) rectangle (64,-64);

	\end{tikzpicture}
}

\setbeamercolor{background canvas}{bg=}

\newcommand{\PreFirstFrame}{
	\AddToShipoutPictureFG*{
		\AtPageLowerLeft{
			\put(\LenToUnit{0.05\paperwidth},\LenToUnit{0.1\paperheight}){
				\footnotesize
				这个指引文档在
				\href{https://creativecommons.org/licenses/by-sa/3.0/deed.zh}
				{知识共享 署名-相同方式共享 3.0协议}之条款下提供
			}
			\put(\LenToUnit{0.05\paperwidth},\LenToUnit{0.05\paperheight}){
				\footnotesize
				This guidance is available under the 
				\href{https://creativecommons.org/licenses/by-sa/3.0/}
				{Creative Commons Attribution-ShareAlike License}
			}
			\put(\LenToUnit{0.6\paperwidth},\LenToUnit{0.15\paperheight}){
				\HCCLogoFull
			}
		}
	}
}

\newcommand{\PostFirstFrame}{
	\AddToShipoutPictureBG{
		\AtPageLowerLeft{
			\put(\LenToUnit{0.8\paperwidth},\LenToUnit{0.15\paperheight}){
				\HCCLogoSimp
			}
		}
	}
}

\newcommand{\PreLastFrame}{
	\ClearShipoutPictureBG

	\AddToShipoutPictureFG*{
		\AtPageLowerLeft{
			\put(\LenToUnit{0.6\paperwidth},\LenToUnit{0.15\paperheight}){
				\HCCLogoFull
			}
		}
	}
}

% Note: this C style differs a lot from gedit's
\lstdefinestyle{cstyle}{
	language=c,
	basicstyle=\ttfamily,
	morekeywords={with},
	keywordstyle=\bfseries\color[HTML]{a52a2a},	
	commentstyle=\color[HTML]{0000ff},
	stringstyle=\color[HTML]{ff0bff},
	keywordstyle=[3]\color[HTML]{008a8c},
	alsoletter={0,1,2,3,4,5,6,7,8,9,.},
	morekeywords=[4]{0,1,2,100,999},
	keywordstyle=[4]\color[HTML]{ff0bff},
	upquote=true,
	breaklines=true,
}

\lstdefinestyle{pythonstyle}{
	language=python,
	basicstyle=\ttfamily,
	% frame=single,
	morekeywords={with,yield},
	keywordstyle=\bfseries\color[HTML]{a52a2a},	
	keywordstyle=[2]\color[HTML]{008a8c},
	commentstyle=\color[HTML]{0000ff},
	stringstyle=\color[HTML]{ff0bff},
	keywordstyle=[3]\color[HTML]{008a8c},
	alsoletter={0123456789.},
	morekeywords=[4]{False,True,
		0,1,2,3,4,5,6,7,8,9,10,11,12,13,15,17,19,16,20,24,27,31,32,33,34,35,38,
		45,56,60,64,81,95,97,99,100,123,243,256,400,512,576,729,999,1024,1234,
		1365,1366,2000,2187,2836,2957,3856,3857,5678,6561,9274,100000,1000000,
		0.5,3.14,3.4,
		0x1234,},
	keywordstyle=[4]\color[HTML]{ff0bff},
	upquote=true,
	breaklines=true,
	showstringspaces=false,
}

\lstset{
	tabsize=4,
	columns=fixed,
	extendedchars=false,
}

\newcommand{\inlinePython}{\lstinline[style=pythonstyle]}



\begin{document}
\begin{CJK}{UTF8}{gbsn}

\PreFirstFrame
\begin{frame} [fragile]
	\centerline{\fontsize{42}{42}\selectfont Bash Talk 4}
\end{frame}
\PostFirstFrame

\begin{frame} [fragile]
	\frametitle{复习}
	\linespread{1.5}
	\begin{columns}[T]
		\begin{column}[T]{.5\textwidth}
			\begin{itemize}
			\item 如何……?
				\begin{itemize}
				\item 查看进程信息
				\item 查看网络连接
				\item 将数据输出到文件
				\item 依次执行两个命令
				\item 访问父目录
				\end{itemize}
			\end{itemize}
		\end{column}
		\begin{column}[T]{.5\textwidth}
			\begin{lstlisting}[style=bashstyle, gobble=12, texcl]
			*		?		/	//
			~		;		<	<<
			.		..		>	>>
			uname	top		ps -aux
			sleep	time	date
			sync	lspci
			free	ifconfig
			\end{lstlisting}
		\end{column}
	\end{columns}
\end{frame}

\begin{frame} [fragile]
	\frametitle{用户变更}
	\linespread{1.25}
	\begin{itemize}
	\item 命令
	\begin{lstlisting}[style=bashstyle, gobble=4, texcl, escapechar=@]
	su @用户 @		# 变成某个用户(默认root)
	sudo @命令 @	# 用root执行命令
	\end{lstlisting}
	\item 尝试
	\begin{lstlisting}[style=bashstyle, gobble=4, texcl]
	su root		 # 切换为用户
	su hcc		 # root 不用输密码
	exit		 # 退出 su hcc
	exit		 # 退出 su root
	sudo su
	exit		 # 退出 sudo su
	su hcc		 # 需要密码
	\end{lstlisting}
	\end{itemize}
\end{frame}

\begin{frame} [fragile]
	\frametitle{历史记录}
	\linespread{1.25}
	\begin{itemize}
	\item 方法
		\begin{itemize}
		\item 按动上下键
		\item 文件 \inlineBash{.bash_history}
		\item \inlineBash{history}
		\end{itemize}
	\item 问题:后两个有什么区别?
	\item 尝试
	\begin{lstlisting}[style=bashstyle, gobble=4, texcl]
	history
	cat .bash_history
	echo "HCC, I'm"
	echo "HCC"
	history
	cat .bash_history
	\end{lstlisting}
	\end{itemize}
\end{frame}

\begin{frame} [fragile]
	\frametitle{软件包管理}
	\linespread{1.25}
	\begin{itemize}
	\item Linux:我们将介绍 Fedora 上的软件包管理方式
		\begin{itemize}
		\item Fedora 22 之前应使用 \inlineBash{yum} 而非 \inlineBash{dnf}
		\end{itemize}
	\begin{lstlisting}[style=bashstyle, gobble=4, texcl, escapechar=@]
	rpm -aq					  # 查看已安装软件包
	sudo yum install @程序@ ...@ @# 安装
	sudo yum update			  # 更新
	sudo yum remove @程序@ ...@ @ # 卸载
	\end{lstlisting}
	\item Windows
		\begin{itemize}
		\item 安装:浏览器上用搜索引擎搜索软件名,下载(可能需要付费),安装
		\item 卸载:``卸载或更改程序'' $\to$ 选中一个程序 $\to$ 卸载
		\end{itemize}
	\end{itemize}
\end{frame}

\begin{frame} [fragile]
	\frametitle{7-Zip}
	\begin{itemize}
	\item 使用\inlineBash{7za}命令处理压缩包
	\begin{lstlisting}[style=bashstyle, gobble=4, texcl, escapechar=@]
	7za a @\color{Pink}压缩包.7z@ @\color{Pink}文件@ ...@ @	# 压缩
	7za l @\color{Pink}压缩包.7z@			 # 列出内容
	7za x @\color{Pink}压缩包.7z@			 # 提取
	7za t @\color{Pink}压缩包.7z@			 # 检测是否损坏
	\end{lstlisting}
	\item 探索:压缩时如何指定压缩率?
	\item 尝试
	\begin{lstlisting}[style=bashstyle, gobble=4, texcl, escapechar=@]
	# 创建一个包含一些文件的目录,命名为 mydoc
	7za a mydoc.7z mydoc
	7za l mydoc.7z
	7za t mydoc.7z
	mv mydoc.7z /tmp
	7za x /tmp/@\color{Pink}m*.7z@
	\end{lstlisting}
	\end{itemize}
\end{frame}

\begin{frame} [fragile]
	\frametitle{\inlineBash{nano}}
	\linespread{1.25}
	\begin{itemize}
	\item 使用 \inlineBash{nano} 编辑文本
	\begin{lstlisting}[style=bashstyle, gobble=4, texcl, escapechar=@]
	nano
	nano @\color{Pink}文件@
	\end{lstlisting}
	\item 如何探索 \inlineBash{ls} 的排序功能?
		\begin{itemize}
		\item 用 \inlineBash{nano} 在 \inlineBash{man ls} 中查找 ``\texttt{sort}''
		\end{itemize}
	\item 操控按键
	\begin{lstlisting}[basicstyle=\ttfamily]
	Ctrl + X	退出
	Ctrl + O	保存(输入路径)
	Ctrl + W	搜索
	Ctrl + \	替换
	\end{lstlisting}
	\end{itemize}
\end{frame}

\begin{frame} [fragile]
	\frametitle{\inlineBash{wget}}
	\linespread{1.25}
	\begin{itemize}
	\item \inlineBash{wget}可以下载网页
	\begin{lstlisting}[style=bashstyle, gobble=4, texcl]
	wget 网址
	wget 网址 -r		# 递归下载
	wget 网址 -rl 3		# 深度为3
	\end{lstlisting}
	\item 尝试
	\begin{lstlisting}[style=bashstyle, gobble=4, texcl, escapechar=@]
	# 如果 shiyiquan.net 无法访问,试试 shierquan.tk 或者别的网站
	wget shiyiquan.net
	wget shiyiquan.net -r -l 1
	@递归下载维基百科?@
	\end{lstlisting}
	\end{itemize}
\end{frame}

\begin{frame} [fragile]
	\frametitle{连线:使用这些命令可以干什么?}
	\linespread{1.5}
	\begin{columns}[T]
		\begin{column}[T]{.5\textwidth}
			\begin{itemize}
			\item 编辑文档(记事本)
			\item 下载网页(浏览器)
			\item 安装软件(控制面板?)
			\item 切换用户(无)
			\item (括号内为Windows中有相应功能的程序)
			\end{itemize}
		\end{column}
		\begin{column}[T]{.5\textwidth}
			\begin{itemize}
			\item \inlineBash{yum / dnf}
			\item \inlineBash{nano}
			\item \inlineBash{su / sudo}
			\item \inlineBash{wget}
			\end{itemize}
		\end{column}
	\end{columns}
\end{frame}

\PreLastFrame
\begin{frame}
	\centerline{\fontsize{32}{32}\selectfont 感谢参加此次活动}
\end{frame}

\newpage
\end{CJK}
\end{document}

