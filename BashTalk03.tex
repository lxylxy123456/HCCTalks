\documentclass
% [aspectratio=169]
{beamer}

\usepackage{CJKutf8}
\usepackage{listings}
\usepackage{tikz}
\usepackage{hyperref}
\usepackage{xcolor}
\usepackage{verbatim}
\usepackage{eso-pic}
\usepackage{courier}
\usepackage{textcomp}

\newcommand{\link}[1]{\href{#1}{#1}}

\newcommand{\insertGraph}[3]{

	\centerline{\includegraphics[scale=#1]{#2}} % 0/0

	\centerline{#3}

}

\newcommand{\HCCLogoSimp}{
	\begin{tikzpicture}[scale=0.1]
		\definecolor{_00ccff}{HTML}{00ccff}
\def\center{(54.9655,-19.629)}
\def\radius{7.924}
\fill [color=_00ccff]
	(55.371,-11.075) -- (57.286,-11.417) -- (55.693,-20.323) -- 
	(53.803,-19.976);
\begin{scope}
	% (47.042,-11.705) rectangle (62.889,-27.553)
	% (49.018,-13.682) rectangle (60.913,-25.576)
%	\clip
%		(59.4605,-11.563) -- (54.9665,-19.629) -- (53.3695,-11.511) -- 
%		(47.041,-11.511) -- (47.041,-27.553) -- (62.890,-27.553) -- 
%		(62.890,-11.511);
%	\fill [color=_00ccff, even odd rule]
%		\center circle [radius=5.9475]
%		(59.4605,-11.563) -- (53.3695,-11.511) -- (54.9665,-19.629)
%		\center circle [radius=\radius]; 
	% https://tex.stackexchange.com/questions/510281/tikz-fill-only-the-a-b-c
	\path \center circle [radius=\radius];
	\clip[overlay]
		(53.3695,-11.511) -- (59.4605,-11.563) -- (54.9665,-19.629) [rev];
	\clip[overlay, eo] \center circle [radius=5.9475, rev];
	\fill[color=_00ccff] \center circle [radius=\radius];
\end{scope}


	\end{tikzpicture}
}

\newcommand{\HCCLogoFull}{
	\begin{tikzpicture}[scale=0.05]
		\definecolor{_00ccff}{HTML}{00ccff}
\def\center{(54.9655,-19.629)}
\def\radius{7.924}
\fill [color=_00ccff]
	(55.371,-11.075) -- (57.286,-11.417) -- (55.693,-20.323) -- 
	(53.803,-19.976);
\begin{scope}
	% (47.042,-11.705) rectangle (62.889,-27.553)
	% (49.018,-13.682) rectangle (60.913,-25.576)
%	\clip
%		(59.4605,-11.563) -- (54.9665,-19.629) -- (53.3695,-11.511) -- 
%		(47.041,-11.511) -- (47.041,-27.553) -- (62.890,-27.553) -- 
%		(62.890,-11.511);
%	\fill [color=_00ccff, even odd rule]
%		\center circle [radius=5.9475]
%		(59.4605,-11.563) -- (53.3695,-11.511) -- (54.9665,-19.629)
%		\center circle [radius=\radius]; 
	% https://tex.stackexchange.com/questions/510281/tikz-fill-only-the-a-b-c
	\path \center circle [radius=\radius];
	\clip[overlay]
		(53.3695,-11.511) -- (59.4605,-11.563) -- (54.9665,-19.629) [rev];
	\clip[overlay, eo] \center circle [radius=5.9475, rev];
	\fill[color=_00ccff] \center circle [radius=\radius];
\end{scope}


		\definecolor{_004455}{HTML}{004455}
\definecolor{_006680}{HTML}{006680}
\definecolor{_0088aa}{HTML}{0088aa}
\definecolor{_00aad4}{HTML}{00aad4}
\fill [color=_00aad4] (49.0,-46.0) rectangle (64.0,-50.5);
\fill [color=_0088aa] (44.5,-35.5) rectangle (49.0,-50.5);
\fill [color=_006680] (44.5,-31.0) rectangle (64.0,-35.5);
\fill [color=_00aad4] (27.5,-46.0) rectangle (42.5,-50.5);
\fill [color=_0088aa] (22.5,-35.5) rectangle (27.5,-50.5);
\fill [color=_006680] (22.5,-31.0) rectangle (42.0,-35.5);
\fill [color=_0088aa] (16.0,-35.5) rectangle (20.5,-51.0);
\fill [color=_006680] (4.5,-31.0) rectangle (20.5,-35.5);
\fill [color=_004455] (0.0,-24.0) rectangle (4.5,-53.5);
% \draw [color=red] (0.0,0.0) rectangle (64,-64);

	\end{tikzpicture}
}

\setbeamercolor{background canvas}{bg=}

\newcommand{\PreFirstFrame}{
	\AddToShipoutPictureFG*{
		\AtPageLowerLeft{
			\put(\LenToUnit{0.05\paperwidth},\LenToUnit{0.1\paperheight}){
				\footnotesize
				这个指引文档在
				\href{https://creativecommons.org/licenses/by-sa/3.0/deed.zh}
				{知识共享 署名-相同方式共享 3.0协议}之条款下提供
			}
			\put(\LenToUnit{0.05\paperwidth},\LenToUnit{0.05\paperheight}){
				\footnotesize
				This guidance is available under the 
				\href{https://creativecommons.org/licenses/by-sa/3.0/}
				{Creative Commons Attribution-ShareAlike License}
			}
			\put(\LenToUnit{0.6\paperwidth},\LenToUnit{0.15\paperheight}){
				\HCCLogoFull
			}
		}
	}
}

\newcommand{\PostFirstFrame}{
	\AddToShipoutPictureBG{
		\AtPageLowerLeft{
			\put(\LenToUnit{0.8\paperwidth},\LenToUnit{0.15\paperheight}){
				\HCCLogoSimp
			}
		}
	}
}

\newcommand{\PreLastFrame}{
	\ClearShipoutPictureBG

	\AddToShipoutPictureFG*{
		\AtPageLowerLeft{
			\put(\LenToUnit{0.6\paperwidth},\LenToUnit{0.15\paperheight}){
				\HCCLogoFull
			}
		}
	}
}

% Note: this C style differs a lot from gedit's
\lstdefinestyle{cstyle}{
	language=c,
	basicstyle=\ttfamily,
	morekeywords={with},
	keywordstyle=\bfseries\color[HTML]{a52a2a},	
	commentstyle=\color[HTML]{0000ff},
	stringstyle=\color[HTML]{ff0bff},
	keywordstyle=[3]\color[HTML]{008a8c},
	alsoletter={0,1,2,3,4,5,6,7,8,9,.},
	morekeywords=[4]{0,1,2,100,999},
	keywordstyle=[4]\color[HTML]{ff0bff},
	upquote=true,
	breaklines=true,
}

\lstdefinestyle{pythonstyle}{
	language=python,
	basicstyle=\ttfamily,
	% frame=single,
	morekeywords={with,yield},
	keywordstyle=\bfseries\color[HTML]{a52a2a},	
	keywordstyle=[2]\color[HTML]{008a8c},
	commentstyle=\color[HTML]{0000ff},
	stringstyle=\color[HTML]{ff0bff},
	keywordstyle=[3]\color[HTML]{008a8c},
	alsoletter={0123456789.},
	morekeywords=[4]{False,True,
		0,1,2,3,4,5,6,7,8,9,10,11,12,13,15,17,19,16,20,24,27,31,32,33,34,35,38,
		45,56,60,64,81,95,97,99,100,123,243,256,400,512,576,729,999,1024,1234,
		1365,1366,2000,2187,2836,2957,3856,3857,5678,6561,9274,100000,1000000,
		0.5,3.14,3.4,
		0x1234,},
	keywordstyle=[4]\color[HTML]{ff0bff},
	upquote=true,
	breaklines=true,
	showstringspaces=false,
}

\lstset{
	tabsize=4,
	columns=fixed,
	extendedchars=false,
}

\newcommand{\inlinePython}{\lstinline[style=pythonstyle]}



\begin{document}

\PreFirstFrame
\begin{frame} [fragile]
	\centerline{\fontsize{42}{42}\selectfont Bash Talk 3}
\end{frame}
\PostFirstFrame

\begin{frame} [fragile]
	\frametitle{复习}
	\linespread{1.5}
	\begin{columns}[T]
		\begin{column}[T]{.5\textwidth}
			\begin{itemize}
			\item 什么命令可以……?
				\begin{itemize}
				\item 在文件中查找
				\item 得到一个文件的末尾
				\item 动态浏览文件
				\item 计算文件的哈希值
				\item 对比文件
				\item 统计文件大小
				\end{itemize}
			\end{itemize}
		\end{column}
		\begin{column}[T]{.5\textwidth}
			\begin{lstlisting}[style=bashstyle, gobble=12, texcl]
			grep	head	tail
			less	nano	vi
			cut		wc		md5sum
			diff	more
			hexdump
			sha1sum
			\end{lstlisting}
		\end{column}
	\end{columns}
\end{frame}

\begin{frame} [fragile]
	\frametitle{用命令管理系统}
	\begin{itemize}
	\item 命令
	\begin{lstlisting}[style=bashstyle, gobble=4, texcl]
	uname		# 系统版本
	top			# 系统信息
	ps -aux		# 列出进程
	sync		# 同步数据
	ifconfig	# 网络信息
	lspci		# 设备信息
	free		# 内存用量
	\end{lstlisting}
	\item 参数
	\begin{lstlisting}[style=bashstyle, gobble=4, texcl]
	uname -r	# 内核版本
	uname -a	# 列出所有
	free -h		# 可读性高
	\end{lstlisting}
	\item 提示:尝试在\inlineBash{top}中按方向键
	\end{itemize}
\end{frame}

\begin{frame} [fragile]
	\frametitle{数据流重导向}
	\begin{itemize}
	\item 符号
	\begin{lstlisting}[style=bashstyle, gobble=4, texcl, escapechar=@]
	@\color{Pink}程序@ < @\color{Pink}文件 @		# 指定 stdin 为文件
	@\color{Pink}程序@ << @\color{Pink}字符串@	# 指定 EOF 为字符串
	@\color{Pink}程序@ > @\color{Pink}文件 @		# 指定 stdout 为文件
	@\color{Pink}程序@ >> @\color{Pink}文件 @	# 追加模式
	@\color{Pink}程序@ 2> @\color{Pink}文件 @	# 指定 stderr
	@\color{Pink}程序@ &> @\color{Pink}文件 @	# 指定 stderr + stdout
	\end{lstlisting}
	\item 尝试
	\begin{lstlisting}[style=bashstyle, gobble=4, texcl]
	cd /tmp				# 确保此目录里有一些文件
	ls -l > a
	head -n3 < a > b
	tail << hcc			# 输入hcc以退出
	ls -l >> a
	ls -l > a
	\end{lstlisting}
	\end{itemize}
\end{frame}

\begin{frame} [fragile]
	\frametitle{和时间相关的命令}
	\linespread{1.25}
	\begin{itemize}
	\item 符号
	\begin{lstlisting}[style=bashstyle, gobble=4, texcl, escapechar=@]
	sleep @秒数@		# 停止工作一些时间
	time @命令@		# 记时器
	date @  @		# 显示时间
	\end{lstlisting}
	\item 尝试
	\begin{lstlisting}[style=bashstyle, gobble=4, texcl, escapechar=@]
	sleep 1
	sleep @\color{Pink}1m@
	time time time
	time ls
	date
	date --help
	\end{lstlisting}
	\end{itemize}
\end{frame}

\begin{frame} [fragile]
	\frametitle{符号}
	\linespread{0.9}
	\begin{itemize}
	\item 符号
	\begin{lstlisting}[style=bashstyle, gobble=4, texcl, escapechar=@]
	/ @或@ // @  @	 # 根目录
	.			  # 当前目录
	..			  # 父目录
	~			  # 家目录(波浪线,键盘上1左侧)
	* @和@ ? @  @	 # 通配符
	@\color{Blue}\#@			 # 注释
	@命令@ ; @命令@ @  @# 依次执行命令
	\end{lstlisting}
	\item 尝试
	\begin{lstlisting}[style=bashstyle, gobble=4, texcl, escapechar=@]
	ls / ; ls //
	ls /@\color{Pink}s*@		# s开头
	ls /@\color{Pink}s??@	   # s后面跟随两个字母
	ls .
	cd ..
	cd ~
	sleep 1 ; echo hcc
	\end{lstlisting}
	\end{itemize}
\end{frame}

\begin{frame} [fragile]
	\frametitle{更高级的符号}
	\linespread{1.5}
	\begin{itemize}
	\item 符号
	\begin{lstlisting}[style=bashstyle, gobble=4, texcl, escapechar=@]
	\ @    @		  # 命令换行
	@命令1@ && @命令2@	# 如果1正确则执行2
	@命令1@ || @命令2@	# 如果1错误则执行2
	\end{lstlisting}
	\item 参见《\href{http://cn.linux.vbird.org/linux\_basic/0320bash\_5.php}
					{鸟哥的Linux私房菜}》
		\begin{itemize}
		\item 以及在\href{http://cn.linux.vbird.org/linux\_basic/0320bash\_2.php}
						{这个页面}中查找``跳脱符号''
		\end{itemize}
	\end{itemize}
\end{frame}

\begin{frame} [fragile]
	\frametitle{Linux的目录}
	\linespread{1.25}
	\begin{itemize}
	\item 常见根挂载点上的目录
		(参见\href{https://en.wikipedia.org/wiki/Unix\_filesystem}{维基百科})
	\begin{lstlisting}[style=bashstyle, gobble=4, texcl,
						escapebegin=\obeyspaces]
	/bin	# binary    二进制文件(如ls)
	/dev	# device    设备文件
	/etc	# et cetera 设置和数据
	/home	# home      用户主目录
	/tmp	# temporary 临时目录
	/srv	# server    服务器数据
	/usr	# user      非关键数据和文件
	\end{lstlisting}
	\end{itemize}
\end{frame}

\begin{frame} [fragile]
	\frametitle{常用文件}
	\small
	\begin{itemize}
	\item \inlineBash{/dev}
		\begin{lstlisting}[style=bashstyle, gobble=8, texcl]
		null			# 黑洞(吞没数据)
		zero			# 白洞(输出\textquotesingle{\textbackslash}0\textquotesingle)
		random			# 输出随机字符(更随机)
		urandom			# 输出随机字符(更快)
		sda				# 磁盘(sdb, sdc, ...)
		sda1			# 磁盘sda的分区1(sda2, ...)
		sr0				# 光盘
		\end{lstlisting}
	\item \inlineBash{/etc}
		\begin{lstlisting}[style=bashstyle, gobble=8, texcl]
		passwd			# 用户登录方式
		shadow			# 用户密码
		hosts			# 修改DNS
		fstab			# 挂载点配置
		sudoers			# 配置管理员
		locale.conf		# 配置语言
		resolv.conf		# DNS服务器地址
		\end{lstlisting}
	\end{itemize}
\end{frame}

\begin{frame} [fragile]
	\frametitle{设备和挂载点}
	\linespread{1.5}
	\begin{columns}[T]
		\begin{column}[T]{.5\textwidth}
			挂载点和对应的设备
			\begin{lstlisting}[style=bashstyle, gobble=12, texcl]
			/			/dev/sda1
			/home		/dev/sda2
			/mnt/USB	/dev/sdb1
			/opt		/dev/sda3
			/srv		/dev/sda4
			/tmp		tmpfs
			\end{lstlisting}
			用\inlineBash{df}可以查看实际的挂载情况
		\end{column}
		\begin{column}[T]{.5\textwidth}
			Windows下
			\begin{lstlisting}[style=bashstyle, gobble=12, texcl, escapechar=@]
			@\color{Pink}C:@		@某个物理磁盘@
			@\color{Pink}D:@		@某个物理磁盘@
			@\color{Pink}E:@		@某个U盘@
			@\color{Pink}F:@		@某个U盘@
			@\color{Pink}G:@		@某个移动硬盘@
			\end{lstlisting}
			如何分清哪个设备对应哪个标卷?
		\end{column}
	\end{columns}
\end{frame}

\begin{frame} [fragile]
	\frametitle{回顾}
	\linespread{1.5}
	\begin{columns}[T]
		\begin{column}[T]{.5\textwidth}
			\begin{itemize}
			\item 查看进程信息
			\item 查看网络连接
			\item 将数据输出到文件
			\item 依次执行两个命令
			\item 访问父目录
			\end{itemize}
		\end{column}
		\begin{column}[T]{.5\textwidth}
			\begin{lstlisting}[style=bashstyle, gobble=12, texcl]
			*		?		/	//
			~		;		<	<<
			.		..		>	>>
			uname	top		ps -aux
			sleep	time	date
			sync	lspci
			free	ifconfig
			\end{lstlisting}
		\end{column}
	\end{columns}
\end{frame}

\PreLastFrame
\begin{frame}
	\centerline{\fontsize{32}{32}\selectfont 感谢参加此次活动}
\end{frame}

\newpage
\end{document}

