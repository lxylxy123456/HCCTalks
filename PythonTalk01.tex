\documentclass
% [aspectratio=169]
{beamer}

\usepackage{CJKutf8}
\usepackage{listings}
\usepackage{tikz}
\usepackage{hyperref}
\usepackage{xcolor}
\usepackage{verbatim}
\usepackage{eso-pic}
\usepackage{courier}
\usepackage{textcomp}

\newcommand{\link}[1]{\href{#1}{#1}}

\newcommand{\insertGraph}[3]{

	\centerline{\includegraphics[scale=#1]{#2}} % 0/0

	\centerline{#3}

}

\newcommand{\HCCLogoSimp}{
	\begin{tikzpicture}[scale=0.1]
		\definecolor{_00ccff}{HTML}{00ccff}
\def\center{(54.9655,-19.629)}
\def\radius{7.924}
\fill [color=_00ccff]
	(55.371,-11.075) -- (57.286,-11.417) -- (55.693,-20.323) -- 
	(53.803,-19.976);
\begin{scope}
	% (47.042,-11.705) rectangle (62.889,-27.553)
	% (49.018,-13.682) rectangle (60.913,-25.576)
%	\clip
%		(59.4605,-11.563) -- (54.9665,-19.629) -- (53.3695,-11.511) -- 
%		(47.041,-11.511) -- (47.041,-27.553) -- (62.890,-27.553) -- 
%		(62.890,-11.511);
%	\fill [color=_00ccff, even odd rule]
%		\center circle [radius=5.9475]
%		(59.4605,-11.563) -- (53.3695,-11.511) -- (54.9665,-19.629)
%		\center circle [radius=\radius]; 
	% https://tex.stackexchange.com/questions/510281/tikz-fill-only-the-a-b-c
	\path \center circle [radius=\radius];
	\clip[overlay]
		(53.3695,-11.511) -- (59.4605,-11.563) -- (54.9665,-19.629) [rev];
	\clip[overlay, eo] \center circle [radius=5.9475, rev];
	\fill[color=_00ccff] \center circle [radius=\radius];
\end{scope}


	\end{tikzpicture}
}

\newcommand{\HCCLogoFull}{
	\begin{tikzpicture}[scale=0.05]
		\definecolor{_00ccff}{HTML}{00ccff}
\def\center{(54.9655,-19.629)}
\def\radius{7.924}
\fill [color=_00ccff]
	(55.371,-11.075) -- (57.286,-11.417) -- (55.693,-20.323) -- 
	(53.803,-19.976);
\begin{scope}
	% (47.042,-11.705) rectangle (62.889,-27.553)
	% (49.018,-13.682) rectangle (60.913,-25.576)
%	\clip
%		(59.4605,-11.563) -- (54.9665,-19.629) -- (53.3695,-11.511) -- 
%		(47.041,-11.511) -- (47.041,-27.553) -- (62.890,-27.553) -- 
%		(62.890,-11.511);
%	\fill [color=_00ccff, even odd rule]
%		\center circle [radius=5.9475]
%		(59.4605,-11.563) -- (53.3695,-11.511) -- (54.9665,-19.629)
%		\center circle [radius=\radius]; 
	% https://tex.stackexchange.com/questions/510281/tikz-fill-only-the-a-b-c
	\path \center circle [radius=\radius];
	\clip[overlay]
		(53.3695,-11.511) -- (59.4605,-11.563) -- (54.9665,-19.629) [rev];
	\clip[overlay, eo] \center circle [radius=5.9475, rev];
	\fill[color=_00ccff] \center circle [radius=\radius];
\end{scope}


		\definecolor{_004455}{HTML}{004455}
\definecolor{_006680}{HTML}{006680}
\definecolor{_0088aa}{HTML}{0088aa}
\definecolor{_00aad4}{HTML}{00aad4}
\fill [color=_00aad4] (49.0,-46.0) rectangle (64.0,-50.5);
\fill [color=_0088aa] (44.5,-35.5) rectangle (49.0,-50.5);
\fill [color=_006680] (44.5,-31.0) rectangle (64.0,-35.5);
\fill [color=_00aad4] (27.5,-46.0) rectangle (42.5,-50.5);
\fill [color=_0088aa] (22.5,-35.5) rectangle (27.5,-50.5);
\fill [color=_006680] (22.5,-31.0) rectangle (42.0,-35.5);
\fill [color=_0088aa] (16.0,-35.5) rectangle (20.5,-51.0);
\fill [color=_006680] (4.5,-31.0) rectangle (20.5,-35.5);
\fill [color=_004455] (0.0,-24.0) rectangle (4.5,-53.5);
% \draw [color=red] (0.0,0.0) rectangle (64,-64);

	\end{tikzpicture}
}

\setbeamercolor{background canvas}{bg=}

\newcommand{\PreFirstFrame}{
	\AddToShipoutPictureFG*{
		\AtPageLowerLeft{
			\put(\LenToUnit{0.05\paperwidth},\LenToUnit{0.1\paperheight}){
				\footnotesize
				这个指引文档在
				\href{https://creativecommons.org/licenses/by-sa/3.0/deed.zh}
				{知识共享 署名-相同方式共享 3.0协议}之条款下提供
			}
			\put(\LenToUnit{0.05\paperwidth},\LenToUnit{0.05\paperheight}){
				\footnotesize
				This guidance is available under the 
				\href{https://creativecommons.org/licenses/by-sa/3.0/}
				{Creative Commons Attribution-ShareAlike License}
			}
			\put(\LenToUnit{0.6\paperwidth},\LenToUnit{0.15\paperheight}){
				\HCCLogoFull
			}
		}
	}
}

\newcommand{\PostFirstFrame}{
	\AddToShipoutPictureBG{
		\AtPageLowerLeft{
			\put(\LenToUnit{0.8\paperwidth},\LenToUnit{0.15\paperheight}){
				\HCCLogoSimp
			}
		}
	}
}

\newcommand{\PreLastFrame}{
	\ClearShipoutPictureBG

	\AddToShipoutPictureFG*{
		\AtPageLowerLeft{
			\put(\LenToUnit{0.6\paperwidth},\LenToUnit{0.15\paperheight}){
				\HCCLogoFull
			}
		}
	}
}

% Note: this C style differs a lot from gedit's
\lstdefinestyle{cstyle}{
	language=c,
	basicstyle=\ttfamily,
	morekeywords={with},
	keywordstyle=\bfseries\color[HTML]{a52a2a},	
	commentstyle=\color[HTML]{0000ff},
	stringstyle=\color[HTML]{ff0bff},
	keywordstyle=[3]\color[HTML]{008a8c},
	alsoletter={0,1,2,3,4,5,6,7,8,9,.},
	morekeywords=[4]{0,1,2,100,999},
	keywordstyle=[4]\color[HTML]{ff0bff},
	upquote=true,
	breaklines=true,
}

\lstdefinestyle{pythonstyle}{
	language=python,
	basicstyle=\ttfamily,
	% frame=single,
	morekeywords={with,yield},
	keywordstyle=\bfseries\color[HTML]{a52a2a},	
	keywordstyle=[2]\color[HTML]{008a8c},
	commentstyle=\color[HTML]{0000ff},
	stringstyle=\color[HTML]{ff0bff},
	keywordstyle=[3]\color[HTML]{008a8c},
	alsoletter={0123456789.},
	morekeywords=[4]{False,True,
		0,1,2,3,4,5,6,7,8,9,10,11,12,13,15,17,19,16,20,24,27,31,32,33,34,35,38,
		45,56,60,64,81,95,97,99,100,123,243,256,400,512,576,729,999,1024,1234,
		1365,1366,2000,2187,2836,2957,3856,3857,5678,6561,9274,100000,1000000,
		0.5,3.14,3.4,
		0x1234,},
	keywordstyle=[4]\color[HTML]{ff0bff},
	upquote=true,
	breaklines=true,
	showstringspaces=false,
}

\lstset{
	tabsize=4,
	columns=fixed,
	extendedchars=false,
}

\newcommand{\inlinePython}{\lstinline[style=pythonstyle]}



\begin{document}

\PreFirstFrame
\begin{frame} [fragile]
	\centerline{\fontsize{42}{42}\selectfont Python Talk 1}
\end{frame}
\PostFirstFrame

\begin{frame}
	\frametitle{Python 是什么?}
	\linespread{2}
	\begin{itemize}
	\item Python 是一个多用途的高级语言
	\item Python 可以用简单的代码完成复杂的工作
	\item 我们使用 Python3.7 进行编程
	\item 推荐使用 Python3 和更高版本
	\end{itemize}
\end{frame}

\begin{frame} [fragile]
	\frametitle{2的999次方最后两位是多少?}
	\begin{columns}[T]
		\begin{column}[T]{.2\textwidth}
			\begin{itemize}
			\item 计算器
			\begin{lstlisting}[style=pythonstyle, gobble=12]
			2^999
			\end{lstlisting}
			(不精确)
			\end{itemize}
		\end{column}
		\begin{column}[T]{.8\textwidth}
			\begin{itemize}
			\item Python 3
			\begin{lstlisting}[style=pythonstyle, gobble=12]
			print(2 ** 999 % 100)
			\end{lstlisting}
			\item C
			\begin{lstlisting}[style=cstyle, gobble=12]
			#include "stdio.h"
			int main(void) {
			  int last = 1;
			  for(int i = 0; i < 999; i += 1){
				last *= 2;
				last %= 100;
			  }
			  printf("%d\n", last);
			  return 0;
			}
			\end{lstlisting}
			\end{itemize}
		\end{column}
	\end{columns}
\end{frame}

\begin{frame} [fragile]
	\frametitle{基本计算}
	\begin{itemize}
	\item Python可以轻松完成基本的数学计算
	\item 像做数学题一样输入方程即可!
	\item 运算符
	\begin{lstlisting}[style=pythonstyle, gobble=4, texcl]
	+		# 加
	-		# 减
	*		# 乘
	/		# 除
	**		# 乘方
	\end{lstlisting}
	\end{itemize}
\end{frame}

\begin{frame} [fragile]
	\frametitle{计算以下数值}
	\begin{lstlisting}[style=pythonstyle, gobble=4,
						basicstyle=\linespread{2}\ttfamily]
	1234 + 5678 - 2836
	3856 * 9274 * 576
	2957 / 3857
	7 的 123 次方
	\end{lstlisting}
\end{frame}

\begin{frame} [fragile]
	\frametitle{Tips}
	\begin{itemize}
	\item 得到上次的结果
	\begin{lstlisting}[style=pythonstyle, gobble=4]
	3 + 7
	_ * 2
	\end{lstlisting}
	\item 这个符号是什么?
	\begin{lstlisting}[style=pythonstyle, gobble=4]
	3 // 5
	3 / 5
	\end{lstlisting}
	\item 求余数
	\begin{lstlisting}[style=pythonstyle, gobble=4, texcl]
	8 % 5   # 得到8÷5的余数
	# 负数会怎样?
	\end{lstlisting}
	\item 将结果保存
	\begin{lstlisting}[style=pythonstyle, gobble=4]
	5 + 8
	a = _
	\end{lstlisting}
	\end{itemize}
\end{frame}

\begin{frame} [fragile]
	\frametitle{数值比较}
	\begin{itemize}
	\item 使用这些运算符
	\begin{lstlisting}[style=pythonstyle, gobble=4, texcl,
						basicstyle=\linespread{1.5}\ttfamily]
	>		# 大于
	<		# 小于
	==		# 等于,注意不是 =
	!=		# 不等于
	>=		# 大于等于
	<=		# 小于等于
	\end{lstlisting}
	\end{itemize}
\end{frame}

\begin{frame} [fragile]
	\frametitle{逻辑运算}
	\begin{itemize}
	\item 数值比较的结果是真假值,如 \texttt{2 < 3} 为真;\texttt{4 < 3} 为假
	\begin{lstlisting}[style=pythonstyle, gobble=4, texcl,
						basicstyle=\linespread{1.5}\ttfamily]
	True	# 真
	False	# 假
	and		# 且 / 与
	or		# 或
	not		# 非
	\end{lstlisting}
	\end{itemize}
\end{frame}

\begin{frame} [fragile]
	\frametitle{试试看}
	\begin{itemize}
	\item 这些变量是真是假?
	\begin{lstlisting}[style=pythonstyle, gobble=4]
	0
	1
	'HCC'
	3.14
	\end{lstlisting}
	\item 判断方法
	\begin{lstlisting}[style=pythonstyle, gobble=4, texcl]
	a = 0
	a == True # 不对?
	bool(a)   # 正确
	# 继续尝试 a = 1 等
	\end{lstlisting}
	\end{itemize}
\end{frame}

\begin{frame} [fragile]
	\frametitle{Python类型}
	\begin{columns}[T]
		\begin{column}[T]{.6\textwidth}
			\begin{itemize}
			\item 常见基本类型
			\begin{lstlisting}[style=pythonstyle, gobble=12, texcl]
			int		# 整数
			str		# 字符串
			float	# 浮点数
			bool	# 真假值 / 布尔值
			\end{lstlisting}
			\item 如何查看一个变量的类型
			\begin{lstlisting}[style=pythonstyle, gobble=12, texcl]
			type(a)
			\end{lstlisting}
			\item 尝试
			\begin{lstlisting}[style=pythonstyle, gobble=12, texcl]
			int(3.4)
			help(int)
			\end{lstlisting}
			\end{itemize}
		\end{column}
		\begin{column}[T]{.4\textwidth}
			\begin{itemize}
			\item 判断以下变量的类型
			\begin{lstlisting}[style=pythonstyle, gobble=12, texcl]
			1
			'HCC'
			3.14
			True
			\end{lstlisting}
			\end{itemize}
		\end{column}
	\end{columns}
\end{frame}

\PreLastFrame
\begin{frame}
	\centerline{\fontsize{32}{32}\selectfont 感谢参加此次活动}
\end{frame}

\newpage
\end{document}

