\documentclass
% [aspectratio=169]
{beamer}

\usepackage{CJKutf8}
\usepackage{listings}
\usepackage{tikz}
\usepackage{hyperref}
\usepackage{xcolor}
\usepackage{verbatim}
\usepackage{eso-pic}
\usepackage{courier}
\usepackage{textcomp}

\newcommand{\link}[1]{\href{#1}{#1}}

\newcommand{\insertGraph}[3]{

	\centerline{\includegraphics[scale=#1]{#2}} % 0/0

	\centerline{#3}

}

\newcommand{\HCCLogoSimp}{
	\begin{tikzpicture}[scale=0.1]
		\definecolor{_00ccff}{HTML}{00ccff}
\def\center{(54.9655,-19.629)}
\def\radius{7.924}
\fill [color=_00ccff]
	(55.371,-11.075) -- (57.286,-11.417) -- (55.693,-20.323) -- 
	(53.803,-19.976);
\begin{scope}
	% (47.042,-11.705) rectangle (62.889,-27.553)
	% (49.018,-13.682) rectangle (60.913,-25.576)
%	\clip
%		(59.4605,-11.563) -- (54.9665,-19.629) -- (53.3695,-11.511) -- 
%		(47.041,-11.511) -- (47.041,-27.553) -- (62.890,-27.553) -- 
%		(62.890,-11.511);
%	\fill [color=_00ccff, even odd rule]
%		\center circle [radius=5.9475]
%		(59.4605,-11.563) -- (53.3695,-11.511) -- (54.9665,-19.629)
%		\center circle [radius=\radius]; 
	% https://tex.stackexchange.com/questions/510281/tikz-fill-only-the-a-b-c
	\path \center circle [radius=\radius];
	\clip[overlay]
		(53.3695,-11.511) -- (59.4605,-11.563) -- (54.9665,-19.629) [rev];
	\clip[overlay, eo] \center circle [radius=5.9475, rev];
	\fill[color=_00ccff] \center circle [radius=\radius];
\end{scope}


	\end{tikzpicture}
}

\newcommand{\HCCLogoFull}{
	\begin{tikzpicture}[scale=0.05]
		\definecolor{_00ccff}{HTML}{00ccff}
\def\center{(54.9655,-19.629)}
\def\radius{7.924}
\fill [color=_00ccff]
	(55.371,-11.075) -- (57.286,-11.417) -- (55.693,-20.323) -- 
	(53.803,-19.976);
\begin{scope}
	% (47.042,-11.705) rectangle (62.889,-27.553)
	% (49.018,-13.682) rectangle (60.913,-25.576)
%	\clip
%		(59.4605,-11.563) -- (54.9665,-19.629) -- (53.3695,-11.511) -- 
%		(47.041,-11.511) -- (47.041,-27.553) -- (62.890,-27.553) -- 
%		(62.890,-11.511);
%	\fill [color=_00ccff, even odd rule]
%		\center circle [radius=5.9475]
%		(59.4605,-11.563) -- (53.3695,-11.511) -- (54.9665,-19.629)
%		\center circle [radius=\radius]; 
	% https://tex.stackexchange.com/questions/510281/tikz-fill-only-the-a-b-c
	\path \center circle [radius=\radius];
	\clip[overlay]
		(53.3695,-11.511) -- (59.4605,-11.563) -- (54.9665,-19.629) [rev];
	\clip[overlay, eo] \center circle [radius=5.9475, rev];
	\fill[color=_00ccff] \center circle [radius=\radius];
\end{scope}


		\definecolor{_004455}{HTML}{004455}
\definecolor{_006680}{HTML}{006680}
\definecolor{_0088aa}{HTML}{0088aa}
\definecolor{_00aad4}{HTML}{00aad4}
\fill [color=_00aad4] (49.0,-46.0) rectangle (64.0,-50.5);
\fill [color=_0088aa] (44.5,-35.5) rectangle (49.0,-50.5);
\fill [color=_006680] (44.5,-31.0) rectangle (64.0,-35.5);
\fill [color=_00aad4] (27.5,-46.0) rectangle (42.5,-50.5);
\fill [color=_0088aa] (22.5,-35.5) rectangle (27.5,-50.5);
\fill [color=_006680] (22.5,-31.0) rectangle (42.0,-35.5);
\fill [color=_0088aa] (16.0,-35.5) rectangle (20.5,-51.0);
\fill [color=_006680] (4.5,-31.0) rectangle (20.5,-35.5);
\fill [color=_004455] (0.0,-24.0) rectangle (4.5,-53.5);
% \draw [color=red] (0.0,0.0) rectangle (64,-64);

	\end{tikzpicture}
}

\setbeamercolor{background canvas}{bg=}

\newcommand{\PreFirstFrame}{
	\AddToShipoutPictureFG*{
		\AtPageLowerLeft{
			\put(\LenToUnit{0.05\paperwidth},\LenToUnit{0.1\paperheight}){
				\footnotesize
				这个指引文档在
				\href{https://creativecommons.org/licenses/by-sa/3.0/deed.zh}
				{知识共享 署名-相同方式共享 3.0协议}之条款下提供
			}
			\put(\LenToUnit{0.05\paperwidth},\LenToUnit{0.05\paperheight}){
				\footnotesize
				This guidance is available under the 
				\href{https://creativecommons.org/licenses/by-sa/3.0/}
				{Creative Commons Attribution-ShareAlike License}
			}
			\put(\LenToUnit{0.6\paperwidth},\LenToUnit{0.15\paperheight}){
				\HCCLogoFull
			}
		}
	}
}

\newcommand{\PostFirstFrame}{
	\AddToShipoutPictureBG{
		\AtPageLowerLeft{
			\put(\LenToUnit{0.8\paperwidth},\LenToUnit{0.15\paperheight}){
				\HCCLogoSimp
			}
		}
	}
}

\newcommand{\PreLastFrame}{
	\ClearShipoutPictureBG

	\AddToShipoutPictureFG*{
		\AtPageLowerLeft{
			\put(\LenToUnit{0.6\paperwidth},\LenToUnit{0.15\paperheight}){
				\HCCLogoFull
			}
		}
	}
}

% Note: this C style differs a lot from gedit's
\lstdefinestyle{cstyle}{
	language=c,
	basicstyle=\ttfamily,
	morekeywords={with},
	keywordstyle=\bfseries\color[HTML]{a52a2a},	
	commentstyle=\color[HTML]{0000ff},
	stringstyle=\color[HTML]{ff0bff},
	keywordstyle=[3]\color[HTML]{008a8c},
	alsoletter={0,1,2,3,4,5,6,7,8,9,.},
	morekeywords=[4]{0,1,2,100,999},
	keywordstyle=[4]\color[HTML]{ff0bff},
	upquote=true,
	breaklines=true,
}

\lstdefinestyle{pythonstyle}{
	language=python,
	basicstyle=\ttfamily,
	% frame=single,
	morekeywords={with,yield},
	keywordstyle=\bfseries\color[HTML]{a52a2a},	
	keywordstyle=[2]\color[HTML]{008a8c},
	commentstyle=\color[HTML]{0000ff},
	stringstyle=\color[HTML]{ff0bff},
	keywordstyle=[3]\color[HTML]{008a8c},
	alsoletter={0123456789.},
	morekeywords=[4]{False,True,
		0,1,2,3,4,5,6,7,8,9,10,11,12,13,15,17,19,16,20,24,27,31,32,33,34,35,38,
		45,56,60,64,81,95,97,99,100,123,243,256,400,512,576,729,999,1024,1234,
		1365,1366,2000,2187,2836,2957,3856,3857,5678,6561,9274,100000,1000000,
		0.5,3.14,3.4,
		0x1234,},
	keywordstyle=[4]\color[HTML]{ff0bff},
	upquote=true,
	breaklines=true,
	showstringspaces=false,
}

\lstset{
	tabsize=4,
	columns=fixed,
	extendedchars=false,
}

\newcommand{\inlinePython}{\lstinline[style=pythonstyle]}



\begin{document}
\begin{CJK}{UTF8}{gbsn}

\PreFirstFrame
\begin{frame} [fragile]
	\centerline{\fontsize{42}{42}\selectfont Bash Talk 2}
\end{frame}
\PostFirstFrame

\begin{frame} [fragile]
	\frametitle{复习}
	\linespread{1.5}
	\begin{columns}[T]
		\begin{column}[T]{.5\textwidth}
			\begin{itemize}
			\item 什么命令可以……?
				\begin{itemize}
				\item 查看目录树和其他信息
				\item 查看磁盘空间
				\item 复制、移动和删除文件
				\item 创建目录
				\item 切换工作目录
				\end{itemize}
			\end{itemize}
		\end{column}
		\begin{column}[T]{.5\textwidth}
			\begin{lstlisting}[style=bashstyle, gobble=12, texcl]
			ll		ls		pwd
			du		df		tree
			cat		rm		mkdir
			mv		cp		find
			cd
			\end{lstlisting}
		\end{column}
	\end{columns}
\end{frame}

\begin{frame} [fragile]
	\frametitle{处理文件}
	\linespread{1.5}
	\begin{itemize}
	\item 在这次Bash Talk中我们将学习如何处理文件
	\begin{lstlisting}[style=bashstyle, gobble=4, texcl, escapechar=@]
	@查找@	grep
	@截断@	head	tail	cut
	@统计@	wc		md5sum	sha1sum
	@查看@	more	less	nano	vi
	@转换@	hexdump
	@比较@	diff
	\end{lstlisting}
	\end{itemize}
\end{frame}

\begin{frame} [fragile]
	\frametitle{查找和截断}
	\linespread{1.25}
	\begin{itemize}
	\item 命令
	\begin{lstlisting}[style=bashstyle, gobble=4, texcl, escapechar=@]
	grep @单词@ @\color[HTML]{ff0bff}文件  @		 # 查找单词
	head @\color[HTML]{ff0bff}文件    @		 # 获得文件开头
	tail @\color[HTML]{ff0bff}文件    @		 # 获得文件结尾
	cut -b @开始-结束@ @\color[HTML]{ff0bff}文件@	# 截取文件的某些列
	\end{lstlisting}
	\item 尝试
	\begin{lstlisting}[style=bashstyle, gobble=4, texcl, escapechar=@]
	ls --help > a
	cat a
	grep sort=WORD a
	head a
	tail a
	cut -b 10-40 a
	\end{lstlisting}
	\end{itemize}
\end{frame}

\begin{frame} [fragile]
	\frametitle{关于文件输入输出}
	\linespread{1.5}
	\begin{itemize}
	\item 所有的命令都支持多文件
	\item 在 \inlineBash{man} 手册或者帮助中会有提示,例如
		\begin{itemize}
		\item \inlineBash{more [file ...]}
		\end{itemize}
	\item 如果没有给定文件,将会用 \inlineBash{stdin} 作为输入
		\begin{itemize}
		\item 文件 \inlineBash{-} 可以代表 \inlineBash{stdin} 来进行处理
		\end{itemize}
	\item 大多命令的输出是 \inlineBash{stdout}
	\item 所有命令的错误数据会被转到 \inlineBash{stderr}
	\end{itemize}
\end{frame}

\begin{frame} [fragile]
	\frametitle{标准输入输出}
	\linespread{1.25}
	\begin{columns}[T]
		\begin{column}[T]{.5\textwidth}
			\includesvg[scale=0.4]{asserts/Stdstreams-notitle}
		\end{column}
		\begin{column}[T]{.5\textwidth}
			\begin{itemize}
			\item 标准输入输出
				\begin{lstlisting}[style=bashstyle, gobble=16, texcl]
				stdin	标准输入
				stdout	标准输出
				stderr	标准错误
				\end{lstlisting}
			\item 文件也可用作输入输出
			\end{itemize}
		\end{column}
	\end{columns}
\end{frame}

\begin{frame} [fragile]
	\frametitle{统计}
	\linespread{1.25}
	\begin{itemize}
	\item 命令
	\begin{lstlisting}[style=bashstyle, gobble=4, texcl, escapechar=@]
	wc @\color[HTML]{ff0bff}文件@			# 统计文件行数、词数、字数
	md5sum @\color[HTML]{ff0bff}文件@		# 得到文件的md5哈希值
	sha1sum @\color[HTML]{ff0bff}文件@	# 得到文件的sha1哈希值
	\end{lstlisting}
	\item 这些命令只会读取文件
	\item wc的全称是什么?
	\item 你能发现输出的规律吗?
	\item 如果还有时间,尝试 
	\begin{lstlisting}[style=bashstyle, gobble=4, texcl, escapechar=@]
	sha224sum		sha256sum
	sha384sum		sha512sum
	shasum
	\end{lstlisting}
	\end{itemize}
\end{frame}

\begin{frame} [fragile]
	\frametitle{管线}
	\linespread{1.5}
	\begin{columns}[T]
		\begin{column}[T]{.5\textwidth}
			\includesvg[scale=0.4]{asserts/Pipeline}
		\end{column}
		\begin{column}[T]{.5\textwidth}
			\begin{itemize}
			\item 管线(\href{https://en.wikipedia.org/wiki/Pipeline\_(Unix)}
							{pipeline})可以将一个程序的 \inlineBash{stdout}
							接到另一个程序的 \inlineBash{stdin} 上
				\begin{itemize}
				\item \inlineBash{pr1 | pr2 | pr3}
				\item \inlineBash{ls --help | wc}
				\end{itemize}
			\item 连接数量几乎没有限制
			\end{itemize}
		\end{column}
	\end{columns}
\end{frame}

\begin{frame} [fragile]
	\frametitle{查看和编辑}
	\linespread{1.25}
	\begin{itemize}
	\item 命令
	\begin{lstlisting}[style=bashstyle, gobble=4, texcl, escapechar=@]
	more @\color[HTML]{ff0bff}文件@	# 查看
	less @\color[HTML]{ff0bff}文件@	# 查看
	nano @\color[HTML]{ff0bff}文件@	# 编辑
	vi @\color[HTML]{ff0bff}文件@		# 编辑
	\end{lstlisting}
	\item 尝试用方向、翻页键和Home、End控制
		\begin{itemize}
		\item 如果没东西了尝试一直按下左方向键
		\end{itemize}
	\item 按 \texttt{q} 可以退出
		\begin{itemize}
		\item \inlineBash{vi}是 \texttt{:q}
		\end{itemize}
	\item 参数
		\begin{itemize}
		\item \inlineBash{journalctl | less -S}
		\end{itemize}
	\end{itemize}
\end{frame}

\begin{frame} [fragile]
	\frametitle{转换和比较}
	\linespread{1.25}
	\begin{itemize}
	\item 转换
		\begin{lstlisting}[style=bashstyle, gobble=8, texcl, escapechar=@]
		hexdump @\color[HTML]{ff0bff}文件@		# 将二进制文件转换为可读形式
		\end{lstlisting}
	\item 比较
		\begin{lstlisting}[style=bashstyle, gobble=8, texcl, escapechar=@]
		diff @\color[HTML]{ff0bff}文件1@ @\color[HTML]{ff0bff}文件2@
		\end{lstlisting}
	\item 尝试
		\begin{lstlisting}[style=bashstyle, gobble=8, texcl, escapechar=@]
		cat /dev/urandom
		hexdump /dev/urandom
		ls -la > a
		mkdir jkl
		ls -l > b
		diff a b
		\end{lstlisting}
	\end{itemize}
\end{frame}

\begin{frame} [fragile]
	\frametitle{这些命令可以干什么}
	\linespread{1.5}
	\begin{columns}[T]
		\begin{column}[T]{.5\textwidth}
			\begin{itemize}
			\item 在文件中查找
			\item 得到一个文件的末尾
			\item 动态浏览文件
			\item 计算文件的哈希值
			\item 对比文件
			\item 统计文件大小
			\end{itemize}
		\end{column}
		\begin{column}[T]{.5\textwidth}
			\begin{lstlisting}[style=bashstyle, gobble=12, texcl]
			grep	head	tail
			less	nano	vi
			cut		wc		md5sum
			diff	more
			hexdump
			sha1sum
			\end{lstlisting}
		\end{column}
	\end{columns}
\end{frame}

\begin{frame} [fragile]
	\frametitle{Windows可以干什么}
	\linespread{1.5}
	\begin{columns}[T]
		\begin{column}[T]{.5\textwidth}
			\begin{itemize}
			\item 在文件中查找 (\texttt{find})
			\item {\color{gray}得到一个文件的末尾}
			\item {\color{gray}动态浏览文件}
			\item {\color{gray}计算文件的哈希值}
			\item 对比文件 (\texttt{fc})
			\item {\color{gray}统计文件大小}
			\end{itemize}
		\end{column}
		\begin{column}[T]{.5\textwidth}
			\begin{lstlisting}[style=bashstyle, gobble=12, texcl, escapechar=@]
			grep	@{\color{gray}head}@	@{\color{gray}tail}@
			@{\color{gray}less}@	@{\color{gray}nano}@	@{\color{gray}vi}@
			@{\color{gray}cut}@	 @{\color{gray}wc}@	   @{\color{gray}md5sum}@
			diff	@{\color{gray}more}@
			@{\color{gray}hexdump}@
			@{\color{gray}sha1sum}@
			\end{lstlisting}
		\end{column}
	\end{columns}
\end{frame}

\PreLastFrame
\begin{frame}
	\centerline{\fontsize{32}{32}\selectfont 感谢参加此次活动}
\end{frame}

\newpage
\end{CJK}
\end{document}

