\documentclass
% [aspectratio=169]
{beamer}

\usepackage{CJKutf8}
\usepackage{listings}
\usepackage{tikz}
\usepackage{hyperref}
\usepackage{xcolor}
\usepackage{verbatim}
\usepackage{eso-pic}
\usepackage{courier}
\usepackage{textcomp}

\newcommand{\link}[1]{\href{#1}{#1}}

\newcommand{\insertGraph}[3]{

	\centerline{\includegraphics[scale=#1]{#2}} % 0/0

	\centerline{#3}

}

\newcommand{\HCCLogoSimp}{
	\begin{tikzpicture}[scale=0.1]
		\definecolor{_00ccff}{HTML}{00ccff}
\def\center{(54.9655,-19.629)}
\def\radius{7.924}
\fill [color=_00ccff]
	(55.371,-11.075) -- (57.286,-11.417) -- (55.693,-20.323) -- 
	(53.803,-19.976);
\begin{scope}
	% (47.042,-11.705) rectangle (62.889,-27.553)
	% (49.018,-13.682) rectangle (60.913,-25.576)
%	\clip
%		(59.4605,-11.563) -- (54.9665,-19.629) -- (53.3695,-11.511) -- 
%		(47.041,-11.511) -- (47.041,-27.553) -- (62.890,-27.553) -- 
%		(62.890,-11.511);
%	\fill [color=_00ccff, even odd rule]
%		\center circle [radius=5.9475]
%		(59.4605,-11.563) -- (53.3695,-11.511) -- (54.9665,-19.629)
%		\center circle [radius=\radius]; 
	% https://tex.stackexchange.com/questions/510281/tikz-fill-only-the-a-b-c
	\path \center circle [radius=\radius];
	\clip[overlay]
		(53.3695,-11.511) -- (59.4605,-11.563) -- (54.9665,-19.629) [rev];
	\clip[overlay, eo] \center circle [radius=5.9475, rev];
	\fill[color=_00ccff] \center circle [radius=\radius];
\end{scope}


	\end{tikzpicture}
}

\newcommand{\HCCLogoFull}{
	\begin{tikzpicture}[scale=0.05]
		\definecolor{_00ccff}{HTML}{00ccff}
\def\center{(54.9655,-19.629)}
\def\radius{7.924}
\fill [color=_00ccff]
	(55.371,-11.075) -- (57.286,-11.417) -- (55.693,-20.323) -- 
	(53.803,-19.976);
\begin{scope}
	% (47.042,-11.705) rectangle (62.889,-27.553)
	% (49.018,-13.682) rectangle (60.913,-25.576)
%	\clip
%		(59.4605,-11.563) -- (54.9665,-19.629) -- (53.3695,-11.511) -- 
%		(47.041,-11.511) -- (47.041,-27.553) -- (62.890,-27.553) -- 
%		(62.890,-11.511);
%	\fill [color=_00ccff, even odd rule]
%		\center circle [radius=5.9475]
%		(59.4605,-11.563) -- (53.3695,-11.511) -- (54.9665,-19.629)
%		\center circle [radius=\radius]; 
	% https://tex.stackexchange.com/questions/510281/tikz-fill-only-the-a-b-c
	\path \center circle [radius=\radius];
	\clip[overlay]
		(53.3695,-11.511) -- (59.4605,-11.563) -- (54.9665,-19.629) [rev];
	\clip[overlay, eo] \center circle [radius=5.9475, rev];
	\fill[color=_00ccff] \center circle [radius=\radius];
\end{scope}


		\definecolor{_004455}{HTML}{004455}
\definecolor{_006680}{HTML}{006680}
\definecolor{_0088aa}{HTML}{0088aa}
\definecolor{_00aad4}{HTML}{00aad4}
\fill [color=_00aad4] (49.0,-46.0) rectangle (64.0,-50.5);
\fill [color=_0088aa] (44.5,-35.5) rectangle (49.0,-50.5);
\fill [color=_006680] (44.5,-31.0) rectangle (64.0,-35.5);
\fill [color=_00aad4] (27.5,-46.0) rectangle (42.5,-50.5);
\fill [color=_0088aa] (22.5,-35.5) rectangle (27.5,-50.5);
\fill [color=_006680] (22.5,-31.0) rectangle (42.0,-35.5);
\fill [color=_0088aa] (16.0,-35.5) rectangle (20.5,-51.0);
\fill [color=_006680] (4.5,-31.0) rectangle (20.5,-35.5);
\fill [color=_004455] (0.0,-24.0) rectangle (4.5,-53.5);
% \draw [color=red] (0.0,0.0) rectangle (64,-64);

	\end{tikzpicture}
}

\setbeamercolor{background canvas}{bg=}

\newcommand{\PreFirstFrame}{
	\AddToShipoutPictureFG*{
		\AtPageLowerLeft{
			\put(\LenToUnit{0.05\paperwidth},\LenToUnit{0.1\paperheight}){
				\footnotesize
				这个指引文档在
				\href{https://creativecommons.org/licenses/by-sa/3.0/deed.zh}
				{知识共享 署名-相同方式共享 3.0协议}之条款下提供
			}
			\put(\LenToUnit{0.05\paperwidth},\LenToUnit{0.05\paperheight}){
				\footnotesize
				This guidance is available under the 
				\href{https://creativecommons.org/licenses/by-sa/3.0/}
				{Creative Commons Attribution-ShareAlike License}
			}
			\put(\LenToUnit{0.6\paperwidth},\LenToUnit{0.15\paperheight}){
				\HCCLogoFull
			}
		}
	}
}

\newcommand{\PostFirstFrame}{
	\AddToShipoutPictureBG{
		\AtPageLowerLeft{
			\put(\LenToUnit{0.8\paperwidth},\LenToUnit{0.15\paperheight}){
				\HCCLogoSimp
			}
		}
	}
}

\newcommand{\PreLastFrame}{
	\ClearShipoutPictureBG

	\AddToShipoutPictureFG*{
		\AtPageLowerLeft{
			\put(\LenToUnit{0.6\paperwidth},\LenToUnit{0.15\paperheight}){
				\HCCLogoFull
			}
		}
	}
}

% Note: this C style differs a lot from gedit's
\lstdefinestyle{cstyle}{
	language=c,
	basicstyle=\ttfamily,
	morekeywords={with},
	keywordstyle=\bfseries\color[HTML]{a52a2a},	
	commentstyle=\color[HTML]{0000ff},
	stringstyle=\color[HTML]{ff0bff},
	keywordstyle=[3]\color[HTML]{008a8c},
	alsoletter={0,1,2,3,4,5,6,7,8,9,.},
	morekeywords=[4]{0,1,2,100,999},
	keywordstyle=[4]\color[HTML]{ff0bff},
	upquote=true,
	breaklines=true,
}

\lstdefinestyle{pythonstyle}{
	language=python,
	basicstyle=\ttfamily,
	% frame=single,
	morekeywords={with,yield},
	keywordstyle=\bfseries\color[HTML]{a52a2a},	
	keywordstyle=[2]\color[HTML]{008a8c},
	commentstyle=\color[HTML]{0000ff},
	stringstyle=\color[HTML]{ff0bff},
	keywordstyle=[3]\color[HTML]{008a8c},
	alsoletter={0123456789.},
	morekeywords=[4]{False,True,
		0,1,2,3,4,5,6,7,8,9,10,11,12,13,15,17,19,16,20,24,27,31,32,33,34,35,38,
		45,56,60,64,81,95,97,99,100,123,243,256,400,512,576,729,999,1024,1234,
		1365,1366,2000,2187,2836,2957,3856,3857,5678,6561,9274,100000,1000000,
		0.5,3.14,3.4,
		0x1234,},
	keywordstyle=[4]\color[HTML]{ff0bff},
	upquote=true,
	breaklines=true,
	showstringspaces=false,
}

\lstset{
	tabsize=4,
	columns=fixed,
	extendedchars=false,
}

\newcommand{\inlinePython}{\lstinline[style=pythonstyle]}



\begin{document}
\begin{CJK}{UTF8}{gbsn}

\PreFirstFrame
\begin{frame} [fragile]
	\centerline{\fontsize{42}{42}\selectfont Python Talk 5}
\end{frame}
\PostFirstFrame

\begin{frame} [fragile]
	\frametitle{复习}
	\linespread{1.5}
	\begin{itemize}
	\item 输出正整数a是否是平方数和是否是立方数
		\begin{itemize}
		\item 提示:平方根的运算结果是浮点数,不能直接和整数比较
		\end{itemize}
	\end{itemize}
\end{frame}

\begin{frame} [fragile]
	\frametitle{\inlinePython{print}}
	\linespread{1.25}
	\begin{lstlisting}[style=pythonstyle, gobble=4]
	print(value, ..., sep=' ', end='\n', file=sys.stdout, flush=False)}
	\end{lstlisting}
	\
	\begin{columns}[T]
		\begin{column}[T]{.5\textwidth}
			\begin{itemize}
			\item 通过
				\inlinePython{help(print)}
				获得\inlinePython{print}的使用说明(上方)
			\item \inlinePython{end='\n'}
				代表名为\inlinePython{end}的参数的默认值是\inlinePython{'\n'}
			\item 按q退出help
			\end{itemize}
		\end{column}
		\begin{column}[T]{.5\textwidth}
			参数说明
			\begin{lstlisting}[style=pythonstyle, gobble=12]
			value	打印的数值
			...		后面可以有很多位
			sep		分割的字符
			end		结尾的字符
			file 	打印的文件
			flush	是否直接输出
			\end{lstlisting}
		\end{column}
	\end{columns}
\end{frame}

\begin{frame} [fragile]
	\frametitle{\inlinePython{print}实验}
	\begin{columns}[T]
		\begin{column}[T]{.5\textwidth}
			\begin{lstlisting}[style=pythonstyle, gobble=12]
			import time
			while 1 :
				print('a', 'b')
				time.sleep(0.5)

			while 1 :
				print('a', end='')
				time.sleep(0.5)

			while 1 :
				print('a', end='', flush=True)
				time.sleep(0.5)
			\end{lstlisting}
		\end{column}
		\begin{column}[T]{.5\textwidth}
			\linespread{2}
			\begin{itemize}
			\item \inlinePython{while 1} 代表无限循环
				\begin{itemize}
				\item 按下\inlinePython{Crtl + C}停止
				\end{itemize}
			\item 没有换行时\inlinePython{print}不输出
				\begin{itemize}
				\item 通过\inlinePython{flush}解决
				\end{itemize}
			\end{itemize}
		\end{column}
	\end{columns}
\end{frame}

\begin{frame} [fragile]
	\frametitle{方便的函数}
	\linespread{1.25}
	\begin{itemize}
	\item 传入\inlinePython{tuple}或\inlinePython{list}
		\begin{itemize}
		\item \inlinePython{all			any}
		\item \inlinePython{max			min}
		\item e.g. \inlinePython{all([0, 1, 2])}
		\item 别忘了\inlinePython{help}
		\end{itemize}
	\item 传入任何对象
		\begin{itemize}
		\item \inlinePython{id			hash}
		\end{itemize}
	\item 更多信息:\link{https://docs.python.org/3/library/functions.html}
	\end{itemize}
\end{frame}

\begin{frame} [fragile]
	\frametitle{\inlinePython{input}}
	\linespread{1.25}
	\begin{itemize}
	\item 语法
		\begin{itemize}
		\item \inlinePython{input()} 或 \inlinePython{input(prompt)}
		\item 然后输入一行文字
		\end{itemize}
	\item 作用
		\begin{itemize}
		\item 读入输入用户的值
		\item 可以在前面加一条提示
		\end{itemize}
	\item 示例
		\begin{lstlisting}[style=pythonstyle, gobble=8, texcl, escapechar=@]
		a = input('@\color[HTML]{ff0bff}输入一个整数:@')	# str
		b = int(a)				   # int
		print('@\color[HTML]{ff0bff}这个整数减1是@' + str(b - 1))
		\end{lstlisting}
	\end{itemize}
\end{frame}

\begin{frame} [fragile]
	\frametitle{\inlinePython{open}}
	\linespread{1.25}
	\begin{itemize}
	\item \inlinePython{open}可以打开文件
		\begin{lstlisting}[style=pythonstyle, gobble=8, texcl]
		a = open('a.txt','w')	# 用写入模式打开文件
		a.write('hello')		# 写入文字
		a.close()				# 关闭文件
		open('a.txt').read()	# 打开文件并读取
		\end{lstlisting}
	\end{itemize}
\end{frame}

\begin{frame} [fragile]
	\frametitle{\inlinePython{os.system}}
	\linespread{1.25}
	\begin{itemize}
	\item \inlinePython{os}是一个和系统有关的模块
		\begin{itemize}
		\item 用 \inlinePython{import os} 引用
		\item 试一试 \inlinePython{help(os)}
		\end{itemize}
	\item 常用命令
	\begin{lstlisting}[style=pythonstyle, gobble=4, texcl]
	os.system('ls')	# ls 命令列出当前目录的文件
	os.walk('.')	# . 是一个目录
	\end{lstlisting}
	\item 更多:\link{https://docs.python.org/3/library/os.html}
	\end{itemize}
\end{frame}

\begin{frame} [fragile]
	\frametitle{\inlinePython{open}和\inlinePython{os}练习}
	\begin{enumerate}
	\item 用\inlinePython{open}创建以下10个文件 \\
		\texttt{1.txt, 2.txt, ..., 10.txt}
	\item 用\inlinePython{os}将他们更名为 \\
		\texttt{1.out, 2.out, ..., 10.out}
	\item 将他们删除
	\end{enumerate}
	\begin{itemize}
	\item Windows下获取命令提示
		\begin{itemize}
		\item \texttt{rename /?}
		\item \texttt{erase /?}
		\end{itemize}
	\item Linux下获取命令提示
		\begin{itemize}
		\item \texttt{mv --help}
		\item \texttt{rm --help}
		\end{itemize}
	\end{itemize}
\end{frame}

\begin{frame} [fragile]
	\frametitle{\inlinePython{in}}
	\linespread{1.25}
	\begin{itemize}
	\item 尝试这些代码,尝试总结\inlinePython{in}的作用
		\begin{lstlisting}[style=pythonstyle, gobble=8]
		a = "HCC, I'm. "
		'HCC' in a
		'hcc' in a
		3 in [1, 3, 5]
		3 in [3]
		[3] in [3]
		b = {1: 2, 2: 3}
		2 in b
		3 in b
		\end{lstlisting}
	\end{itemize}
\end{frame}

\PreLastFrame
\begin{frame}
	\centerline{\fontsize{32}{32}\selectfont 感谢参加此次活动}
\end{frame}

\newpage
\end{CJK}
\end{document}

