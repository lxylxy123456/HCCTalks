\documentclass
% [aspectratio=169]
{beamer}

\usepackage{CJKutf8}
\usepackage{listings}
\usepackage{tikz}
\usepackage{hyperref}
\usepackage{xcolor}
\usepackage{verbatim}
\usepackage{eso-pic}
\usepackage{courier}
\usepackage{textcomp}

\newcommand{\link}[1]{\href{#1}{#1}}

\newcommand{\insertGraph}[3]{

	\centerline{\includegraphics[scale=#1]{#2}} % 0/0

	\centerline{#3}

}

\newcommand{\HCCLogoSimp}{
	\begin{tikzpicture}[scale=0.1]
		\definecolor{_00ccff}{HTML}{00ccff}
\def\center{(54.9655,-19.629)}
\def\radius{7.924}
\fill [color=_00ccff]
	(55.371,-11.075) -- (57.286,-11.417) -- (55.693,-20.323) -- 
	(53.803,-19.976);
\begin{scope}
	% (47.042,-11.705) rectangle (62.889,-27.553)
	% (49.018,-13.682) rectangle (60.913,-25.576)
%	\clip
%		(59.4605,-11.563) -- (54.9665,-19.629) -- (53.3695,-11.511) -- 
%		(47.041,-11.511) -- (47.041,-27.553) -- (62.890,-27.553) -- 
%		(62.890,-11.511);
%	\fill [color=_00ccff, even odd rule]
%		\center circle [radius=5.9475]
%		(59.4605,-11.563) -- (53.3695,-11.511) -- (54.9665,-19.629)
%		\center circle [radius=\radius]; 
	% https://tex.stackexchange.com/questions/510281/tikz-fill-only-the-a-b-c
	\path \center circle [radius=\radius];
	\clip[overlay]
		(53.3695,-11.511) -- (59.4605,-11.563) -- (54.9665,-19.629) [rev];
	\clip[overlay, eo] \center circle [radius=5.9475, rev];
	\fill[color=_00ccff] \center circle [radius=\radius];
\end{scope}


	\end{tikzpicture}
}

\newcommand{\HCCLogoFull}{
	\begin{tikzpicture}[scale=0.05]
		\definecolor{_00ccff}{HTML}{00ccff}
\def\center{(54.9655,-19.629)}
\def\radius{7.924}
\fill [color=_00ccff]
	(55.371,-11.075) -- (57.286,-11.417) -- (55.693,-20.323) -- 
	(53.803,-19.976);
\begin{scope}
	% (47.042,-11.705) rectangle (62.889,-27.553)
	% (49.018,-13.682) rectangle (60.913,-25.576)
%	\clip
%		(59.4605,-11.563) -- (54.9665,-19.629) -- (53.3695,-11.511) -- 
%		(47.041,-11.511) -- (47.041,-27.553) -- (62.890,-27.553) -- 
%		(62.890,-11.511);
%	\fill [color=_00ccff, even odd rule]
%		\center circle [radius=5.9475]
%		(59.4605,-11.563) -- (53.3695,-11.511) -- (54.9665,-19.629)
%		\center circle [radius=\radius]; 
	% https://tex.stackexchange.com/questions/510281/tikz-fill-only-the-a-b-c
	\path \center circle [radius=\radius];
	\clip[overlay]
		(53.3695,-11.511) -- (59.4605,-11.563) -- (54.9665,-19.629) [rev];
	\clip[overlay, eo] \center circle [radius=5.9475, rev];
	\fill[color=_00ccff] \center circle [radius=\radius];
\end{scope}


		\definecolor{_004455}{HTML}{004455}
\definecolor{_006680}{HTML}{006680}
\definecolor{_0088aa}{HTML}{0088aa}
\definecolor{_00aad4}{HTML}{00aad4}
\fill [color=_00aad4] (49.0,-46.0) rectangle (64.0,-50.5);
\fill [color=_0088aa] (44.5,-35.5) rectangle (49.0,-50.5);
\fill [color=_006680] (44.5,-31.0) rectangle (64.0,-35.5);
\fill [color=_00aad4] (27.5,-46.0) rectangle (42.5,-50.5);
\fill [color=_0088aa] (22.5,-35.5) rectangle (27.5,-50.5);
\fill [color=_006680] (22.5,-31.0) rectangle (42.0,-35.5);
\fill [color=_0088aa] (16.0,-35.5) rectangle (20.5,-51.0);
\fill [color=_006680] (4.5,-31.0) rectangle (20.5,-35.5);
\fill [color=_004455] (0.0,-24.0) rectangle (4.5,-53.5);
% \draw [color=red] (0.0,0.0) rectangle (64,-64);

	\end{tikzpicture}
}

\setbeamercolor{background canvas}{bg=}

\newcommand{\PreFirstFrame}{
	\AddToShipoutPictureFG*{
		\AtPageLowerLeft{
			\put(\LenToUnit{0.05\paperwidth},\LenToUnit{0.1\paperheight}){
				\footnotesize
				这个指引文档在
				\href{https://creativecommons.org/licenses/by-sa/3.0/deed.zh}
				{知识共享 署名-相同方式共享 3.0协议}之条款下提供
			}
			\put(\LenToUnit{0.05\paperwidth},\LenToUnit{0.05\paperheight}){
				\footnotesize
				This guidance is available under the 
				\href{https://creativecommons.org/licenses/by-sa/3.0/}
				{Creative Commons Attribution-ShareAlike License}
			}
			\put(\LenToUnit{0.6\paperwidth},\LenToUnit{0.15\paperheight}){
				\HCCLogoFull
			}
		}
	}
}

\newcommand{\PostFirstFrame}{
	\AddToShipoutPictureBG{
		\AtPageLowerLeft{
			\put(\LenToUnit{0.8\paperwidth},\LenToUnit{0.15\paperheight}){
				\HCCLogoSimp
			}
		}
	}
}

\newcommand{\PreLastFrame}{
	\ClearShipoutPictureBG

	\AddToShipoutPictureFG*{
		\AtPageLowerLeft{
			\put(\LenToUnit{0.6\paperwidth},\LenToUnit{0.15\paperheight}){
				\HCCLogoFull
			}
		}
	}
}

% Note: this C style differs a lot from gedit's
\lstdefinestyle{cstyle}{
	language=c,
	basicstyle=\ttfamily,
	morekeywords={with},
	keywordstyle=\bfseries\color[HTML]{a52a2a},	
	commentstyle=\color[HTML]{0000ff},
	stringstyle=\color[HTML]{ff0bff},
	keywordstyle=[3]\color[HTML]{008a8c},
	alsoletter={0,1,2,3,4,5,6,7,8,9,.},
	morekeywords=[4]{0,1,2,100,999},
	keywordstyle=[4]\color[HTML]{ff0bff},
	upquote=true,
	breaklines=true,
}

\lstdefinestyle{pythonstyle}{
	language=python,
	basicstyle=\ttfamily,
	% frame=single,
	morekeywords={with,yield},
	keywordstyle=\bfseries\color[HTML]{a52a2a},	
	keywordstyle=[2]\color[HTML]{008a8c},
	commentstyle=\color[HTML]{0000ff},
	stringstyle=\color[HTML]{ff0bff},
	keywordstyle=[3]\color[HTML]{008a8c},
	alsoletter={0123456789.},
	morekeywords=[4]{False,True,
		0,1,2,3,4,5,6,7,8,9,10,11,12,13,15,17,19,16,20,24,27,31,32,33,34,35,38,
		45,56,60,64,81,95,97,99,100,123,243,256,400,512,576,729,999,1024,1234,
		1365,1366,2000,2187,2836,2957,3856,3857,5678,6561,9274,100000,1000000,
		0.5,3.14,3.4,
		0x1234,},
	keywordstyle=[4]\color[HTML]{ff0bff},
	upquote=true,
	breaklines=true,
	showstringspaces=false,
}

\lstset{
	tabsize=4,
	columns=fixed,
	extendedchars=false,
}

\newcommand{\inlinePython}{\lstinline[style=pythonstyle]}



\begin{document}
\begin{CJK}{UTF8}{gbsn}

\PreFirstFrame
\begin{frame} [fragile]
	\centerline{\fontsize{42}{42}\selectfont Python Talk 10}
\end{frame}
\PostFirstFrame

\begin{frame}
	\frametitle{多线程编程}
	\begin{itemize}
	\item 多线程编程可以让程序更加灵活
		\begin{itemize}
		\item e.g. 一个网络服务器需要处理很多个客户端的请求,但每个请求之间没有很大的关联
		\end{itemize}
	\item 多线程在某些情况下可以提升程序效率
		\begin{itemize}
		\item e.g. 一个程序需要处理很多互相独立的图像,此时可以利用多个CPU的优势平行处理
		\end{itemize}
	\item 多线程程序相对容易出错、难以调试
	\item 为了防止竞争条件,需要用锁、条件变量等进行控制
	\item Python的\texttt{threading}库支持多线程编程
	\end{itemize}
\end{frame}

\begin{frame} [fragile]
	\frametitle{计算一百万个1相加}
	\begin{itemize}
	\item 单线程做法
	\begin{lstlisting}[style=pythonstyle, gobble=0]
s = 0
for i in range(1000000) :
	s += 1
print(s)
	\end{lstlisting}
	\item 多线程做法
		\begin{itemize}
		\item 开10个线程,共享变量s;每个线程进行100000次加法
		\end{itemize}
	\end{itemize}
\end{frame}

\begin{frame} [fragile]
	\frametitle{多线程程序}
	\begin{columns}[T]
		\begin{column}[T]{.5\textwidth}
			\begin{lstlisting} [style=pythonstyle, gobble=0, tabsize=2, 
								basicstyle=\footnotesize\ttfamily]
import time, threading

s = 0

class MyThread(threading.Thread) :
	def run(self) :
		global s
		for i in range(100000) :
			s += 1

if __name__ == '__main__' :
	for i in range(10) :
		MyThread().start()
	time.sleep(3)
	print(s)
			\end{lstlisting}
		\end{column}
		\begin{column}[T]{.5\textwidth}
			\begin{itemize}
			\item import:引入threading库
			\item threading.Thread:线程的基类
				\begin{itemize}
				\item run()为线程的执行内容
				\item 调用start()以开始线程
				\item 如果调用run()就不是多线程调用了
				\end{itemize}
			\item global表示引用全局变量
			\item 通过time.sleep(3)等待所有线程结束
			\item 问题:得到的结果是1000000吗?
			\end{itemize}
		\end{column}
	\end{columns}
\end{frame}

\begin{frame}[fragile]
	\frametitle{竞争条件}
	\begin{itemize}
	\item 看似简单的 \texttt{s += 1} 其实包含多个指令
	\begin{lstlisting} [gobble=4, basicstyle=\small\ttfamily]
	load  register,       memory[0x1234]
	add   register,       1
	store memory[0x1234], register
	\end{lstlisting}
	\item 翻译成Python就是
	\begin{lstlisting} [style=pythonstyle, gobble=4, basicstyle=\small\ttfamily]
	reg = mem[0x1234]	# load
	reg += 1			# execute
	mem[0x1234] = reg	# store
	\end{lstlisting}
	\item CPU可能在任何两个指令中间暂停执行,并切换到另一个线程
	\end{itemize}
\end{frame}

\begin{frame}[fragile]
	\frametitle{竞争条件}
	\begin{itemize}
	\item 思考如下左右两个线程的执行先后顺序,时间从上到下
	\begin{lstlisting} [style=pythonstyle, gobble=4,
						basicstyle=\footnotesize\ttfamily]
	# Thread 1			# Thread 2
											# mem[] = 0
	reg1 = memory[]							# reg1 = 1
						reg2 = memory[]		# reg2 = 1
	reg1 += 1								# reg1 = 2
						reg2 += 1			# reg2 = 2
	mem[] = reg1							# mem[] = 2
						mem[] = reg2		# mem[] = 2
	\end{lstlisting}
	\item 竞争条件(race
			condition)的定义:两个线程同时访问一个内存地址,其中至少一个访问是写入
	\item 竞争条件有很多解决方法
		\begin{itemize}
		\item 其中最常用的是通过锁控制访问
		\end{itemize}
	\end{itemize}
\end{frame}

\begin{frame}[fragile]
	\frametitle{锁}
	\begin{itemize}
	\item 锁(Lock)是一种特殊变量,每个线程需要使用某些资源时获取锁,在用完后释放锁
	\item 通过各种硬件和软件机制,任何时刻只能有最多一个线程获得锁
		\begin{itemize}
		\item 如果线程尝试获取被占用的锁,一直等待到锁被释放
		\end{itemize}
	\begin{lstlisting} [style=pythonstyle, gobble=4, basicstyle=\ttfamily, texcl]
	lock = threading.Lock()	# 创建锁
	lock.acquire()			# 获取锁
	lock.release()			# 释放锁
	\end{lstlisting}
	\end{itemize}
\end{frame}

\begin{frame}[fragile]
	\frametitle{锁示例}
	\begin{columns}[T]
		\begin{column}[T]{.5\textwidth}
			\begin{lstlisting} [style=pythonstyle, gobble=0, tabsize=2, 
								basicstyle=\footnotesize\ttfamily]
import time, threading

s = 0
s_lock = threading.Lock()

class MyThread(threading.Thread) :
	def run(self) :
		global s, s_lock
		for i in range(100000) :
			s_lock.acquire()
			s += 1
			s_lock.release()

if __name__ == '__main__' :
	for i in range(10) :
		MyThread().start()
	time.sleep(5)
	print(s)
			\end{lstlisting}
		\end{column}
		\begin{column}[T]{.5\textwidth}
			\begin{itemize}
			\item 锁s\_lock用于管理对s的访问
			\item 访问s之前用acquire获取锁
			\item 访问s之后用release获取锁
			\item 程序的运算时间有所增长,因此sleep从3秒变为5秒
				\begin{itemize}
				\item 因为当另一个线程在访问s时,当前线程需要等待
				\item 如何避免使用sleep?
				\item 真实的程序中无法预测线程执行时间
				\end{itemize}
			\end{itemize}
		\end{column}
	\end{columns}
\end{frame}

\begin{frame}
	\frametitle{条件变量}
	\begin{itemize}
	\item 实现线程之间的信号传递
	\item 问题:线程A希望读取变量X,但是需要在线程B写入变量之后进行。
			X的访问通过X\_lock进行限制,通过X\_ready标明是否已经写入。
	\item 实现1
		\begin{itemize}
		\item 线程A通过一个循环,不断获取X\_lock并通过X\_ready判断是否可以读取X
		\item 问题:如果X一直没有被写入,A会进行不停的无用计算
		\end{itemize}
	\item 实现2
		\begin{itemize}
		\item A的每次循环后用\texttt{time.sleep}等待一个很短的时间
		\item 问题:虽然等待很短,但还是有延迟
		\end{itemize}
	\item 实现3
		\begin{itemize}
		\item 线程B在写入后``通知''线程A
		\end{itemize}
	\end{itemize}
\end{frame}

\begin{frame} [fragile]
	\frametitle{条件变量}
	\begin{itemize}
	\item 条件变量允许线程传递``某个事件发生''的信息
	\item 条件变量需要和锁共同使用,当一个线程使用条件变量时必须已获得对应的锁
	\begin{lstlisting} [style=pythonstyle, gobble=4, basicstyle=\ttfamily, texcl]
	cv = threading.Condition(lock)
	lock.acquire()		# 必须先获取lock才能使用cv
	cv.wait()			# 等待一个事件的发生
	cv.notify()			# 通知一个等待的线程
	cv.notify_all()		# 通知所有等待的线程
	lock.release()
	\end{lstlisting}
	\item \texttt{cv.wait()}返回时不保证是被\texttt{notify}的
		\begin{itemize}
		\item 因此应该用一个\texttt{while}循环重新判断是否等待
		\end{itemize}
	\item \texttt{cv.wait()}时会释放\texttt{lock},否则会阻塞其它线程
	\end{itemize}
\end{frame}

\begin{frame} [fragile]
	\frametitle{条件变量示例}
	\begin{itemize}
	\item 线程A读取X,线程B写入X
	\begin{lstlisting} [style=pythonstyle, gobble=4, basicstyle=\small\ttfamily]
	X = 0
	X_lock = threading.Lock()
	X_ready = False
	X_cv = threading.Condition(X_lock)
	class ThreadA(threading.Thread) :
		def run(self) :
			X_lock.acquire()
			while not X_ready :
				X_cv.wait()
			print(X)
			X_lock.release()
	class ThreadB(threading.Thread) :
		def run(self) :
			X_lock.acquire()
			X = 1234
			X_cv.notify()
			X_lock.release()
	\end{lstlisting}
	\end{itemize}
\end{frame}

\begin{frame} [fragile]
	\frametitle{\texttt{with}}
	\begin{itemize}
	\item Python中\texttt{with}有多种用途,在锁中可以用于获取一个锁
	\begin{lstlisting} [style=pythonstyle, gobble=4, texcl]
	lock = threading.Lock()
	with lock :
		# 锁被获取
		# Do something
	# 锁被释放
	\end{lstlisting}
	\item 等价于
	\begin{lstlisting} [style=pythonstyle, gobble=4, escapechar=@]
	lock = threading.Lock()
	lock.acquire()
	# @\color{blue}锁被获取@
	# Do something
	lock.release()
	# @\color{blue}锁被释放@
	\end{lstlisting}
	\end{itemize}
\end{frame}

\begin{frame} [fragile]
	\frametitle{条件变量示例2}
	\begin{itemize}
	\item 让主线程等待所有线程退出
	\begin{lstlisting} [style=pythonstyle, basicstyle=\footnotesize\ttfamily]
r = 0
r_lock = threading.Lock()
r_cv = threading.Condition(r_lock)

class MyThread(threading.Thread) :
	def run(self) :
		global r, r_lock, r_cv
		...
		with r_lock :
			r -= 1
			r_cv.notify()

if __name__ == '__main__' :
	for i in range(10) :
		with r_lock :
			r += 1
		MyThread().start()
	with r_lock :
		while r :
			r_cv.wait()
	\end{lstlisting}
	\end{itemize}
\end{frame}

\begin{comment}
import threading

s = 0
s_lock = threading.Lock()
r = 0
r_lock = threading.Lock()
r_cv = threading.Condition(r_lock)

class MyThread(threading.Thread) :
	def run(self) :
		global s, s_lock, r, r_lock, r_cv
		for i in range(100000) :
			s_lock.acquire()
			s += 1
			s_lock.release()
		with r_lock :
			r -= 1
			r_cv.notify()

if __name__ == '__main__' :
	for i in range(10) :
		with r_lock :
			r += 1
		MyThread().start()
	with r_lock :
		while r :
			r_cv.wait()
	print(s)
\end{comment}

\begin{frame} [fragile]
	\frametitle{练习}
	\begin{enumerate}
	\item 实现一个线程安全的队列
		\begin{itemize}
		\item 队列的长度限制为n
		\item 对于\texttt{enqueue(elem)}操作,如果队列不到n个元素,将\texttt
				{elem}插入队列;如果队列已经有n个元素,阻塞至队列长度变小
		\item 对于\texttt{dequeue()}操作,如果队列中有元素,返回队列中最早被插入的元素;否则阻塞至队列长度变大
		\end{itemize}
	\item 实现一个主要是读的读写锁
		\begin{itemize}
		\item 读写锁允许很多个读锁定线程,但是不可以多个写阻塞线程
		\item 读和写锁定不能同时存在
		\item 读锁定和解锁应该尽可能被优化
		\item 写不能被长时间阻塞(当没有读时写不能被阻塞)
		\item 方法:\texttt{read\_acquire(), read\_release(), write\_acquire(), write\_release()}
		\end{itemize}
\begin{comment}
# from `locate blocked.py`
class ReadWriteLock :
	def __init__(self) :
		self.writing = False	# whether a writer is waiting / writing
		self.reading = 0		# 0: no readers; > 0: number of readers
		self.lock = threading.Lock()
		self.read_cv = threading.Condition(self.lock)
		self.write_cv = threading.Condition(self.lock)
	def read_acquire(self) :
		with self.lock :
			while self.writing :
				self.read_cv.wait()
			self.reading += 1
	def read_release(self) :
		with self.lock :
			self.reading -= 1
			if self.writing and not self.reading :
				self.write_cv.notify()
	def write_acquire(self) :
		with self.lock :
			while self.writing or self.reading :
				self.write_cv.wait()
			self.writing = True
	def write_release(self) :
		with self.lock :
			self.writing = False
			self.read_cv.notify_all()
\end{comment}
	\end{enumerate}
\end{frame}

\begin{frame}
	\frametitle{参考资料}
	\begin{itemize}
	\item Operating Systems - Principles \& Practice, Second Edition,
		by Thomas Anderson and Mike Dahlin
	\item \link{https://docs.python.org/3/library/threading.html}
	\end{itemize}
\end{frame}

\PreLastFrame
\begin{frame}
	\centerline{\fontsize{32}{32}\selectfont 感谢参加此次活动}
\end{frame}

\newpage
\end{CJK}
\end{document}

