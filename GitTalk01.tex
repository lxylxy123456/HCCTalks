\documentclass
% [aspectratio=169]
{beamer}

\usepackage{CJKutf8}
\usepackage{listings}
\usepackage{tikz}
\usepackage{hyperref}
\usepackage{xcolor}
\usepackage{verbatim}
\usepackage{eso-pic}
\usepackage{courier}
\usepackage{textcomp}

\newcommand{\link}[1]{\href{#1}{#1}}

\newcommand{\insertGraph}[3]{

	\centerline{\includegraphics[scale=#1]{#2}} % 0/0

	\centerline{#3}

}

\newcommand{\HCCLogoSimp}{
	\begin{tikzpicture}[scale=0.1]
		\definecolor{_00ccff}{HTML}{00ccff}
\def\center{(54.9655,-19.629)}
\def\radius{7.924}
\fill [color=_00ccff]
	(55.371,-11.075) -- (57.286,-11.417) -- (55.693,-20.323) -- 
	(53.803,-19.976);
\begin{scope}
	% (47.042,-11.705) rectangle (62.889,-27.553)
	% (49.018,-13.682) rectangle (60.913,-25.576)
%	\clip
%		(59.4605,-11.563) -- (54.9665,-19.629) -- (53.3695,-11.511) -- 
%		(47.041,-11.511) -- (47.041,-27.553) -- (62.890,-27.553) -- 
%		(62.890,-11.511);
%	\fill [color=_00ccff, even odd rule]
%		\center circle [radius=5.9475]
%		(59.4605,-11.563) -- (53.3695,-11.511) -- (54.9665,-19.629)
%		\center circle [radius=\radius]; 
	% https://tex.stackexchange.com/questions/510281/tikz-fill-only-the-a-b-c
	\path \center circle [radius=\radius];
	\clip[overlay]
		(53.3695,-11.511) -- (59.4605,-11.563) -- (54.9665,-19.629) [rev];
	\clip[overlay, eo] \center circle [radius=5.9475, rev];
	\fill[color=_00ccff] \center circle [radius=\radius];
\end{scope}


	\end{tikzpicture}
}

\newcommand{\HCCLogoFull}{
	\begin{tikzpicture}[scale=0.05]
		\definecolor{_00ccff}{HTML}{00ccff}
\def\center{(54.9655,-19.629)}
\def\radius{7.924}
\fill [color=_00ccff]
	(55.371,-11.075) -- (57.286,-11.417) -- (55.693,-20.323) -- 
	(53.803,-19.976);
\begin{scope}
	% (47.042,-11.705) rectangle (62.889,-27.553)
	% (49.018,-13.682) rectangle (60.913,-25.576)
%	\clip
%		(59.4605,-11.563) -- (54.9665,-19.629) -- (53.3695,-11.511) -- 
%		(47.041,-11.511) -- (47.041,-27.553) -- (62.890,-27.553) -- 
%		(62.890,-11.511);
%	\fill [color=_00ccff, even odd rule]
%		\center circle [radius=5.9475]
%		(59.4605,-11.563) -- (53.3695,-11.511) -- (54.9665,-19.629)
%		\center circle [radius=\radius]; 
	% https://tex.stackexchange.com/questions/510281/tikz-fill-only-the-a-b-c
	\path \center circle [radius=\radius];
	\clip[overlay]
		(53.3695,-11.511) -- (59.4605,-11.563) -- (54.9665,-19.629) [rev];
	\clip[overlay, eo] \center circle [radius=5.9475, rev];
	\fill[color=_00ccff] \center circle [radius=\radius];
\end{scope}


		\definecolor{_004455}{HTML}{004455}
\definecolor{_006680}{HTML}{006680}
\definecolor{_0088aa}{HTML}{0088aa}
\definecolor{_00aad4}{HTML}{00aad4}
\fill [color=_00aad4] (49.0,-46.0) rectangle (64.0,-50.5);
\fill [color=_0088aa] (44.5,-35.5) rectangle (49.0,-50.5);
\fill [color=_006680] (44.5,-31.0) rectangle (64.0,-35.5);
\fill [color=_00aad4] (27.5,-46.0) rectangle (42.5,-50.5);
\fill [color=_0088aa] (22.5,-35.5) rectangle (27.5,-50.5);
\fill [color=_006680] (22.5,-31.0) rectangle (42.0,-35.5);
\fill [color=_0088aa] (16.0,-35.5) rectangle (20.5,-51.0);
\fill [color=_006680] (4.5,-31.0) rectangle (20.5,-35.5);
\fill [color=_004455] (0.0,-24.0) rectangle (4.5,-53.5);
% \draw [color=red] (0.0,0.0) rectangle (64,-64);

	\end{tikzpicture}
}

\setbeamercolor{background canvas}{bg=}

\newcommand{\PreFirstFrame}{
	\AddToShipoutPictureFG*{
		\AtPageLowerLeft{
			\put(\LenToUnit{0.05\paperwidth},\LenToUnit{0.1\paperheight}){
				\footnotesize
				这个指引文档在
				\href{https://creativecommons.org/licenses/by-sa/3.0/deed.zh}
				{知识共享 署名-相同方式共享 3.0协议}之条款下提供
			}
			\put(\LenToUnit{0.05\paperwidth},\LenToUnit{0.05\paperheight}){
				\footnotesize
				This guidance is available under the 
				\href{https://creativecommons.org/licenses/by-sa/3.0/}
				{Creative Commons Attribution-ShareAlike License}
			}
			\put(\LenToUnit{0.6\paperwidth},\LenToUnit{0.15\paperheight}){
				\HCCLogoFull
			}
		}
	}
}

\newcommand{\PostFirstFrame}{
	\AddToShipoutPictureBG{
		\AtPageLowerLeft{
			\put(\LenToUnit{0.8\paperwidth},\LenToUnit{0.15\paperheight}){
				\HCCLogoSimp
			}
		}
	}
}

\newcommand{\PreLastFrame}{
	\ClearShipoutPictureBG

	\AddToShipoutPictureFG*{
		\AtPageLowerLeft{
			\put(\LenToUnit{0.6\paperwidth},\LenToUnit{0.15\paperheight}){
				\HCCLogoFull
			}
		}
	}
}

% Note: this C style differs a lot from gedit's
\lstdefinestyle{cstyle}{
	language=c,
	basicstyle=\ttfamily,
	morekeywords={with},
	keywordstyle=\bfseries\color[HTML]{a52a2a},	
	commentstyle=\color[HTML]{0000ff},
	stringstyle=\color[HTML]{ff0bff},
	keywordstyle=[3]\color[HTML]{008a8c},
	alsoletter={0,1,2,3,4,5,6,7,8,9,.},
	morekeywords=[4]{0,1,2,100,999},
	keywordstyle=[4]\color[HTML]{ff0bff},
	upquote=true,
	breaklines=true,
}

\lstdefinestyle{pythonstyle}{
	language=python,
	basicstyle=\ttfamily,
	% frame=single,
	morekeywords={with,yield},
	keywordstyle=\bfseries\color[HTML]{a52a2a},	
	keywordstyle=[2]\color[HTML]{008a8c},
	commentstyle=\color[HTML]{0000ff},
	stringstyle=\color[HTML]{ff0bff},
	keywordstyle=[3]\color[HTML]{008a8c},
	alsoletter={0123456789.},
	morekeywords=[4]{False,True,
		0,1,2,3,4,5,6,7,8,9,10,11,12,13,15,17,19,16,20,24,27,31,32,33,34,35,38,
		45,56,60,64,81,95,97,99,100,123,243,256,400,512,576,729,999,1024,1234,
		1365,1366,2000,2187,2836,2957,3856,3857,5678,6561,9274,100000,1000000,
		0.5,3.14,3.4,
		0x1234,},
	keywordstyle=[4]\color[HTML]{ff0bff},
	upquote=true,
	breaklines=true,
	showstringspaces=false,
}

\lstset{
	tabsize=4,
	columns=fixed,
	extendedchars=false,
}

\newcommand{\inlinePython}{\lstinline[style=pythonstyle]}



\usepackage{pmboxdraw}
\usepackage{newunicodechar}
\newunicodechar{└}{\textSFii}
\newunicodechar{├}{\textSFviii}
\newunicodechar{─}{\textSFx}
\newunicodechar{│}{\textSFxi}

\begin{document}

\PreFirstFrame
\begin{frame} [fragile]
	\centerline{\fontsize{42}{42}\selectfont Git Talk 1}
\end{frame}
\PostFirstFrame

\begin{frame} [fragile]
	\frametitle{Git是什么}
	\linespread{1.5}
	\begin{itemize}
	\item Git是目前世界上最先进的分布式版本控制系统
		\begin{itemize}
		\item 来自\href{https://www.liaoxuefeng.com/wiki/896043488029600}
						{廖雪峰的官方网站}教程
		\item HCC目前的所有项目,包括幻灯片都在用Git管理
		\end{itemize}
	\end{itemize}
	
	\
	
	\centerline{\includesvg{asserts/git-logo}}
\end{frame}

\begin{frame} [fragile]
	\frametitle{Git设计理念}
	\linespread{1.5}
	\begin{itemize}
	\item Git的项目管理方式
		\begin{itemize}
		\item 工作区
		\item 暂存区
		\item 版本库
		\end{itemize}
	\item 例
	\linespread{0}
	\begin{lstlisting}[style=bashstyle, gobble=4, texcl, escapechar=@]
	shiyiquan			  # 项目目录
	@├─@.git				 # 隐藏目录
	@│@  @├─暂存区@			    # 临时存储
	@│@  @└─版本库@			    # 永久存储
	@│@    @├─@98d22f3		      # 一个版本
	@│@    @└─@118532c		      # 另一个版本
	@├─@shiyiquan			 # 工作区文件
	@└─@README.md			 # 可随意更改
	\end{lstlisting}
	\linespread{1.5}
	\end{itemize}
\end{frame}

\begin{frame} [fragile]
	\frametitle{基础操作}
	\linespread{1.25}
	\begin{itemize}
	\item 创建版本库
	\begin{lstlisting}[style=bashstyle, gobble=4, texcl]
	mkdir hcc	# 创建项目目录
	cd hcc		# cd 进去
	git init	# 创建git目录结构
	\end{lstlisting}
	\item 开始工作
	\begin{lstlisting}[style=bashstyle, gobble=4, texcl]
	cat > jkl	# 假设jkl是我们正在编写的文件
	\end{lstlisting}
	\item 提交代码
	\begin{lstlisting}[style=bashstyle, gobble=4, texcl]
	git add jkl		# 工作区$\to$暂存区
	git add -A		# 添加全部文件
	git commit		# 暂存区$\to$版本库
					# 第一次提交需设置用户
					# 和vi操作一样
	\end{lstlisting}
	\end{itemize}
\end{frame}

\begin{frame} [fragile]
	\frametitle{状态查看}
	\linespread{1.25}
	\begin{itemize}
	\item 命令
	\begin{lstlisting}[style=bashstyle, gobble=4, texcl, escapechar=@]
	git status		# 检查状态
	git @diff@		# 对比工作区和暂存区
	git log			# 查看版本库日志
	\end{lstlisting}
	\item 尝试
	\begin{lstlisting}[style=bashstyle, gobble=4, texcl, escapechar=@]
	cat >> tmp		# 输入一些内容
	git add -A
	cat >> tmp		# 再输入一些内容
	git status
	git @diff@
	git log			# 按q退出
	\end{lstlisting}
	\end{itemize}
\end{frame}

\begin{frame} [fragile]
	\frametitle{高级选项}
	\linespread{1.5}
	\begin{lstlisting}[style=bashstyle, gobble=4, texcl, escapechar=@]
	git log
		--oneline	# 摘要模式(左侧是版本号的前几位)
		--graph		# 显示版本关系图(在复杂时后很有用)
	git commit
		-m "@\color{Pink}提交说明@"	# 不进入vi
	\end{lstlisting}
\end{frame}

\begin{frame} [fragile]
	\frametitle{回溯}
	\linespread{1.25}
	\begin{itemize}
	\item 命令
	\begin{lstlisting}[style=bashstyle, gobble=4, texcl, escapechar=@]
	git reset --hard @版本号@
	HEAD^		# 当前版本的前一个版本
	HEAD~10		# 前10个版本
	\end{lstlisting}
	\item 回溯后将很难看到更靠后的版本
	\item 查看操作记录
		\begin{itemize}
		\item \inlineBash{git reflog}
		\item 可以看到以后的版本并跳转回去
		\end{itemize}
	\end{itemize}
\end{frame}

\begin{frame} [fragile]
	\frametitle{连接远程版本库}
	\linespread{1.5}
	\begin{itemize}
	\item 通过连接提供git服务的服务器和开发团队同步代码
		\begin{itemize}
		\item \inlineBash{git remote add origin 链接}
		\end{itemize}
	\item 如何获得链接
		\begin{itemize}
		\item 服务商一般会提供链接,复制粘贴即可
		\item 一般链接是有规律的,例如
			\begin{itemize}
			\item \texttt{https://github.com/用户/项目.git}
			\item \texttt{git@github.com:用户/项目.git}
			\end{itemize}
		\item 链接甚至可以是一个本地的git目录
		\item ``\texttt{origin}''是链接的名称,用于快速访问链接
		\end{itemize}
	\end{itemize}
\end{frame}

\begin{frame} [fragile]
	\frametitle{推送和接收}
	\linespread{1.5}
	\begin{itemize}
	\item 推送
		\begin{itemize}
		\item \inlineBash{git push origin master}
		\item 将当前所在分支推送到origin的master分支
		\end{itemize}
	\item 接收
		\begin{itemize}
		\item \inlineBash{git pull origin master}
		\item 将origin的master分支拉到当前分支
		\end{itemize}
	\item ``master''是默认分支名称
	\end{itemize}
\end{frame}

\begin{frame} [fragile]
	\frametitle{分支}
	\linespread{1.25}
	\begin{itemize}
	\item 分支可以代表不同阶段的代码,可以互相合并
	\item 创建
		\begin{itemize}
		\item \inlineBash{git branch dev}
		\item 从当前分支创建dev分支
		\end{itemize}
	\item 跳转
		\begin{itemize}
		\item \inlineBash{git checkout dev}
		\item 切换到dev分支
		\end{itemize}
	\item 合并
		\begin{itemize}
		\item \inlineBash{git merge master}
		\item 将master分支合并到当前分支
		\item 从网络拉取(\inlineBash{git pull})也可能会合并分支
		\end{itemize}
	\end{itemize}
\end{frame}

\begin{frame} [fragile]
	\frametitle{分支练习}
	\linespread{1.5}
	\begin{itemize}
	\item 从master开始执行
	\begin{lstlisting}[style=bashstyle, gobble=4, texcl, escapechar=@]
	git branch dev
	echo "master ver" >> jkl
	git commit -am "Master"	# -am等于add后提交
	git checkout dev		# jkl变了
	echo "dev version" >> jkl
	git commit -am "Dev"
	git merge master		# 出现冲突
	\end{lstlisting}
	\end{itemize}
\end{frame}

\begin{frame} [fragile]
	\frametitle{冲突}
	\linespread{1.5}
	\begin{itemize}
	\item 冲突后git会提示冲突的文件,只要按照标记手动合并然后重新提交即可
	\begin{lstlisting}[basicstyle=\ttfamily, gobble=4, texcl, escapechar=@]
	@\color{Pink}<<<<<<< HEAD@			# @当前分支@
	dev version
	@\color{Pink}=======@				 # @分割线@
	master ver
	@\color{Pink}>>>>>>> master@		  # @合并分支@
	\end{lstlisting}
	\end{itemize}
\end{frame}

\begin{frame} [fragile]
	\frametitle{其它功能}
	\linespread{1.5}
	\begin{lstlisting}[style=bashstyle, gobble=4, texcl, escapechar=@]
	git tag					# \href
	{https://www.liaoxuefeng.com/wiki/896043488029600/900788941487552}{标记代码版本}
	git stash				# \href
	{https://www.liaoxuefeng.com/wiki/896043488029600/900785521032192}{隐藏版本}
	git config				# 自定义git
	.gitignore				# \href
	{https://www.liaoxuefeng.com/wiki/896043488029600/900004590234208}{忽略文件}
	git grep				# 全文搜索
	\end{lstlisting}
\end{frame}

\PreLastFrame
\begin{frame}
	\centerline{\fontsize{32}{32}\selectfont 感谢参加此次活动}
\end{frame}

\newpage
\end{document}

