\documentclass
% [aspectratio=169]
{beamer}

\usepackage{CJKutf8}
\usepackage{listings}
\usepackage{tikz}
\usepackage{hyperref}
\usepackage{xcolor}
\usepackage{verbatim}
\usepackage{eso-pic}
\usepackage{courier}
\usepackage{textcomp}
\usepackage[inkscapepath=/tmp/TalkSvg/]{svg}

\newcommand{\link}[1]{\href{#1}{#1}}

\newcommand{\insertGraph}[3]{

	\centerline{\includegraphics[scale=#1]{#2}} % 0/0

	\centerline{#3}

}

\newcommand{\HCCLogoSimp}{
	\begin{tikzpicture}[scale=0.1]
		\definecolor{_00ccff}{HTML}{00ccff}
\def\center{(54.9655,-19.629)}
\def\radius{7.924}
\fill [color=_00ccff]
	(55.371,-11.075) -- (57.286,-11.417) -- (55.693,-20.323) -- 
	(53.803,-19.976);
\begin{scope}
	% (47.042,-11.705) rectangle (62.889,-27.553)
	% (49.018,-13.682) rectangle (60.913,-25.576)
%	\clip
%		(59.4605,-11.563) -- (54.9665,-19.629) -- (53.3695,-11.511) -- 
%		(47.041,-11.511) -- (47.041,-27.553) -- (62.890,-27.553) -- 
%		(62.890,-11.511);
%	\fill [color=_00ccff, even odd rule]
%		\center circle [radius=5.9475]
%		(59.4605,-11.563) -- (53.3695,-11.511) -- (54.9665,-19.629)
%		\center circle [radius=\radius]; 
	% https://tex.stackexchange.com/questions/510281/tikz-fill-only-the-a-b-c
	\path \center circle [radius=\radius];
	\clip[overlay]
		(53.3695,-11.511) -- (59.4605,-11.563) -- (54.9665,-19.629) [rev];
	\clip[overlay, eo] \center circle [radius=5.9475, rev];
	\fill[color=_00ccff] \center circle [radius=\radius];
\end{scope}


	\end{tikzpicture}
}

\newcommand{\HCCLogoFull}{
	\begin{tikzpicture}[scale=0.05]
		\definecolor{_00ccff}{HTML}{00ccff}
\def\center{(54.9655,-19.629)}
\def\radius{7.924}
\fill [color=_00ccff]
	(55.371,-11.075) -- (57.286,-11.417) -- (55.693,-20.323) -- 
	(53.803,-19.976);
\begin{scope}
	% (47.042,-11.705) rectangle (62.889,-27.553)
	% (49.018,-13.682) rectangle (60.913,-25.576)
%	\clip
%		(59.4605,-11.563) -- (54.9665,-19.629) -- (53.3695,-11.511) -- 
%		(47.041,-11.511) -- (47.041,-27.553) -- (62.890,-27.553) -- 
%		(62.890,-11.511);
%	\fill [color=_00ccff, even odd rule]
%		\center circle [radius=5.9475]
%		(59.4605,-11.563) -- (53.3695,-11.511) -- (54.9665,-19.629)
%		\center circle [radius=\radius]; 
	% https://tex.stackexchange.com/questions/510281/tikz-fill-only-the-a-b-c
	\path \center circle [radius=\radius];
	\clip[overlay]
		(53.3695,-11.511) -- (59.4605,-11.563) -- (54.9665,-19.629) [rev];
	\clip[overlay, eo] \center circle [radius=5.9475, rev];
	\fill[color=_00ccff] \center circle [radius=\radius];
\end{scope}


		\definecolor{_004455}{HTML}{004455}
\definecolor{_006680}{HTML}{006680}
\definecolor{_0088aa}{HTML}{0088aa}
\definecolor{_00aad4}{HTML}{00aad4}
\fill [color=_00aad4] (49.0,-46.0) rectangle (64.0,-50.5);
\fill [color=_0088aa] (44.5,-35.5) rectangle (49.0,-50.5);
\fill [color=_006680] (44.5,-31.0) rectangle (64.0,-35.5);
\fill [color=_00aad4] (27.5,-46.0) rectangle (42.5,-50.5);
\fill [color=_0088aa] (22.5,-35.5) rectangle (27.5,-50.5);
\fill [color=_006680] (22.5,-31.0) rectangle (42.0,-35.5);
\fill [color=_0088aa] (16.0,-35.5) rectangle (20.5,-51.0);
\fill [color=_006680] (4.5,-31.0) rectangle (20.5,-35.5);
\fill [color=_004455] (0.0,-24.0) rectangle (4.5,-53.5);
% \draw [color=red] (0.0,0.0) rectangle (64,-64);

	\end{tikzpicture}
}

\setbeamercolor{background canvas}{bg=}

\newcommand{\PreFirstFrame}{
	\AddToShipoutPictureFG*{
		\AtPageLowerLeft{
			\put(\LenToUnit{0.05\paperwidth},\LenToUnit{0.1\paperheight}){
				\footnotesize
				这个指引文档在
				\href{https://creativecommons.org/licenses/by-sa/3.0/deed.zh}
				{知识共享 署名-相同方式共享 3.0协议}之条款下提供
			}
			\put(\LenToUnit{0.05\paperwidth},\LenToUnit{0.05\paperheight}){
				\footnotesize
				This guidance is available under the
				\href{https://creativecommons.org/licenses/by-sa/3.0/}
				{Creative Commons Attribution-ShareAlike License}
			}
			\put(\LenToUnit{0.6\paperwidth},\LenToUnit{0.15\paperheight}){
				\HCCLogoFull
			}
		}
	}
}

\newcommand{\PostFirstFrame}{
	\AddToShipoutPictureBG{
		\AtPageLowerLeft{
			\put(\LenToUnit{0.8\paperwidth},\LenToUnit{0.15\paperheight}){
				\HCCLogoSimp
			}
		}
	}
}

\newcommand{\PreLastFrame}{
	\ClearShipoutPictureBG

	\AddToShipoutPictureFG*{
		\AtPageLowerLeft{
			\put(\LenToUnit{0.6\paperwidth},\LenToUnit{0.15\paperheight}){
				\HCCLogoFull
			}
		}
	}
}

\definecolor{Pink}{HTML}{ff0bff}	% Use: \color{Pink} OR \color[HTML]{ff0bff}
\definecolor{Blue}{HTML}{0000ff}
\definecolor{Green}{HTML}{008a8c}
\definecolor{Brown}{HTML}{a52a2a}

% Note: this C style differs a lot from gedit's
\lstdefinestyle{cstyle}{
	language=c,
	basicstyle=\ttfamily,
	morekeywords={with},
	keywordstyle=\bfseries\color{Brown},
	commentstyle=\color{Blue},
	stringstyle=\color{Pink},
	keywordstyle=[3]\color{Green},
	alsoletter={0123456789.},
	morekeywords=[4]{0,1,2,100,999},
	keywordstyle=[4]\color{Pink},
	upquote=true,
	breaklines=true,
}

\lstdefinestyle{pythonstyle}{
	language=python,
	basicstyle=\ttfamily,
	% frame=single,
	morekeywords={with,yield},
	keywordstyle=\bfseries\color{Brown},
	keywordstyle=[2]\color{Green},
	commentstyle=\color{Blue},
	stringstyle=\color{Pink},
	keywordstyle=[3]\color{Green},
	alsoletter={0123456789.},
	morekeywords=[4]{False,True,
		0,1,2,3,4,5,6,7,8,9,10,11,12,13,15,17,19,16,20,24,27,31,32,33,34,35,38,
		45,56,60,64,81,95,97,99,100,123,243,256,400,512,576,729,999,1024,1234,
		1365,1366,2000,2187,2836,2957,3856,3857,5678,6561,9274,100000,1000000,
		0.5,3.14,3.4,
		0x1234,},
	keywordstyle=[4]\color{Pink},
	upquote=true,
	breaklines=true,
	showstringspaces=false,
}

\lstdefinestyle{bashstyle}{
	language=bash,
	basicstyle=\ttfamily,
	% frame=single,
	morekeywords=[2]{
		cp,df,du,find,ll,ls,mkdir,mv,rm,tree,
		cut,diff,grep,head,hexdump,less,man,md5sum,more,nano,sha1sum,tail,vi,wc,
		date,free,ifconfig,lspci,sleep,sync,time,top,uname,
		7za,dnf,history,rpm,su,sudo,wget,yum,
		git,
	},
	keywordstyle=\bfseries\color{Brown},
	keywordstyle=[2]\color{Green},
	commentstyle=\color{Blue},
	stringstyle=\color{Pink},
	keywordstyle=[3]\color{Green},
	alsoletter={-=0123456789.|><&~*?;},
	morekeywords={&,&&,|,||,>,<,>>,<<,2>,&>,;},
	morekeywords=[4]{
		0,1,2,3,4,5,6,7,8,9,10,
		a,b,c,d,e,f,g,h,i,j,k,l,m,n,o,p,q,r,s,t,u,v,w,x,y,z,
		bin,dev,etc,hcc,home,jkl,mnt,opt,root,srv,tmp,usr,
		*,?,.,..,~,-,stdin,stdout,stderr,
		null,zero,random,urandom,sda,sda1,sda2,sda3,sda4,sdb1,sr0,USB,tmpfs,
		passwd,shadow,hosts,fstab,sudoers,locale.conf,resolv.conf,
		.bash_history,mydoc.7z,mydoc,master,
	},
	keywordstyle=[4]\color{Pink},
	upquote=true,
	breaklines=true,
	showstringspaces=false,
}

\lstset{
	tabsize=4,
	columns=fixed,
	extendedchars=false,
}

\newcommand{\inlinePython}{\lstinline[style=pythonstyle]}

\newcommand{\inlineBash}{\lstinline[style=bashstyle]}

\newcommand{\inlineListing}{\lstinline[basicstyle=\ttfamily]}

