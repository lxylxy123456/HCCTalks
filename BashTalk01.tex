\documentclass
% [aspectratio=169]
{beamer}

\usepackage{CJKutf8}
\usepackage{listings}
\usepackage{tikz}
\usepackage{hyperref}
\usepackage{xcolor}
\usepackage{verbatim}
\usepackage{eso-pic}
\usepackage{courier}
\usepackage{textcomp}

\newcommand{\link}[1]{\href{#1}{#1}}

\newcommand{\insertGraph}[3]{

	\centerline{\includegraphics[scale=#1]{#2}} % 0/0

	\centerline{#3}

}

\newcommand{\HCCLogoSimp}{
	\begin{tikzpicture}[scale=0.1]
		\definecolor{_00ccff}{HTML}{00ccff}
\def\center{(54.9655,-19.629)}
\def\radius{7.924}
\fill [color=_00ccff]
	(55.371,-11.075) -- (57.286,-11.417) -- (55.693,-20.323) -- 
	(53.803,-19.976);
\begin{scope}
	% (47.042,-11.705) rectangle (62.889,-27.553)
	% (49.018,-13.682) rectangle (60.913,-25.576)
%	\clip
%		(59.4605,-11.563) -- (54.9665,-19.629) -- (53.3695,-11.511) -- 
%		(47.041,-11.511) -- (47.041,-27.553) -- (62.890,-27.553) -- 
%		(62.890,-11.511);
%	\fill [color=_00ccff, even odd rule]
%		\center circle [radius=5.9475]
%		(59.4605,-11.563) -- (53.3695,-11.511) -- (54.9665,-19.629)
%		\center circle [radius=\radius]; 
	% https://tex.stackexchange.com/questions/510281/tikz-fill-only-the-a-b-c
	\path \center circle [radius=\radius];
	\clip[overlay]
		(53.3695,-11.511) -- (59.4605,-11.563) -- (54.9665,-19.629) [rev];
	\clip[overlay, eo] \center circle [radius=5.9475, rev];
	\fill[color=_00ccff] \center circle [radius=\radius];
\end{scope}


	\end{tikzpicture}
}

\newcommand{\HCCLogoFull}{
	\begin{tikzpicture}[scale=0.05]
		\definecolor{_00ccff}{HTML}{00ccff}
\def\center{(54.9655,-19.629)}
\def\radius{7.924}
\fill [color=_00ccff]
	(55.371,-11.075) -- (57.286,-11.417) -- (55.693,-20.323) -- 
	(53.803,-19.976);
\begin{scope}
	% (47.042,-11.705) rectangle (62.889,-27.553)
	% (49.018,-13.682) rectangle (60.913,-25.576)
%	\clip
%		(59.4605,-11.563) -- (54.9665,-19.629) -- (53.3695,-11.511) -- 
%		(47.041,-11.511) -- (47.041,-27.553) -- (62.890,-27.553) -- 
%		(62.890,-11.511);
%	\fill [color=_00ccff, even odd rule]
%		\center circle [radius=5.9475]
%		(59.4605,-11.563) -- (53.3695,-11.511) -- (54.9665,-19.629)
%		\center circle [radius=\radius]; 
	% https://tex.stackexchange.com/questions/510281/tikz-fill-only-the-a-b-c
	\path \center circle [radius=\radius];
	\clip[overlay]
		(53.3695,-11.511) -- (59.4605,-11.563) -- (54.9665,-19.629) [rev];
	\clip[overlay, eo] \center circle [radius=5.9475, rev];
	\fill[color=_00ccff] \center circle [radius=\radius];
\end{scope}


		\definecolor{_004455}{HTML}{004455}
\definecolor{_006680}{HTML}{006680}
\definecolor{_0088aa}{HTML}{0088aa}
\definecolor{_00aad4}{HTML}{00aad4}
\fill [color=_00aad4] (49.0,-46.0) rectangle (64.0,-50.5);
\fill [color=_0088aa] (44.5,-35.5) rectangle (49.0,-50.5);
\fill [color=_006680] (44.5,-31.0) rectangle (64.0,-35.5);
\fill [color=_00aad4] (27.5,-46.0) rectangle (42.5,-50.5);
\fill [color=_0088aa] (22.5,-35.5) rectangle (27.5,-50.5);
\fill [color=_006680] (22.5,-31.0) rectangle (42.0,-35.5);
\fill [color=_0088aa] (16.0,-35.5) rectangle (20.5,-51.0);
\fill [color=_006680] (4.5,-31.0) rectangle (20.5,-35.5);
\fill [color=_004455] (0.0,-24.0) rectangle (4.5,-53.5);
% \draw [color=red] (0.0,0.0) rectangle (64,-64);

	\end{tikzpicture}
}

\setbeamercolor{background canvas}{bg=}

\newcommand{\PreFirstFrame}{
	\AddToShipoutPictureFG*{
		\AtPageLowerLeft{
			\put(\LenToUnit{0.05\paperwidth},\LenToUnit{0.1\paperheight}){
				\footnotesize
				这个指引文档在
				\href{https://creativecommons.org/licenses/by-sa/3.0/deed.zh}
				{知识共享 署名-相同方式共享 3.0协议}之条款下提供
			}
			\put(\LenToUnit{0.05\paperwidth},\LenToUnit{0.05\paperheight}){
				\footnotesize
				This guidance is available under the 
				\href{https://creativecommons.org/licenses/by-sa/3.0/}
				{Creative Commons Attribution-ShareAlike License}
			}
			\put(\LenToUnit{0.6\paperwidth},\LenToUnit{0.15\paperheight}){
				\HCCLogoFull
			}
		}
	}
}

\newcommand{\PostFirstFrame}{
	\AddToShipoutPictureBG{
		\AtPageLowerLeft{
			\put(\LenToUnit{0.8\paperwidth},\LenToUnit{0.15\paperheight}){
				\HCCLogoSimp
			}
		}
	}
}

\newcommand{\PreLastFrame}{
	\ClearShipoutPictureBG

	\AddToShipoutPictureFG*{
		\AtPageLowerLeft{
			\put(\LenToUnit{0.6\paperwidth},\LenToUnit{0.15\paperheight}){
				\HCCLogoFull
			}
		}
	}
}

% Note: this C style differs a lot from gedit's
\lstdefinestyle{cstyle}{
	language=c,
	basicstyle=\ttfamily,
	morekeywords={with},
	keywordstyle=\bfseries\color[HTML]{a52a2a},	
	commentstyle=\color[HTML]{0000ff},
	stringstyle=\color[HTML]{ff0bff},
	keywordstyle=[3]\color[HTML]{008a8c},
	alsoletter={0,1,2,3,4,5,6,7,8,9,.},
	morekeywords=[4]{0,1,2,100,999},
	keywordstyle=[4]\color[HTML]{ff0bff},
	upquote=true,
	breaklines=true,
}

\lstdefinestyle{pythonstyle}{
	language=python,
	basicstyle=\ttfamily,
	% frame=single,
	morekeywords={with,yield},
	keywordstyle=\bfseries\color[HTML]{a52a2a},	
	keywordstyle=[2]\color[HTML]{008a8c},
	commentstyle=\color[HTML]{0000ff},
	stringstyle=\color[HTML]{ff0bff},
	keywordstyle=[3]\color[HTML]{008a8c},
	alsoletter={0123456789.},
	morekeywords=[4]{False,True,
		0,1,2,3,4,5,6,7,8,9,10,11,12,13,15,17,19,16,20,24,27,31,32,33,34,35,38,
		45,56,60,64,81,95,97,99,100,123,243,256,400,512,576,729,999,1024,1234,
		1365,1366,2000,2187,2836,2957,3856,3857,5678,6561,9274,100000,1000000,
		0.5,3.14,3.4,
		0x1234,},
	keywordstyle=[4]\color[HTML]{ff0bff},
	upquote=true,
	breaklines=true,
	showstringspaces=false,
}

\lstset{
	tabsize=4,
	columns=fixed,
	extendedchars=false,
}

\newcommand{\inlinePython}{\lstinline[style=pythonstyle]}



\begin{document}
\begin{CJK}{UTF8}{gbsn}

\PreFirstFrame
\begin{frame} [fragile]
	\centerline{\fontsize{42}{42}\selectfont Bash Talk 1}
\end{frame}
\PostFirstFrame

\begin{frame}
	\frametitle{Bash 是什么?}
	\linespread{2}
	\begin{itemize}
	\item Bash是广泛用于Linux和OS X的命令行软件
	\item Bash是开源软件,可以自由复制和更改
	\item Bash可以轻松处理各种系统任务
		\begin{itemize}
		\item 例如文件更名、网页下载、远程控制
		\end{itemize}
	\end{itemize}
\end{frame}

\begin{frame} [fragile]
	\frametitle{一切皆为文件}
	\linespread{2}
	\begin{itemize}
	\item \href{https://en.wikipedia.org/wiki/Everything\_is\_a\_file}
				{Everything is a file}
	\item Unix的设计观念是一切皆为文件
		\begin{itemize}
		\item 甚至是你的U盘
		\end{itemize}
	\item 因此,bash的设计主要基于文件
	\end{itemize}
\end{frame}

\begin{frame} [fragile]
	\frametitle{Linux \& Fedora}
	\linespread{2}
	\begin{itemize}
	\item 我们推荐使用主流开源操作系统Linux
	\item Fedora是我们常用的Linux发行版本
	\item Bash是为Linux设计的,也可用于Windows
	\item Windows使用Bash需要额外软件
		\begin{itemize}
		\item 我们会在Git经验交流会介绍安装方法
		\end{itemize}
	\end{itemize}
\end{frame}

\begin{frame} [fragile]
	\frametitle{目录访问}
	\linespread{1.25}
	\begin{itemize}
	\item 命令
		\begin{lstlisting}[style=bashstyle, gobble=8, texcl]
		cd		# 跳转到目录
		pwd		# 查看当前目录
		ls		# 查看目录的文件
		ll		# 文件详细信息
		\end{lstlisting}
	\item 这些命令不会改磁盘
	\item 示例
		\begin{lstlisting}[style=bashstyle, gobble=8, texcl]
		cd /dev
		pwd
		cd		# 到主目录
		ls /etc
		ls
		ll /etc
	\end{lstlisting}
	\end{itemize}
\end{frame}

\begin{frame} [fragile]
	\frametitle{这些命令如何工作}
	\linespread{1.25}
	\begin{itemize}
	\item 语法
	\begin{lstlisting}[style=bashstyle, gobble=4, escapechar=@]
	ls [@选项@] [@文件名@...]
	\end{lstlisting}
	\item 举例
	\begin{lstlisting}[style=bashstyle, gobble=4, texcl]
	ls -l /				# 相当于ll
	ls /home/hcc -a		# 前后都可以
	\end{lstlisting}
	\end{itemize}
\end{frame}

\begin{frame} [fragile]
	\frametitle{关于\inlineBash{ll}}
	\linespread{1.25}
	\begin{itemize}
	\item 第一行:占用大小
	\item 第1列:文件属性和权限
		\begin{itemize}
		\item 见英文维基百科:\href{https://en.wikipedia.org/wiki/Chmod}{chmod}
		\end{itemize}
	\item 第2列:硬链接数量
		\begin{itemize}
		\item 内部文件(夹)数量+1
		\end{itemize}
	\item 第3列:所属用户组
	\item 第4列:所属用户
	\item 第5列:占用空间(B)
	\item 第6至8列:修改时间
	\item 第9列:名称
	\end{itemize}
\end{frame}

\begin{frame} [fragile]
	\frametitle{更加高级的命令}
	\begin{itemize}
	\item \inlineBash{cd /etc}
	\item 命令
	\begin{lstlisting}[style=bashstyle, gobble=4, texcl]
	tree			# 获取目录树
	find			# 递归查找文件
	du				# 查看文件大小
	df				# 磁盘使用情况
	\end{lstlisting}
	\item 这些命令不会改磁盘
	\item 选项
	\begin{lstlisting}[style=bashstyle, gobble=4, texcl]
	tree -d
	du --max-depth=1
	df -h			# 适用面很广
	\end{lstlisting}
	\item 思考:Windows中有吗?
	\end{itemize}
\end{frame}

\begin{frame} [fragile]
	\frametitle{创建文件和目录}
	\begin{itemize}
	\item \inlineBash{cd /tmp}
	\item 命令
	\begin{lstlisting}[style=bashstyle, gobble=4, texcl, escapechar=@]
	cat > @\color[HTML]{ff0bff}文件名@		# 创建文件
					 # 输入文本,然后按Ctrl + D
	cat @\color[HTML]{ff0bff}文件名@		# 打印文件内容
	mkdir @\color[HTML]{ff0bff}目录名@		# 创建目录
	\end{lstlisting}
	\item 尝试(请严格按照下面的代码操作)
	\begin{lstlisting}[style=bashstyle, gobble=4, texcl]
	cd /tmp
	cat > a
	mkdir b
	mkdir b/c
	cat > b/d
	tree
	\end{lstlisting}
	\end{itemize}
\end{frame}

\begin{frame} [fragile]
	\frametitle{管理文件和目录}
	\linespread{1.25}
	\begin{itemize}
	\item 命令(执行每一行后可以用\inlineBash{tree}查看效果)
	\begin{lstlisting}[style=bashstyle, gobble=4, texcl]
	mv a b				# 移动文件a到目录b
	mv b/a b/e			# 将b中的a更名为e
	cp b/e f			# 将b中的a复制到f
	cp b g -r			# 复制目录b到g
	rm f				# 删除文件f
	rm g -r				# 删除目录g,可尝试-f
	\end{lstlisting}
	\end{itemize}
\end{frame}

\begin{frame} [fragile]
	\frametitle{这些命令可以干什么}
	\linespread{1.5}
	\begin{columns}[T]
		\begin{column}[T]{.5\textwidth}
			\begin{lstlisting}[style=bashstyle, gobble=12, texcl]
			ll		ls		pwd
			du		df		tree
			cat		rm		mkdir
			mv		cp		find
			cd
			\end{lstlisting}
		\end{column}
		\begin{column}[T]{.5\textwidth}
			\begin{itemize}
			\item 查看目录树和其他信息
			\item 查看磁盘空间
			\item 复制、移动和删除文件
			\item 创建目录
			\item 切换工作目录
			\end{itemize}
		\end{column}
	\end{columns}
\end{frame}

\begin{frame} [fragile]
	\frametitle{这些命令可以干什么}
	\linespread{1.5}
	\begin{columns}[T]
		\begin{column}[T]{.5\textwidth}
			\begin{lstlisting}[style=bashstyle, gobble=12, texcl]
			ll		ls		pwd
			du		df		tree
			cat		rm		mkdir
			mv		cp		find
			cd
			\end{lstlisting}
		\end{column}
		\begin{column}[T]{.5\textwidth}
			\begin{itemize}
			\item 完成Windows可以做到的任何事情
			\item 并超越Windows
				\begin{itemize}
				\item 分析空间
				\item 批量操纵
				\end{itemize}
			\end{itemize}
		\end{column}
	\end{columns}
\end{frame}

\PreLastFrame
\begin{frame}
	\centerline{\fontsize{32}{32}\selectfont 感谢参加此次活动}
\end{frame}

\newpage
\end{CJK}
\end{document}

