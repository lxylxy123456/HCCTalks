\documentclass
% [aspectratio=169]
{beamer}

\usepackage{CJKutf8}
\usepackage{listings}
\usepackage{tikz}
\usepackage{hyperref}
\usepackage{xcolor}
\usepackage{verbatim}
\usepackage{eso-pic}
\usepackage{courier}
\usepackage{textcomp}

\newcommand{\link}[1]{\href{#1}{#1}}

\newcommand{\insertGraph}[3]{

	\centerline{\includegraphics[scale=#1]{#2}} % 0/0

	\centerline{#3}

}

\newcommand{\HCCLogoSimp}{
	\begin{tikzpicture}[scale=0.1]
		\definecolor{_00ccff}{HTML}{00ccff}
\def\center{(54.9655,-19.629)}
\def\radius{7.924}
\fill [color=_00ccff]
	(55.371,-11.075) -- (57.286,-11.417) -- (55.693,-20.323) -- 
	(53.803,-19.976);
\begin{scope}
	% (47.042,-11.705) rectangle (62.889,-27.553)
	% (49.018,-13.682) rectangle (60.913,-25.576)
%	\clip
%		(59.4605,-11.563) -- (54.9665,-19.629) -- (53.3695,-11.511) -- 
%		(47.041,-11.511) -- (47.041,-27.553) -- (62.890,-27.553) -- 
%		(62.890,-11.511);
%	\fill [color=_00ccff, even odd rule]
%		\center circle [radius=5.9475]
%		(59.4605,-11.563) -- (53.3695,-11.511) -- (54.9665,-19.629)
%		\center circle [radius=\radius]; 
	% https://tex.stackexchange.com/questions/510281/tikz-fill-only-the-a-b-c
	\path \center circle [radius=\radius];
	\clip[overlay]
		(53.3695,-11.511) -- (59.4605,-11.563) -- (54.9665,-19.629) [rev];
	\clip[overlay, eo] \center circle [radius=5.9475, rev];
	\fill[color=_00ccff] \center circle [radius=\radius];
\end{scope}


	\end{tikzpicture}
}

\newcommand{\HCCLogoFull}{
	\begin{tikzpicture}[scale=0.05]
		\definecolor{_00ccff}{HTML}{00ccff}
\def\center{(54.9655,-19.629)}
\def\radius{7.924}
\fill [color=_00ccff]
	(55.371,-11.075) -- (57.286,-11.417) -- (55.693,-20.323) -- 
	(53.803,-19.976);
\begin{scope}
	% (47.042,-11.705) rectangle (62.889,-27.553)
	% (49.018,-13.682) rectangle (60.913,-25.576)
%	\clip
%		(59.4605,-11.563) -- (54.9665,-19.629) -- (53.3695,-11.511) -- 
%		(47.041,-11.511) -- (47.041,-27.553) -- (62.890,-27.553) -- 
%		(62.890,-11.511);
%	\fill [color=_00ccff, even odd rule]
%		\center circle [radius=5.9475]
%		(59.4605,-11.563) -- (53.3695,-11.511) -- (54.9665,-19.629)
%		\center circle [radius=\radius]; 
	% https://tex.stackexchange.com/questions/510281/tikz-fill-only-the-a-b-c
	\path \center circle [radius=\radius];
	\clip[overlay]
		(53.3695,-11.511) -- (59.4605,-11.563) -- (54.9665,-19.629) [rev];
	\clip[overlay, eo] \center circle [radius=5.9475, rev];
	\fill[color=_00ccff] \center circle [radius=\radius];
\end{scope}


		\definecolor{_004455}{HTML}{004455}
\definecolor{_006680}{HTML}{006680}
\definecolor{_0088aa}{HTML}{0088aa}
\definecolor{_00aad4}{HTML}{00aad4}
\fill [color=_00aad4] (49.0,-46.0) rectangle (64.0,-50.5);
\fill [color=_0088aa] (44.5,-35.5) rectangle (49.0,-50.5);
\fill [color=_006680] (44.5,-31.0) rectangle (64.0,-35.5);
\fill [color=_00aad4] (27.5,-46.0) rectangle (42.5,-50.5);
\fill [color=_0088aa] (22.5,-35.5) rectangle (27.5,-50.5);
\fill [color=_006680] (22.5,-31.0) rectangle (42.0,-35.5);
\fill [color=_0088aa] (16.0,-35.5) rectangle (20.5,-51.0);
\fill [color=_006680] (4.5,-31.0) rectangle (20.5,-35.5);
\fill [color=_004455] (0.0,-24.0) rectangle (4.5,-53.5);
% \draw [color=red] (0.0,0.0) rectangle (64,-64);

	\end{tikzpicture}
}

\setbeamercolor{background canvas}{bg=}

\newcommand{\PreFirstFrame}{
	\AddToShipoutPictureFG*{
		\AtPageLowerLeft{
			\put(\LenToUnit{0.05\paperwidth},\LenToUnit{0.1\paperheight}){
				\footnotesize
				这个指引文档在
				\href{https://creativecommons.org/licenses/by-sa/3.0/deed.zh}
				{知识共享 署名-相同方式共享 3.0协议}之条款下提供
			}
			\put(\LenToUnit{0.05\paperwidth},\LenToUnit{0.05\paperheight}){
				\footnotesize
				This guidance is available under the 
				\href{https://creativecommons.org/licenses/by-sa/3.0/}
				{Creative Commons Attribution-ShareAlike License}
			}
			\put(\LenToUnit{0.6\paperwidth},\LenToUnit{0.15\paperheight}){
				\HCCLogoFull
			}
		}
	}
}

\newcommand{\PostFirstFrame}{
	\AddToShipoutPictureBG{
		\AtPageLowerLeft{
			\put(\LenToUnit{0.8\paperwidth},\LenToUnit{0.15\paperheight}){
				\HCCLogoSimp
			}
		}
	}
}

\newcommand{\PreLastFrame}{
	\ClearShipoutPictureBG

	\AddToShipoutPictureFG*{
		\AtPageLowerLeft{
			\put(\LenToUnit{0.6\paperwidth},\LenToUnit{0.15\paperheight}){
				\HCCLogoFull
			}
		}
	}
}

% Note: this C style differs a lot from gedit's
\lstdefinestyle{cstyle}{
	language=c,
	basicstyle=\ttfamily,
	morekeywords={with},
	keywordstyle=\bfseries\color[HTML]{a52a2a},	
	commentstyle=\color[HTML]{0000ff},
	stringstyle=\color[HTML]{ff0bff},
	keywordstyle=[3]\color[HTML]{008a8c},
	alsoletter={0,1,2,3,4,5,6,7,8,9,.},
	morekeywords=[4]{0,1,2,100,999},
	keywordstyle=[4]\color[HTML]{ff0bff},
	upquote=true,
	breaklines=true,
}

\lstdefinestyle{pythonstyle}{
	language=python,
	basicstyle=\ttfamily,
	% frame=single,
	morekeywords={with,yield},
	keywordstyle=\bfseries\color[HTML]{a52a2a},	
	keywordstyle=[2]\color[HTML]{008a8c},
	commentstyle=\color[HTML]{0000ff},
	stringstyle=\color[HTML]{ff0bff},
	keywordstyle=[3]\color[HTML]{008a8c},
	alsoletter={0123456789.},
	morekeywords=[4]{False,True,
		0,1,2,3,4,5,6,7,8,9,10,11,12,13,15,17,19,16,20,24,27,31,32,33,34,35,38,
		45,56,60,64,81,95,97,99,100,123,243,256,400,512,576,729,999,1024,1234,
		1365,1366,2000,2187,2836,2957,3856,3857,5678,6561,9274,100000,1000000,
		0.5,3.14,3.4,
		0x1234,},
	keywordstyle=[4]\color[HTML]{ff0bff},
	upquote=true,
	breaklines=true,
	showstringspaces=false,
}

\lstset{
	tabsize=4,
	columns=fixed,
	extendedchars=false,
}

\newcommand{\inlinePython}{\lstinline[style=pythonstyle]}



\begin{document}
\begin{CJK}{UTF8}{gbsn}

\PreFirstFrame
\begin{frame} [fragile]
	\centerline{\fontsize{42}{42}\selectfont Python Talk 7}
\end{frame}
\PostFirstFrame

\begin{frame} [fragile]
	\frametitle{复习}
	\linespread{1.25}
	\begin{enumerate}
	\item 按照最后一个元素的大小排序
		\begin{itemize}
		\item \inlinePython{a = [(1, 2), (4, ), (6, 5, 3), [1]]}
		\end{itemize}
	\item 将3进制转为8进制
		\begin{itemize}
		\item \inlinePython{'20122011201020'}
		\end{itemize}
	\item 将字符串变为字节
		\begin{itemize}
		\item \begin{lstlisting}[style=pythonstyle, gobble=8, escapechar=@]
		'@\color{Pink}十一圈@'
		\end{lstlisting}
		\end{itemize}
	\item 用以下格式输出当前时间
		\begin{itemize}
		\item 小时.分钟
		\end{itemize}
	\end{enumerate}
\end{frame}

\begin{frame} [fragile]
	\frametitle{随机}
	\linespread{1.25}
	\begin{itemize}
	\item 函数
		\begin{lstlisting}[style=pythonstyle, gobble=8, texcl]
		import random
		random.randint
		random.sample
		# 使用 help 探索
		\end{lstlisting}
	\item 提示
		\begin{itemize}
		\item \inlinePython{randint} 会取到左右短点
		\item \inlinePython{randrange} 不会取右短点
		\item \inlinePython{sample} 可以取数组中的多个项目
		\end{itemize}
	\end{itemize}
\end{frame}

\begin{frame} [fragile]
	\frametitle{\inlinePython{sys}}
	\begin{itemize}
	\item 常用用法
		\begin{lstlisting}[style=pythonstyle, gobble=8, texcl]
		import sys
		sys.stdin		# 标准输入
		sys.stdout		# 标准输出
		sys.argv		# 命令参数
		\end{lstlisting}
	\item \inlinePython{argv}示例
		\begin{lstlisting}[style=pythonstyle, gobble=8, texcl, escapechar=@]
		$ cat > a.py	# 编辑 a.py
		import sys
		print(sys.argv)
		$ python3 a.py hello HCC    Im
		['a.py', 'hello', 'HCC', 'Im']
		\end{lstlisting}
	\end{itemize}
\end{frame}

\begin{frame} [fragile]
	\frametitle{\inlinePython{uuid}}
	\linespread{1.25}
	\begin{itemize}
	\item \inlinePython{uuid}用来生成唯一标识符
		\begin{lstlisting}[style=pythonstyle, gobble=8, texcl, escapechar=@]
		import uuid
		uuid.uuid1()		# str() 可将其转换为文本
		help(uuid)			# 探索uuid2等函数
		\end{lstlisting}
	\item 应用:十一圈的分享链接
		\begin{lstlisting}[style=pythonstyle, gobble=8, texcl, escapechar=@]
		def uuid_create():
			while True:
				u = uuid.uuid1()
				s = str(u)[0: 8]
				if ... :		# 确认没有重复
					return s
		\end{lstlisting}
	\end{itemize}
\end{frame}

\begin{frame} [fragile]
	\frametitle{\inlinePython{getch}}
	\linespread{1.25}
	\begin{itemize}
	\item \inlinePython{getch}可以从键盘读入一个字符,不需按回车
		\begin{lstlisting}[style=pythonstyle, gobble=8, texcl, escapechar=@]
		import getch
		getch.getch()		# 不会输出到屏幕
		getch.getche()		# 获取字符并输出到屏幕
		\end{lstlisting}
	\item 需要额外安装软件包:\texttt{pip3 install getch}
	\item 任务
		\begin{itemize}
		\item 制作不回显的密码输入器
		\item 用户按下回车时退出,并返回用户已输入的密码
		\end{itemize}
	\item 提高
		\begin{itemize}
		\item 支持退格
		\item 支持左右键移动光标
		\end{itemize}
	\end{itemize}
\begin{comment}
import getch
def getpass() :
	l, r = '', ''
	while True :
		c = getch.getch()
		if c == '\n' :
			return l + r
		elif c == '\x7f' :
			l = l[:-1]
		elif c == '\x1b' :	# Backspace
			assert getch.getch() == '['
			c = getch.getch()
			if c == 'C' :	# Right
				if r :
					l += r[0]
					r = r[1:]
			elif c == 'D' :	# Left
				if l :
					r = l[-1] + r
					l = l[:-1]
			elif c == '3' :	# Delete
				assert getch.getch() == '~'
				r = r[1:]
		else :
			l += c
\end{comment}
\end{frame}

\begin{frame} [fragile]
	\frametitle{\inlinePython{qrcode}}
	\linespread{1.5}
	\begin{itemize}
	\item \inlinePython{qrcode}用来生成二维码
		\begin{lstlisting}[style=pythonstyle, gobble=8, texcl, escapechar=@]
		import qrcode
		qrcode.run_example('@\color{Pink}内容字符串@')
		\end{lstlisting}
	\item 需要额外安装软件包:\texttt{pip3 install qrcode}
	\item 十一圈的二维码都是通过这个工具生成的
		\begin{lstlisting}[style=pythonstyle, gobble=8, texcl]
		qrcode.make		# 生成可保存的图像
		\end{lstlisting}
	\end{itemize}
\end{frame}

\begin{frame} [fragile]
	\frametitle{提示}
	\begin{itemize}
	\item 尝试
		\begin{lstlisting}[style=pythonstyle, gobble=8, texcl, escapechar=@]
		1 + 1
		3 * _		# 下划线表示上次运算结果
		_ = 'HCC'
		1 + 1
		1 + _		# 被污染了
		\end{lstlisting}
	\item 解决
		\begin{lstlisting}[style=pythonstyle, gobble=8, texcl, escapechar=@]
		del(_)
		1 + 1
		1 + _		# 恢复了
		\end{lstlisting}
	\item 除了 \inlinePython{_} ,还适用于
		\inlinePython{int list max os ...}
	\end{itemize}
\end{frame}

\begin{frame} [fragile]
	\frametitle{连线:这些功能可以做什么}
	\linespread{1.5}
	\begin{columns}[T]
		\begin{column}[T]{.5\textwidth}
			\begin{enumerate}
			\item 产生随机数
			\item 制造唯一链接
			\item 安全输入密码
			\item 生成二维码
			\item 架设一个网站
			\end{enumerate}
		\end{column}
		\begin{column}[T]{.5\textwidth}
			\begin{itemize}
			\item \inlinePython{uuid}
			\item \inlinePython{qrcode}
			\item \inlinePython{python3}
			\item \inlinePython{getch}
			\item \inlinePython{random}
			\end{itemize}
		\end{column}
	\end{columns}
\end{frame}

\PreLastFrame
\begin{frame}
	\centerline{\fontsize{32}{32}\selectfont 感谢参加此次活动}
\end{frame}

\newpage
\end{CJK}
\end{document}

