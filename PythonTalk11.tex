\documentclass
% [aspectratio=169]
{beamer}

\usepackage{CJKutf8}
\usepackage{listings}
\usepackage{tikz}
\usepackage{hyperref}
\usepackage{xcolor}
\usepackage{verbatim}
\usepackage{eso-pic}
\usepackage{courier}
\usepackage{textcomp}

\newcommand{\link}[1]{\href{#1}{#1}}

\newcommand{\insertGraph}[3]{

	\centerline{\includegraphics[scale=#1]{#2}} % 0/0

	\centerline{#3}

}

\newcommand{\HCCLogoSimp}{
	\begin{tikzpicture}[scale=0.1]
		\definecolor{_00ccff}{HTML}{00ccff}
\def\center{(54.9655,-19.629)}
\def\radius{7.924}
\fill [color=_00ccff]
	(55.371,-11.075) -- (57.286,-11.417) -- (55.693,-20.323) -- 
	(53.803,-19.976);
\begin{scope}
	% (47.042,-11.705) rectangle (62.889,-27.553)
	% (49.018,-13.682) rectangle (60.913,-25.576)
%	\clip
%		(59.4605,-11.563) -- (54.9665,-19.629) -- (53.3695,-11.511) -- 
%		(47.041,-11.511) -- (47.041,-27.553) -- (62.890,-27.553) -- 
%		(62.890,-11.511);
%	\fill [color=_00ccff, even odd rule]
%		\center circle [radius=5.9475]
%		(59.4605,-11.563) -- (53.3695,-11.511) -- (54.9665,-19.629)
%		\center circle [radius=\radius]; 
	% https://tex.stackexchange.com/questions/510281/tikz-fill-only-the-a-b-c
	\path \center circle [radius=\radius];
	\clip[overlay]
		(53.3695,-11.511) -- (59.4605,-11.563) -- (54.9665,-19.629) [rev];
	\clip[overlay, eo] \center circle [radius=5.9475, rev];
	\fill[color=_00ccff] \center circle [radius=\radius];
\end{scope}


	\end{tikzpicture}
}

\newcommand{\HCCLogoFull}{
	\begin{tikzpicture}[scale=0.05]
		\definecolor{_00ccff}{HTML}{00ccff}
\def\center{(54.9655,-19.629)}
\def\radius{7.924}
\fill [color=_00ccff]
	(55.371,-11.075) -- (57.286,-11.417) -- (55.693,-20.323) -- 
	(53.803,-19.976);
\begin{scope}
	% (47.042,-11.705) rectangle (62.889,-27.553)
	% (49.018,-13.682) rectangle (60.913,-25.576)
%	\clip
%		(59.4605,-11.563) -- (54.9665,-19.629) -- (53.3695,-11.511) -- 
%		(47.041,-11.511) -- (47.041,-27.553) -- (62.890,-27.553) -- 
%		(62.890,-11.511);
%	\fill [color=_00ccff, even odd rule]
%		\center circle [radius=5.9475]
%		(59.4605,-11.563) -- (53.3695,-11.511) -- (54.9665,-19.629)
%		\center circle [radius=\radius]; 
	% https://tex.stackexchange.com/questions/510281/tikz-fill-only-the-a-b-c
	\path \center circle [radius=\radius];
	\clip[overlay]
		(53.3695,-11.511) -- (59.4605,-11.563) -- (54.9665,-19.629) [rev];
	\clip[overlay, eo] \center circle [radius=5.9475, rev];
	\fill[color=_00ccff] \center circle [radius=\radius];
\end{scope}


		\definecolor{_004455}{HTML}{004455}
\definecolor{_006680}{HTML}{006680}
\definecolor{_0088aa}{HTML}{0088aa}
\definecolor{_00aad4}{HTML}{00aad4}
\fill [color=_00aad4] (49.0,-46.0) rectangle (64.0,-50.5);
\fill [color=_0088aa] (44.5,-35.5) rectangle (49.0,-50.5);
\fill [color=_006680] (44.5,-31.0) rectangle (64.0,-35.5);
\fill [color=_00aad4] (27.5,-46.0) rectangle (42.5,-50.5);
\fill [color=_0088aa] (22.5,-35.5) rectangle (27.5,-50.5);
\fill [color=_006680] (22.5,-31.0) rectangle (42.0,-35.5);
\fill [color=_0088aa] (16.0,-35.5) rectangle (20.5,-51.0);
\fill [color=_006680] (4.5,-31.0) rectangle (20.5,-35.5);
\fill [color=_004455] (0.0,-24.0) rectangle (4.5,-53.5);
% \draw [color=red] (0.0,0.0) rectangle (64,-64);

	\end{tikzpicture}
}

\setbeamercolor{background canvas}{bg=}

\newcommand{\PreFirstFrame}{
	\AddToShipoutPictureFG*{
		\AtPageLowerLeft{
			\put(\LenToUnit{0.05\paperwidth},\LenToUnit{0.1\paperheight}){
				\footnotesize
				这个指引文档在
				\href{https://creativecommons.org/licenses/by-sa/3.0/deed.zh}
				{知识共享 署名-相同方式共享 3.0协议}之条款下提供
			}
			\put(\LenToUnit{0.05\paperwidth},\LenToUnit{0.05\paperheight}){
				\footnotesize
				This guidance is available under the 
				\href{https://creativecommons.org/licenses/by-sa/3.0/}
				{Creative Commons Attribution-ShareAlike License}
			}
			\put(\LenToUnit{0.6\paperwidth},\LenToUnit{0.15\paperheight}){
				\HCCLogoFull
			}
		}
	}
}

\newcommand{\PostFirstFrame}{
	\AddToShipoutPictureBG{
		\AtPageLowerLeft{
			\put(\LenToUnit{0.8\paperwidth},\LenToUnit{0.15\paperheight}){
				\HCCLogoSimp
			}
		}
	}
}

\newcommand{\PreLastFrame}{
	\ClearShipoutPictureBG

	\AddToShipoutPictureFG*{
		\AtPageLowerLeft{
			\put(\LenToUnit{0.6\paperwidth},\LenToUnit{0.15\paperheight}){
				\HCCLogoFull
			}
		}
	}
}

% Note: this C style differs a lot from gedit's
\lstdefinestyle{cstyle}{
	language=c,
	basicstyle=\ttfamily,
	morekeywords={with},
	keywordstyle=\bfseries\color[HTML]{a52a2a},	
	commentstyle=\color[HTML]{0000ff},
	stringstyle=\color[HTML]{ff0bff},
	keywordstyle=[3]\color[HTML]{008a8c},
	alsoletter={0,1,2,3,4,5,6,7,8,9,.},
	morekeywords=[4]{0,1,2,100,999},
	keywordstyle=[4]\color[HTML]{ff0bff},
	upquote=true,
	breaklines=true,
}

\lstdefinestyle{pythonstyle}{
	language=python,
	basicstyle=\ttfamily,
	% frame=single,
	morekeywords={with,yield},
	keywordstyle=\bfseries\color[HTML]{a52a2a},	
	keywordstyle=[2]\color[HTML]{008a8c},
	commentstyle=\color[HTML]{0000ff},
	stringstyle=\color[HTML]{ff0bff},
	keywordstyle=[3]\color[HTML]{008a8c},
	alsoletter={0123456789.},
	morekeywords=[4]{False,True,
		0,1,2,3,4,5,6,7,8,9,10,11,12,13,15,17,19,16,20,24,27,31,32,33,34,35,38,
		45,56,60,64,81,95,97,99,100,123,243,256,400,512,576,729,999,1024,1234,
		1365,1366,2000,2187,2836,2957,3856,3857,5678,6561,9274,100000,1000000,
		0.5,3.14,3.4,
		0x1234,},
	keywordstyle=[4]\color[HTML]{ff0bff},
	upquote=true,
	breaklines=true,
	showstringspaces=false,
}

\lstset{
	tabsize=4,
	columns=fixed,
	extendedchars=false,
}

\newcommand{\inlinePython}{\lstinline[style=pythonstyle]}



\begin{document}

\PreFirstFrame
\begin{frame} [fragile]
	\centerline{\fontsize{42}{42}\selectfont Python Talk 11}
\end{frame}
\PostFirstFrame

\begin{frame}
	\frametitle{调试程序}
	\linespread{1.5}
	\begin{itemize}
	\item 调试程序可以更清楚地看到程序的运算过程,远比单纯看代码方便理解
	\item 常见的调试程序的方法有两种
		\begin{itemize}
		\item 在代码中加入 \inlinePython{print} 等语句来观察各个变量的值
		\item (推荐)通过专业的调试器调试,例如
				\inlinePython{gdb}、\inlinePython{pdb}
		\end{itemize}
	\item 使用情景
		\begin{itemize}
		\item 程序运算结果出错,但是不知道是中间哪一步导致的
		\item 拿到别人的复杂程序需要理解,但是不知从何下手
		\item 想更改程序运行时的一个中间变量,但不改代码
		\end{itemize}
	\end{itemize}
\end{frame}

\begin{frame} [fragile]
	\frametitle{\inlinePython{pdb}}
	\linespread{1.5}
	\begin{itemize}
	\item \inlinePython{pdb} 是 Python 的默认调试器,已经在 Python 中预装
	\item \href{https://docs.python.org/3/library/pdb.html}{官方文档}有最详细的教程
	\item 从程序内部调用
		\begin{lstlisting} [style=pythonstyle, gobble=8]
		>>> import pdb; pdb.set_trace()
		\end{lstlisting}
	\item 从命令行调用
		\begin{lstlisting} [style=bashstyle, gobble=8]
		$ python3 -m pdb 程序名.py
		\end{lstlisting}
	\end{itemize}
\end{frame}

\begin{frame} [fragile]
	\frametitle{使用示例}
	\begin{itemize}
	\item 回想 Python Talk 4 中的一段错误程序
		{\small
		\begin{lstlisting}[style=pythonstyle, gobble=8, frame=single, 
							numbers=left, numberstyle=\ttfamily]
		a = 10
		if a == 2 :
			print(True)
		elif a % 2 == 0 :
			print(False)
		else :
			for i in range(3, a, 2) :
				if a % i == 0 :
					print(False)
					break
				print(True)
		\end{lstlisting}
		}
	\item 此程序会判断 \inlinePython{a} 是否是质数,但是输出结果有问题
	\item 将其保存为 \texttt{a.py} ,然后用命令行执行
		\begin{lstlisting} [style=bashstyle, gobble=8]
		$ python3 -m pdb a.py
		\end{lstlisting}
	\end{itemize}
\end{frame}

\begin{frame} [fragile]
	\frametitle{代码步进}
	\begin{columns}[T]
		\begin{column}[T]{.5\textwidth}
			{\small
			\linespread{0.9}
			\begin{lstlisting} [style=bashstyle, gobble=12]
			$ python3 -m pdb a.py
			> a.py(1)<module>()
			-> a = 10
			(Pdb) n
			> a.py(2)<module>()
			-> if a == 2 :
			(Pdb) n
			> a.py(4)<module>()
			-> elif a % 2 == 0 :
			(Pdb) n
			> a.py(5)<module>()
			-> print(False)
			(Pdb) n
			False
			--Return--
			> a.py(5)<module>()->None
			-> print(False)
			(Pdb) quit
			$ 
			\end{lstlisting}
			}
		\end{column}
		\begin{column}[T]{.5\textwidth}
			\linespread{1.5}
			\begin{itemize}
			\item 进入 \inlinePython{pdb} 后输入 \inlinePython{r}
					可以直接执行程序(相当于没有使用调试器)
			\item 输入 \inlinePython{n} 可以一行一行执行代码
			\item 从左侧可以看出程序分别执行了第 1, 2, 4, 5 行
			\item 通过这种方法我们可以看出程序走入了哪个分支
			\end{itemize}
		\end{column}
	\end{columns}
\end{frame}

\begin{frame} [fragile]
	\frametitle{断点}
	\begin{columns}[T]
		\begin{column}[T]{.5\textwidth}
			{\small
			\begin{lstlisting} [style=bashstyle, gobble=12]
			$ python3 -m pdb a.py
			> a.py(1)<module>()
			-> a = 7
			(Pdb) b 7
			Breakpoint 1 at a.py:7
			(Pdb) c
			> a.py(7)<module>()
			-> for i in range(3,a,2):
			(Pdb) n
			> a.py(8)<module>()
			-> if a % i == 0 :
			(Pdb) n
			> a.py(11)<module>()
			-> print(True)
			(Pdb) ...
			\end{lstlisting}
			}
		\end{column}
		\begin{column}[T]{.5\textwidth}
			\linespread{1.5}
			\begin{itemize}
			\item 先将第一行改成 \inlinePython{a = 7}
			\item 通过``\inlinePython{b 行号}''设置断点。程序在执行到断点时会暂停执行
			\item 输入 \inlinePython{c} 从暂停执行恢复
			\item 尝试在第7行设置断点,然后观察程序之后的执行过程
			\item 你能找到改正程序的方法吗?
			\end{itemize}
		\end{column}
	\end{columns}
\end{frame}

\begin{frame} [fragile]
	\frametitle{函数调用}
	\linespread{1.25}
	\begin{itemize}
	\item 以下递归程序会打印一些中括号
		{\small
		\begin{lstlisting}[style=pythonstyle, gobble=8, frame=single, 
							numbers=left, numberstyle=\ttfamily, texcl]
		def f(x) :
			if x == 0 :
				return ''
			s = f(x - 2) * 2
			return '[' + s + ']'
		print(f(4))		# 正确,结果是 [[][]]
		print(f(3))		# 错误,应该是 [[][]]
		\end{lstlisting}
		}
	\item 我们希望每次递归时 \inlinePython{x} 减少 \inlinePython{2} ,然后在
		\inlinePython{0} 时停止
	\item 但是在执行 \inlinePython{f(3)} 时会出现无限递归的情况
	\item 尝试通过 \inlinePython{pdb} 找出原因所在,通过下一页的命令
	\end{itemize}
\end{frame}

\begin{frame} [fragile]
	\frametitle{更多 \inlinePython{pdb} 命令}
	\linespread{1.25}
	\begin{columns}[T]
		\begin{column}[T]{.5\textwidth}
			\begin{itemize}
			\item \inlinePython{n}:执行一行,不跳入函数
			\item \inlinePython{s}:执行一行,可跳入函数
			\item \inlinePython{r}:执行到当前函数退出
			\item \inlinePython{bt}:显示当前调用堆栈
				\begin{itemize}
				\item 越上面是越旧的调用
				\item 一般来说导致错误的代码在最下面
				\end{itemize}
			\item \inlinePython{up}:调用堆栈向上
			\item \inlinePython{down}:调用堆栈向下
			\item \inlinePython{l}:当前执行的代码片段
			\item \inlinePython{p 变量名}:打印变量的值
			\end{itemize}
		\end{column}
		\begin{column}[T]{.5\textwidth}
			\begin{itemize}
			\item 尝试:在第6行设置断点,通过\inlinePython{s}观察程序每一步执行的行号
			\item 每次递归调用时(第4行)用\inlinePython{p}打印\inlinePython{x}的值
			\item 调用到 \inlinePython{x = 0} 时用 \inlinePython{bt} 和
					\inlinePython{up} 查看当前的调用堆栈,并打印上面的调用的
					\inlinePython{x} 的值
				\begin{itemize}
				\item 在迷路时用 \inlinePython{l} 看周围的代码
				\end{itemize}
			\end{itemize}
		\end{column}
	\end{columns}
	\begin{comment}
	b 6
	r	# line 6
	s	# line 1
	s	# line 2
	p x	# 			4
	s	# line 4
	s	# line 1
	s	# line 2
	p x	# 			2
	s	# line 4
	s	# line 1
	s	# line 2
	p x	# 			0
	bt
	up	# line 4, stack -2
	p x	# 			2
	up	# line 4, stack -3
	p x	# 			6
	up	# line 6, stack -4
	l	# line 6
	\end{comment}
\end{frame}

\begin{frame} [fragile]
	\frametitle{更改变量值}
	\linespread{1.25}
	\begin{itemize}
	\item \texttt{!代码}:执行一行代码
		\begin{itemize}
		\item \inlinePython{!x = 4}:将 \inlinePython{x} 的值更改为 \inlinePython{4}
		\item 提示: `\inlinePython{!}' 后面不可有空格
		\end{itemize}
	\item \texttt{b 行号, 条件}:设置有条件的断点
		\begin{itemize}
		\item \inlinePython{b 2, x == -1}:程序执行到第2行且 \inlinePython{x = -1} 时暂停执行
		\end{itemize}
	\item 尝试:运行 \texttt{b.py} 直到第2行且 \inlinePython{x = -3} ,此时将
				\inlinePython{x} 改为 \inlinePython{0} ,然后继续运行(用 
				\inlinePython{c} )
		\begin{itemize}
		\item 结果应该是 \texttt{[[[][]][[][]]]}
		\end{itemize}
	\item 思考:如何更改 \texttt{b.py} 以使其正常运行?
	\end{itemize}
	\begin{comment}
	b 2, x == -3
	r
	p x			# -3
	!x = 0
	c
	\end{comment}
\end{frame}

\begin{frame} [fragile]
	\frametitle{其他类似调试器的方法}
	\linespread{1.5}
	\begin{itemize}
	\item 之前提到的 \inlinePython{set_trace} 可以在代码中进入调试模式
		\begin{itemize}
		\item \inlinePython{import pdb; pdb.set_trace()}
		\item Python 调试模式中运算速度会变慢,因此这样可以提高效率
		\end{itemize}
	\item \inlinePython{code.interact} 可以从代码进入交互模式
		\begin{itemize}
		\item \inlinePython{import code; code.interact()}
		\item \inlinePython{code.interact(local=locals())} 保留所有局部变量
		\item \inlinePython|local={**locals(), **globals()}| 也保留全局变量
		\end{itemize}
	\end{itemize}
\end{frame}

\PreLastFrame
\begin{frame}
	\centerline{\fontsize{32}{32}\selectfont 感谢参加此次活动}
\end{frame}

\newpage
\end{document}

