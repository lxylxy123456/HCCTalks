\documentclass
% [aspectratio=169]
{beamer}

\usepackage{CJKutf8}
\usepackage{listings}
\usepackage{tikz}
\usepackage{hyperref}
\usepackage{xcolor}
\usepackage{verbatim}
\usepackage{eso-pic}
\usepackage{courier}
\usepackage{textcomp}

\newcommand{\link}[1]{\href{#1}{#1}}

\newcommand{\insertGraph}[3]{

	\centerline{\includegraphics[scale=#1]{#2}} % 0/0

	\centerline{#3}

}

\newcommand{\HCCLogoSimp}{
	\begin{tikzpicture}[scale=0.1]
		\definecolor{_00ccff}{HTML}{00ccff}
\def\center{(54.9655,-19.629)}
\def\radius{7.924}
\fill [color=_00ccff]
	(55.371,-11.075) -- (57.286,-11.417) -- (55.693,-20.323) -- 
	(53.803,-19.976);
\begin{scope}
	% (47.042,-11.705) rectangle (62.889,-27.553)
	% (49.018,-13.682) rectangle (60.913,-25.576)
%	\clip
%		(59.4605,-11.563) -- (54.9665,-19.629) -- (53.3695,-11.511) -- 
%		(47.041,-11.511) -- (47.041,-27.553) -- (62.890,-27.553) -- 
%		(62.890,-11.511);
%	\fill [color=_00ccff, even odd rule]
%		\center circle [radius=5.9475]
%		(59.4605,-11.563) -- (53.3695,-11.511) -- (54.9665,-19.629)
%		\center circle [radius=\radius]; 
	% https://tex.stackexchange.com/questions/510281/tikz-fill-only-the-a-b-c
	\path \center circle [radius=\radius];
	\clip[overlay]
		(53.3695,-11.511) -- (59.4605,-11.563) -- (54.9665,-19.629) [rev];
	\clip[overlay, eo] \center circle [radius=5.9475, rev];
	\fill[color=_00ccff] \center circle [radius=\radius];
\end{scope}


	\end{tikzpicture}
}

\newcommand{\HCCLogoFull}{
	\begin{tikzpicture}[scale=0.05]
		\definecolor{_00ccff}{HTML}{00ccff}
\def\center{(54.9655,-19.629)}
\def\radius{7.924}
\fill [color=_00ccff]
	(55.371,-11.075) -- (57.286,-11.417) -- (55.693,-20.323) -- 
	(53.803,-19.976);
\begin{scope}
	% (47.042,-11.705) rectangle (62.889,-27.553)
	% (49.018,-13.682) rectangle (60.913,-25.576)
%	\clip
%		(59.4605,-11.563) -- (54.9665,-19.629) -- (53.3695,-11.511) -- 
%		(47.041,-11.511) -- (47.041,-27.553) -- (62.890,-27.553) -- 
%		(62.890,-11.511);
%	\fill [color=_00ccff, even odd rule]
%		\center circle [radius=5.9475]
%		(59.4605,-11.563) -- (53.3695,-11.511) -- (54.9665,-19.629)
%		\center circle [radius=\radius]; 
	% https://tex.stackexchange.com/questions/510281/tikz-fill-only-the-a-b-c
	\path \center circle [radius=\radius];
	\clip[overlay]
		(53.3695,-11.511) -- (59.4605,-11.563) -- (54.9665,-19.629) [rev];
	\clip[overlay, eo] \center circle [radius=5.9475, rev];
	\fill[color=_00ccff] \center circle [radius=\radius];
\end{scope}


		\definecolor{_004455}{HTML}{004455}
\definecolor{_006680}{HTML}{006680}
\definecolor{_0088aa}{HTML}{0088aa}
\definecolor{_00aad4}{HTML}{00aad4}
\fill [color=_00aad4] (49.0,-46.0) rectangle (64.0,-50.5);
\fill [color=_0088aa] (44.5,-35.5) rectangle (49.0,-50.5);
\fill [color=_006680] (44.5,-31.0) rectangle (64.0,-35.5);
\fill [color=_00aad4] (27.5,-46.0) rectangle (42.5,-50.5);
\fill [color=_0088aa] (22.5,-35.5) rectangle (27.5,-50.5);
\fill [color=_006680] (22.5,-31.0) rectangle (42.0,-35.5);
\fill [color=_0088aa] (16.0,-35.5) rectangle (20.5,-51.0);
\fill [color=_006680] (4.5,-31.0) rectangle (20.5,-35.5);
\fill [color=_004455] (0.0,-24.0) rectangle (4.5,-53.5);
% \draw [color=red] (0.0,0.0) rectangle (64,-64);

	\end{tikzpicture}
}

\setbeamercolor{background canvas}{bg=}

\newcommand{\PreFirstFrame}{
	\AddToShipoutPictureFG*{
		\AtPageLowerLeft{
			\put(\LenToUnit{0.05\paperwidth},\LenToUnit{0.1\paperheight}){
				\footnotesize
				这个指引文档在
				\href{https://creativecommons.org/licenses/by-sa/3.0/deed.zh}
				{知识共享 署名-相同方式共享 3.0协议}之条款下提供
			}
			\put(\LenToUnit{0.05\paperwidth},\LenToUnit{0.05\paperheight}){
				\footnotesize
				This guidance is available under the 
				\href{https://creativecommons.org/licenses/by-sa/3.0/}
				{Creative Commons Attribution-ShareAlike License}
			}
			\put(\LenToUnit{0.6\paperwidth},\LenToUnit{0.15\paperheight}){
				\HCCLogoFull
			}
		}
	}
}

\newcommand{\PostFirstFrame}{
	\AddToShipoutPictureBG{
		\AtPageLowerLeft{
			\put(\LenToUnit{0.8\paperwidth},\LenToUnit{0.15\paperheight}){
				\HCCLogoSimp
			}
		}
	}
}

\newcommand{\PreLastFrame}{
	\ClearShipoutPictureBG

	\AddToShipoutPictureFG*{
		\AtPageLowerLeft{
			\put(\LenToUnit{0.6\paperwidth},\LenToUnit{0.15\paperheight}){
				\HCCLogoFull
			}
		}
	}
}

% Note: this C style differs a lot from gedit's
\lstdefinestyle{cstyle}{
	language=c,
	basicstyle=\ttfamily,
	morekeywords={with},
	keywordstyle=\bfseries\color[HTML]{a52a2a},	
	commentstyle=\color[HTML]{0000ff},
	stringstyle=\color[HTML]{ff0bff},
	keywordstyle=[3]\color[HTML]{008a8c},
	alsoletter={0,1,2,3,4,5,6,7,8,9,.},
	morekeywords=[4]{0,1,2,100,999},
	keywordstyle=[4]\color[HTML]{ff0bff},
	upquote=true,
	breaklines=true,
}

\lstdefinestyle{pythonstyle}{
	language=python,
	basicstyle=\ttfamily,
	% frame=single,
	morekeywords={with,yield},
	keywordstyle=\bfseries\color[HTML]{a52a2a},	
	keywordstyle=[2]\color[HTML]{008a8c},
	commentstyle=\color[HTML]{0000ff},
	stringstyle=\color[HTML]{ff0bff},
	keywordstyle=[3]\color[HTML]{008a8c},
	alsoletter={0123456789.},
	morekeywords=[4]{False,True,
		0,1,2,3,4,5,6,7,8,9,10,11,12,13,15,17,19,16,20,24,27,31,32,33,34,35,38,
		45,56,60,64,81,95,97,99,100,123,243,256,400,512,576,729,999,1024,1234,
		1365,1366,2000,2187,2836,2957,3856,3857,5678,6561,9274,100000,1000000,
		0.5,3.14,3.4,
		0x1234,},
	keywordstyle=[4]\color[HTML]{ff0bff},
	upquote=true,
	breaklines=true,
	showstringspaces=false,
}

\lstset{
	tabsize=4,
	columns=fixed,
	extendedchars=false,
}

\newcommand{\inlinePython}{\lstinline[style=pythonstyle]}



\begin{document}

\PreFirstFrame
\begin{frame} [fragile]
	\centerline{\fontsize{42}{42}\selectfont Python Talk 2}
\end{frame}
\PostFirstFrame

\begin{frame} [fragile]
	\frametitle{复习}
	\linespread{2}
	\begin{columns}[T]
		\begin{column}[T]{.5\textwidth}
			\begin{itemize}
			\item 计算下列数值
				\begin{itemize}
				\item 一百加二十
				\item 一天有多少秒
				\item 真或假
				\item 三的一百次方
				\end{itemize}
			\end{itemize}
		\end{column}
		\begin{column}[T]{.5\textwidth}
			\begin{itemize}
			\item 答案
				\begin{itemize}
				\item \inlinePython{100 + 20}
				\item \inlinePython{24 * 60 * 60}
				\item \inlinePython{True or False}
				\item \inlinePython{3 ** 100}
				\end{itemize}
			\end{itemize}
		\end{column}
	\end{columns}
\end{frame}

\begin{frame} [fragile]
	\frametitle{复杂数据类型}
	\linespread{2}
	\begin{columns}[T]
		\begin{column}[T]{.5\textwidth}
			\begin{itemize}
			\item 常用类型
				\begin{itemize}
				\item 元组:\inlinePython{tuple}
				\item 列表:\inlinePython{list}
				\item 字符串:\inlinePython{str}
				\item 字典:\inlinePython{dict}
				\end{itemize}
			\item 其他类型
				\begin{itemize}
				\item 集合:\inlinePython{set}
				\end{itemize}
			\end{itemize}
		\end{column}
		\begin{column}[T]{.5\textwidth}
			\begin{itemize}
			\item 举例
				\begin{itemize}
				\item \inlinePython{(1, 2, 3)}
				\item \inlinePython{('h', 'c', 'c')}
				\item \inlinePython{"HCC, I'm. "}
				\item \inlinePython|{1: 'H', 2: 'C'}|
				\end{itemize}
			\item 举例
				\begin{itemize}
				\item \inlinePython|{1, 2, 3}|
				\end{itemize}
			\end{itemize}
		\end{column}
	\end{columns}
\end{frame}

\begin{frame} [fragile]
	\frametitle{创建你自己的复杂数据}
	\linespread{1.5}
	\begin{columns}[T]
		\begin{column}[T]{.5\textwidth}
			\begin{itemize}
			\item \inlinePython{(a, b, c)}
			\item \inlinePython{[a, b, c]}
			\item \inlinePython{'Some_Letters'}
			\item
			\begin{lstlisting}[style=pythonstyle, gobble=12]
			{
				key_A: value_A,
				key_B: value_B
			}
			\end{lstlisting}
			\end{itemize}
		\end{column}
		\begin{column}[T]{.5\textwidth}
			\begin{itemize}
			\item 所有元素都可以换成更复杂的类型
			\item 基本类型
				\begin{itemize}
				\item \inlinePython{int		float	bool}
				\end{itemize}
			\item 复杂类型
				\begin{itemize}
				\item \inlinePython{tuple	list	str}
				\item \inlinePython{dict	set}
				\end{itemize}
			\item \inlinePython{type}
			\end{itemize}
		\end{column}
	\end{columns}
\end{frame}

\begin{frame} [fragile]
	\frametitle{细节}
	\linespread{1.25}
	\begin{columns}[T]
		\begin{column}[T]{.5\textwidth}
			\begin{itemize}
			\item 创建一个元素的 \inlinePython{tuple}
			\begin{lstlisting}[style=pythonstyle, gobble=12, texcl]
			('HCC')		# 失败
			('HCC',)	# 正确
			\end{lstlisting}
			\item \inlinePython{list}和\inlinePython{dict}也支持
			\end{itemize}
		\end{column}
		\begin{column}[T]{.5\textwidth}
			\begin{itemize}
			\item \inlinePython{str}创建方式
				\begin{itemize}
				\item 单引号
				\begin{lstlisting}[style=pythonstyle, gobble=16, texcl]
				'HCC, I\'m. '
				\end{lstlisting}
				\end{itemize}
				\begin{itemize}
				\item 双引号
				\begin{lstlisting}[style=pythonstyle, gobble=16, texcl]
				"HCC, I'm. "
				\end{lstlisting}
				\end{itemize}
				\begin{itemize}
				\item 多行字符串
				\begin{lstlisting}[style=pythonstyle, gobble=16, escapechar=@]
				@\color{Pink}\textquotesingle\textquotesingle@'
				HCC
				'''
				\end{lstlisting}
				\end{itemize}
			\end{itemize}
		\end{column}
	\end{columns}
\end{frame}

\begin{frame} [fragile]
	\frametitle{\inlinePython{tuple}和\inlinePython{list}有什么区别}
	\linespread{1.5}
	\begin{columns}[T]
		\begin{column}[T]{.5\textwidth}
			\begin{lstlisting}[style=pythonstyle, gobble=12, texcl]
			a = ()		# tuple
			b = []		# list
			help(a)
			help(b)
			# 你能找到区别吗?
			\end{lstlisting}
		\end{column}
		\begin{column}[T]{.5\textwidth}
			\begin{itemize}
			\item \inlinePython{insert}
			\item \inlinePython{pop}
			\item \inlinePython{remove}
			\item \inlinePython{reverse}
			\item \inlinePython{sort}
			\end{itemize}
		\end{column}
	\end{columns}
\end{frame}

\begin{frame} [fragile]
	\frametitle{尝试基本运算符}
	\begin{itemize}
	\item 加法
	\begin{lstlisting}[style=pythonstyle, gobble=4, texcl]
	(1, 2) + (5, 6)
	[1, 2] + [5, 6]
	'HCC' + ", I'm."
	\end{lstlisting}
	\item 乘法
	\begin{lstlisting}[style=pythonstyle, gobble=4, texcl]
	(1, 2) * 3
	'HCC' * 10
	\end{lstlisting}
	\item 合并\inlinePython{dict}
	\begin{lstlisting}[style=pythonstyle, gobble=4, texcl]
	{1: 2} + {3: 4}		# 失败
	a = {1: 2}
	a.update({3: 4})
	print(a)			# 正确
	\end{lstlisting}
	\end{itemize}
\end{frame}

\begin{frame} [fragile]
	\frametitle{字符串方法}
	\linespread{1.5}
	\begin{lstlisting}[style=pythonstyle, gobble=4, texcl, escapechar=@]
	"HCC, I'm. ".upper()	# 大写
	"HCC, I'm. ".lower()	# 小写
	"HCC, I'm. ".split()	# 按空白字符分割
	len("HCC, I'm. ")		# 得到长度
	'&'.join([1, 2])		# 合并
	@\color{Pink}\textquotesingle\textquotesingle@'Hello from HCC:
	Welcome!
	'''.split('\n')			# 按换行符分割
	\end{lstlisting}
\end{frame}

\begin{frame} [fragile]
	\frametitle{分割}
	\linespread{1.5}
	\begin{lstlisting}[style=pythonstyle, gobble=4, texcl]
	a = [1, 2, 3, 4, 5, 6, 7, 8, 9]
	a[4]
	a[4: 7]
	a[2: -2: 2]
	a = 'abcdefghijklmnopqrstuvwxyz'
		# 重复上面三行
	{'a': 'b', 'b': 'c'}['b']
	\end{lstlisting}
\end{frame}

\begin{frame} [fragile]
	\frametitle{原理}
	\linespread{2}
	\begin{lstlisting}[basicstyle=\ttfamily,upquote=true,
						showstringspaces=false,tabsize=4,columns=fixed]
	   H   C   C   ,   I   '   m   .
	 0   1   2   3   4   5   6   7   8
	-8  -7  -6  -5  -4  -3  -2  -1
	\end{lstlisting}
\end{frame}

\begin{frame} [fragile]
	\frametitle{练习}
	\linespread{1.5}
	\begin{itemize}
	\item 对于这些字符串,截取出用户名(\inlinePython{'@'}前面)
	\begin{lstlisting}[style=pythonstyle, gobble=4, texcl]
	'HCC@shiyiquan.net'
	'lxy@shiyiquan.net'
	'mbl@shiyiquan.net'
	'shiyiquan@shiyiquan.net'
	\end{lstlisting}
	\item 你能使用两种方式吗?提示:分割和 split
	\item 哪种方式还可以正确截取这些字符串?
	\begin{lstlisting}[style=pythonstyle, gobble=4, texcl]
	'HCC@hcc.io'
	'lxy@shierquan.tk'
	\end{lstlisting}
	\end{itemize}
\end{frame}


\PreLastFrame
\begin{frame}
	\centerline{\fontsize{32}{32}\selectfont 感谢参加此次活动}
\end{frame}

\newpage
\end{document}

