\documentclass
% [aspectratio=169]
{beamer}

\usepackage{CJKutf8}
\usepackage{listings}
\usepackage{tikz}
\usepackage{hyperref}
\usepackage{xcolor}
\usepackage{verbatim}
\usepackage{eso-pic}
\usepackage{courier}
\usepackage{textcomp}

\newcommand{\link}[1]{\href{#1}{#1}}

\newcommand{\insertGraph}[3]{

	\centerline{\includegraphics[scale=#1]{#2}} % 0/0

	\centerline{#3}

}

\newcommand{\HCCLogoSimp}{
	\begin{tikzpicture}[scale=0.1]
		\definecolor{_00ccff}{HTML}{00ccff}
\def\center{(54.9655,-19.629)}
\def\radius{7.924}
\fill [color=_00ccff]
	(55.371,-11.075) -- (57.286,-11.417) -- (55.693,-20.323) -- 
	(53.803,-19.976);
\begin{scope}
	% (47.042,-11.705) rectangle (62.889,-27.553)
	% (49.018,-13.682) rectangle (60.913,-25.576)
%	\clip
%		(59.4605,-11.563) -- (54.9665,-19.629) -- (53.3695,-11.511) -- 
%		(47.041,-11.511) -- (47.041,-27.553) -- (62.890,-27.553) -- 
%		(62.890,-11.511);
%	\fill [color=_00ccff, even odd rule]
%		\center circle [radius=5.9475]
%		(59.4605,-11.563) -- (53.3695,-11.511) -- (54.9665,-19.629)
%		\center circle [radius=\radius]; 
	% https://tex.stackexchange.com/questions/510281/tikz-fill-only-the-a-b-c
	\path \center circle [radius=\radius];
	\clip[overlay]
		(53.3695,-11.511) -- (59.4605,-11.563) -- (54.9665,-19.629) [rev];
	\clip[overlay, eo] \center circle [radius=5.9475, rev];
	\fill[color=_00ccff] \center circle [radius=\radius];
\end{scope}


	\end{tikzpicture}
}

\newcommand{\HCCLogoFull}{
	\begin{tikzpicture}[scale=0.05]
		\definecolor{_00ccff}{HTML}{00ccff}
\def\center{(54.9655,-19.629)}
\def\radius{7.924}
\fill [color=_00ccff]
	(55.371,-11.075) -- (57.286,-11.417) -- (55.693,-20.323) -- 
	(53.803,-19.976);
\begin{scope}
	% (47.042,-11.705) rectangle (62.889,-27.553)
	% (49.018,-13.682) rectangle (60.913,-25.576)
%	\clip
%		(59.4605,-11.563) -- (54.9665,-19.629) -- (53.3695,-11.511) -- 
%		(47.041,-11.511) -- (47.041,-27.553) -- (62.890,-27.553) -- 
%		(62.890,-11.511);
%	\fill [color=_00ccff, even odd rule]
%		\center circle [radius=5.9475]
%		(59.4605,-11.563) -- (53.3695,-11.511) -- (54.9665,-19.629)
%		\center circle [radius=\radius]; 
	% https://tex.stackexchange.com/questions/510281/tikz-fill-only-the-a-b-c
	\path \center circle [radius=\radius];
	\clip[overlay]
		(53.3695,-11.511) -- (59.4605,-11.563) -- (54.9665,-19.629) [rev];
	\clip[overlay, eo] \center circle [radius=5.9475, rev];
	\fill[color=_00ccff] \center circle [radius=\radius];
\end{scope}


		\definecolor{_004455}{HTML}{004455}
\definecolor{_006680}{HTML}{006680}
\definecolor{_0088aa}{HTML}{0088aa}
\definecolor{_00aad4}{HTML}{00aad4}
\fill [color=_00aad4] (49.0,-46.0) rectangle (64.0,-50.5);
\fill [color=_0088aa] (44.5,-35.5) rectangle (49.0,-50.5);
\fill [color=_006680] (44.5,-31.0) rectangle (64.0,-35.5);
\fill [color=_00aad4] (27.5,-46.0) rectangle (42.5,-50.5);
\fill [color=_0088aa] (22.5,-35.5) rectangle (27.5,-50.5);
\fill [color=_006680] (22.5,-31.0) rectangle (42.0,-35.5);
\fill [color=_0088aa] (16.0,-35.5) rectangle (20.5,-51.0);
\fill [color=_006680] (4.5,-31.0) rectangle (20.5,-35.5);
\fill [color=_004455] (0.0,-24.0) rectangle (4.5,-53.5);
% \draw [color=red] (0.0,0.0) rectangle (64,-64);

	\end{tikzpicture}
}

\setbeamercolor{background canvas}{bg=}

\newcommand{\PreFirstFrame}{
	\AddToShipoutPictureFG*{
		\AtPageLowerLeft{
			\put(\LenToUnit{0.05\paperwidth},\LenToUnit{0.1\paperheight}){
				\footnotesize
				这个指引文档在
				\href{https://creativecommons.org/licenses/by-sa/3.0/deed.zh}
				{知识共享 署名-相同方式共享 3.0协议}之条款下提供
			}
			\put(\LenToUnit{0.05\paperwidth},\LenToUnit{0.05\paperheight}){
				\footnotesize
				This guidance is available under the 
				\href{https://creativecommons.org/licenses/by-sa/3.0/}
				{Creative Commons Attribution-ShareAlike License}
			}
			\put(\LenToUnit{0.6\paperwidth},\LenToUnit{0.15\paperheight}){
				\HCCLogoFull
			}
		}
	}
}

\newcommand{\PostFirstFrame}{
	\AddToShipoutPictureBG{
		\AtPageLowerLeft{
			\put(\LenToUnit{0.8\paperwidth},\LenToUnit{0.15\paperheight}){
				\HCCLogoSimp
			}
		}
	}
}

\newcommand{\PreLastFrame}{
	\ClearShipoutPictureBG

	\AddToShipoutPictureFG*{
		\AtPageLowerLeft{
			\put(\LenToUnit{0.6\paperwidth},\LenToUnit{0.15\paperheight}){
				\HCCLogoFull
			}
		}
	}
}

% Note: this C style differs a lot from gedit's
\lstdefinestyle{cstyle}{
	language=c,
	basicstyle=\ttfamily,
	morekeywords={with},
	keywordstyle=\bfseries\color[HTML]{a52a2a},	
	commentstyle=\color[HTML]{0000ff},
	stringstyle=\color[HTML]{ff0bff},
	keywordstyle=[3]\color[HTML]{008a8c},
	alsoletter={0,1,2,3,4,5,6,7,8,9,.},
	morekeywords=[4]{0,1,2,100,999},
	keywordstyle=[4]\color[HTML]{ff0bff},
	upquote=true,
	breaklines=true,
}

\lstdefinestyle{pythonstyle}{
	language=python,
	basicstyle=\ttfamily,
	% frame=single,
	morekeywords={with,yield},
	keywordstyle=\bfseries\color[HTML]{a52a2a},	
	keywordstyle=[2]\color[HTML]{008a8c},
	commentstyle=\color[HTML]{0000ff},
	stringstyle=\color[HTML]{ff0bff},
	keywordstyle=[3]\color[HTML]{008a8c},
	alsoletter={0123456789.},
	morekeywords=[4]{False,True,
		0,1,2,3,4,5,6,7,8,9,10,11,12,13,15,17,19,16,20,24,27,31,32,33,34,35,38,
		45,56,60,64,81,95,97,99,100,123,243,256,400,512,576,729,999,1024,1234,
		1365,1366,2000,2187,2836,2957,3856,3857,5678,6561,9274,100000,1000000,
		0.5,3.14,3.4,
		0x1234,},
	keywordstyle=[4]\color[HTML]{ff0bff},
	upquote=true,
	breaklines=true,
	showstringspaces=false,
}

\lstset{
	tabsize=4,
	columns=fixed,
	extendedchars=false,
}

\newcommand{\inlinePython}{\lstinline[style=pythonstyle]}



\begin{document}

\PreFirstFrame
\begin{frame} [fragile]
	\centerline{\fontsize{42}{42}\selectfont Python Talk 4}
\end{frame}
\PostFirstFrame

\begin{frame} [fragile]
	\frametitle{复习}
	\linespread{1.5}
	\begin{itemize}
	\item 使用while计算以下等比数列的\textbf{乘积}
		\begin{itemize}
		\item \inlinePython{27, 81, 243, 729, 2187, 6561}
		\end{itemize}
	\item 使用for和range计算以下等差数列的\textbf{乘积}
		\begin{itemize}
		\item \inlinePython{10, 17, 24, 31, 38, 45}
		\end{itemize}
	\end{itemize}
\end{frame}

\begin{frame} [fragile]
	\frametitle{循环和判断}
	\linespread{2}
	\begin{columns}[T]
		\begin{column}[T]{.5\textwidth}
			\begin{itemize}
			\item 循环(PythonTalk3)
				\begin{itemize}
				\item \inlinePython{for}
				\item \inlinePython{while}
				\end{itemize}
			\item 循环控制(PythonTalk4)
				\begin{itemize}
				\item \inlinePython{break}
				\item \inlinePython{continue}
				\end{itemize}
			\end{itemize}
		\end{column}
		\begin{column}[T]{.5\textwidth}
			\begin{itemize}
			\item 判断(PythonTalk4)
				\begin{itemize}
				\item \inlinePython{if}
					\begin{itemize}
					\item \inlinePython{else}
					\item \inlinePython{elif}
					\end{itemize}
				\end{itemize}
			\end{itemize}
		\end{column}
	\end{columns}
\end{frame}

\begin{frame} [fragile]
	\frametitle{\inlinePython{if}语句}
	\linespread{1.5}
	\begin{columns}[T]
		\begin{column}[T]{.5\textwidth}
			\begin{itemize}
			\item 语法
			\begin{lstlisting}[style=pythonstyle, gobble=12]
			if Expression :
				Code
			elif Expression :
				Code
			else :
				Code
			Code
			\end{lstlisting}
			\end{itemize}
		\end{column}
		\begin{column}[T]{.5\textwidth}
			\begin{itemize}
			\item 对比C语言
			\begin{lstlisting}[style=cstyle, gobble=12]
			if (Expression) {
				Code;
			} else if (Exp) {
				Code;
			} else {
				Code;
			}
			Code;
			\end{lstlisting}
			\end{itemize}
		\end{column}
	\end{columns}
\end{frame}

\begin{frame} [fragile]
	\frametitle{\inlinePython{if}例子}
	\linespread{1.25}
	\begin{itemize}
	\item 解释数值\inlinePython{a}和\inlinePython{b}的大小关系
	\begin{lstlisting}[style=pythonstyle, gobble=4]
	if a > b :
		print('Bigger')
	elif a < b :
		print('Smaller')
	else :
		print('Equal')
	\end{lstlisting}
	\end{itemize}
\end{frame}

\begin{frame} [fragile]
	\frametitle{\inlinePython{if}练习}
	\linespread{1.5}
	\begin{itemize}
	\item 输出a年是否为闰年
		\begin{itemize}
		\item 公元年分除以4不可整除,为平年。
		\item 公元年分除以4可整除但除以100不可整除,为闰年。
		\item 公元年分除以100可整除但除以400不可整除,为平年。
		\item 公元年分除以400可整除,为闰年。
		\item 以上规则来自
				\href{https://zh.wikipedia.org/wiki/\%E9\%97\%B0\%E5\%B9\%B4}
					{维基百科}
		\end{itemize}
	\end{itemize}
\end{frame}

\begin{frame} [fragile]
	\frametitle{答案}
	\linespread{1.25}
	\begin{columns}[T]
		\begin{column}[T]{.5\textwidth}
			\begin{lstlisting}[style=pythonstyle, tabsize=2, gobble=6, texcl]
			# 程序1
			if a % 4 == 0 :
				if a % 100 == 0 :
					if a % 400 == 0 :
						print(True)
					else :
						print(False)
				else :
					print(True)
			else :
				print(False)
			\end{lstlisting}
		\end{column}
		\begin{column}[T]{.47\textwidth}
			\begin{lstlisting}[style=pythonstyle, tabsize=2, gobble=6, texcl]
			# 程序2
			if a % 400 == 0 :
				print(True)
			elif a % 100 == 0 :
				print(False)
			elif a % 4 == 0 :
				print(True)
			else :
				print(False)
			\end{lstlisting}
		\end{column}
	\end{columns}
\end{frame}

\begin{frame} [fragile]
	\frametitle{\inlinePython{break}和\inlinePython{continue}}
	\linespread{1.25}
	\begin{columns}[T]
		\begin{column}[T]{.52\textwidth}
			\begin{itemize}
			\item 作用于最近的\inlinePython{for}或\inlinePython{while}
				\begin{itemize}
				\item Code1上的\inlinePython{break}或\inlinePython{continue}作用于
						``\inlinePython{for j in b :}''
				\end{itemize}
			\item \inlinePython{break}停止循环
				\begin{itemize}
				\item 若Code1为\inlinePython{break},则跳到Flag2
				\end{itemize}
			\item \inlinePython{continue}跳过这次循环
				\begin{itemize}
				\item 若Code1为\inlinePython{continue},则跳到Flag1
				\end{itemize}
			\end{itemize}
		\end{column}
		\begin{column}[T]{.48\textwidth}
			\begin{lstlisting}[style=pythonstyle, gobble=12]
			for i in a :
				for j in b :
					Code1 <<<
					Code2
					# Flag1
				# Flag2
				Code3
			\end{lstlisting}
		\end{column}
	\end{columns}
\end{frame}

\begin{frame} [fragile]
	\frametitle{\inlinePython{break}练习}
	\linespread{1.25}
	\begin{itemize}
	\item 输出整数$a$是否是质数
	\begin{lstlisting}[style=pythonstyle, gobble=4]
	flag = True
	for i in range(2, a) :
		if a % i == 0 :
			print(False)
			flag = False
			break
	if flag :
		print(True)
	\end{lstlisting}
	\item 思考:如何更改程序,在$a \le 1$时输出一个错误信息?
	\end{itemize}
\end{frame}

\begin{frame} [fragile]
	\frametitle{质数算法优化1}
	\linespread{1.25}
	\begin{itemize}
	\item 优化1:不计算的多余偶数
		\begin{lstlisting}[style=pythonstyle, gobble=8]
		if a == 2 :
			print(True)
		elif a % 2 == 0 :
			print(False)
		else :
			for i in range(3, a, 2) :
				if a % i == 0 :
					print(False)
					break
				print(True)
		\end{lstlisting}
	\item 提示:以上代码有误,请自行更正
	\end{itemize}
\end{frame}

\begin{frame} [fragile]
	\frametitle{\inlinePython{import}}
	\linespread{1.25}
	\begin{itemize}
	\item Python程序可以通过\inlinePython{import}调用一个包,来进行复杂计算和操作
	\item 例
		\begin{lstlisting}[style=pythonstyle, gobble=8, texcl]
		import math
		math.sqrt(33)	# 得到33的平方根
		\end{lstlisting}
	\item 对于平方根,可以不使用math包:
		\begin{lstlisting}[style=pythonstyle, gobble=8]
		33**0.5
		\end{lstlisting}
		但是使用诸如\inlinePython{sin}, \inlinePython{cos}的复杂函数时需要math包
	\end{itemize}
\end{frame}

\begin{frame} [fragile]
	\frametitle{质数算法优化2}
	\begin{itemize}
	\item 优化2:计算到$\lfloor\sqrt{a}\rfloor$,而不是$a - 1$
		\begin{itemize}
		\item 例如对于$49$,判断$2, 3, 5, 7$,而不是$2, 3, 5, 7, 9, ..., 47$
		\end{itemize}
		\begin{lstlisting}[style=pythonstyle, gobble=8]
		if a == 2 :
			print(True)
		elif a % 2 == 0 :
			print(False)
		else :
			import math
			e = int(math.sqrt(a)) + 1
			for i in range(3, e, 2) :
				if a % i == 0 :
					print(False)
					break
				print(True)
		\end{lstlisting}
	\item 提示:以上代码有误,请自行更正
	\end{itemize}
\end{frame}

\PreLastFrame
\begin{frame}
	\centerline{\fontsize{32}{32}\selectfont 感谢参加此次活动}
\end{frame}

\newpage
\end{document}

