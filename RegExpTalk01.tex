\documentclass
% [aspectratio=169]
{beamer}

\usepackage{CJKutf8}
\usepackage{listings}
\usepackage{tikz}
\usepackage{hyperref}
\usepackage{xcolor}
\usepackage{verbatim}
\usepackage{eso-pic}
\usepackage{courier}
\usepackage{textcomp}

\newcommand{\link}[1]{\href{#1}{#1}}

\newcommand{\insertGraph}[3]{

	\centerline{\includegraphics[scale=#1]{#2}} % 0/0

	\centerline{#3}

}

\newcommand{\HCCLogoSimp}{
	\begin{tikzpicture}[scale=0.1]
		\definecolor{_00ccff}{HTML}{00ccff}
\def\center{(54.9655,-19.629)}
\def\radius{7.924}
\fill [color=_00ccff]
	(55.371,-11.075) -- (57.286,-11.417) -- (55.693,-20.323) -- 
	(53.803,-19.976);
\begin{scope}
	% (47.042,-11.705) rectangle (62.889,-27.553)
	% (49.018,-13.682) rectangle (60.913,-25.576)
%	\clip
%		(59.4605,-11.563) -- (54.9665,-19.629) -- (53.3695,-11.511) -- 
%		(47.041,-11.511) -- (47.041,-27.553) -- (62.890,-27.553) -- 
%		(62.890,-11.511);
%	\fill [color=_00ccff, even odd rule]
%		\center circle [radius=5.9475]
%		(59.4605,-11.563) -- (53.3695,-11.511) -- (54.9665,-19.629)
%		\center circle [radius=\radius]; 
	% https://tex.stackexchange.com/questions/510281/tikz-fill-only-the-a-b-c
	\path \center circle [radius=\radius];
	\clip[overlay]
		(53.3695,-11.511) -- (59.4605,-11.563) -- (54.9665,-19.629) [rev];
	\clip[overlay, eo] \center circle [radius=5.9475, rev];
	\fill[color=_00ccff] \center circle [radius=\radius];
\end{scope}


	\end{tikzpicture}
}

\newcommand{\HCCLogoFull}{
	\begin{tikzpicture}[scale=0.05]
		\definecolor{_00ccff}{HTML}{00ccff}
\def\center{(54.9655,-19.629)}
\def\radius{7.924}
\fill [color=_00ccff]
	(55.371,-11.075) -- (57.286,-11.417) -- (55.693,-20.323) -- 
	(53.803,-19.976);
\begin{scope}
	% (47.042,-11.705) rectangle (62.889,-27.553)
	% (49.018,-13.682) rectangle (60.913,-25.576)
%	\clip
%		(59.4605,-11.563) -- (54.9665,-19.629) -- (53.3695,-11.511) -- 
%		(47.041,-11.511) -- (47.041,-27.553) -- (62.890,-27.553) -- 
%		(62.890,-11.511);
%	\fill [color=_00ccff, even odd rule]
%		\center circle [radius=5.9475]
%		(59.4605,-11.563) -- (53.3695,-11.511) -- (54.9665,-19.629)
%		\center circle [radius=\radius]; 
	% https://tex.stackexchange.com/questions/510281/tikz-fill-only-the-a-b-c
	\path \center circle [radius=\radius];
	\clip[overlay]
		(53.3695,-11.511) -- (59.4605,-11.563) -- (54.9665,-19.629) [rev];
	\clip[overlay, eo] \center circle [radius=5.9475, rev];
	\fill[color=_00ccff] \center circle [radius=\radius];
\end{scope}


		\definecolor{_004455}{HTML}{004455}
\definecolor{_006680}{HTML}{006680}
\definecolor{_0088aa}{HTML}{0088aa}
\definecolor{_00aad4}{HTML}{00aad4}
\fill [color=_00aad4] (49.0,-46.0) rectangle (64.0,-50.5);
\fill [color=_0088aa] (44.5,-35.5) rectangle (49.0,-50.5);
\fill [color=_006680] (44.5,-31.0) rectangle (64.0,-35.5);
\fill [color=_00aad4] (27.5,-46.0) rectangle (42.5,-50.5);
\fill [color=_0088aa] (22.5,-35.5) rectangle (27.5,-50.5);
\fill [color=_006680] (22.5,-31.0) rectangle (42.0,-35.5);
\fill [color=_0088aa] (16.0,-35.5) rectangle (20.5,-51.0);
\fill [color=_006680] (4.5,-31.0) rectangle (20.5,-35.5);
\fill [color=_004455] (0.0,-24.0) rectangle (4.5,-53.5);
% \draw [color=red] (0.0,0.0) rectangle (64,-64);

	\end{tikzpicture}
}

\setbeamercolor{background canvas}{bg=}

\newcommand{\PreFirstFrame}{
	\AddToShipoutPictureFG*{
		\AtPageLowerLeft{
			\put(\LenToUnit{0.05\paperwidth},\LenToUnit{0.1\paperheight}){
				\footnotesize
				这个指引文档在
				\href{https://creativecommons.org/licenses/by-sa/3.0/deed.zh}
				{知识共享 署名-相同方式共享 3.0协议}之条款下提供
			}
			\put(\LenToUnit{0.05\paperwidth},\LenToUnit{0.05\paperheight}){
				\footnotesize
				This guidance is available under the 
				\href{https://creativecommons.org/licenses/by-sa/3.0/}
				{Creative Commons Attribution-ShareAlike License}
			}
			\put(\LenToUnit{0.6\paperwidth},\LenToUnit{0.15\paperheight}){
				\HCCLogoFull
			}
		}
	}
}

\newcommand{\PostFirstFrame}{
	\AddToShipoutPictureBG{
		\AtPageLowerLeft{
			\put(\LenToUnit{0.8\paperwidth},\LenToUnit{0.15\paperheight}){
				\HCCLogoSimp
			}
		}
	}
}

\newcommand{\PreLastFrame}{
	\ClearShipoutPictureBG

	\AddToShipoutPictureFG*{
		\AtPageLowerLeft{
			\put(\LenToUnit{0.6\paperwidth},\LenToUnit{0.15\paperheight}){
				\HCCLogoFull
			}
		}
	}
}

% Note: this C style differs a lot from gedit's
\lstdefinestyle{cstyle}{
	language=c,
	basicstyle=\ttfamily,
	morekeywords={with},
	keywordstyle=\bfseries\color[HTML]{a52a2a},	
	commentstyle=\color[HTML]{0000ff},
	stringstyle=\color[HTML]{ff0bff},
	keywordstyle=[3]\color[HTML]{008a8c},
	alsoletter={0,1,2,3,4,5,6,7,8,9,.},
	morekeywords=[4]{0,1,2,100,999},
	keywordstyle=[4]\color[HTML]{ff0bff},
	upquote=true,
	breaklines=true,
}

\lstdefinestyle{pythonstyle}{
	language=python,
	basicstyle=\ttfamily,
	% frame=single,
	morekeywords={with,yield},
	keywordstyle=\bfseries\color[HTML]{a52a2a},	
	keywordstyle=[2]\color[HTML]{008a8c},
	commentstyle=\color[HTML]{0000ff},
	stringstyle=\color[HTML]{ff0bff},
	keywordstyle=[3]\color[HTML]{008a8c},
	alsoletter={0123456789.},
	morekeywords=[4]{False,True,
		0,1,2,3,4,5,6,7,8,9,10,11,12,13,15,17,19,16,20,24,27,31,32,33,34,35,38,
		45,56,60,64,81,95,97,99,100,123,243,256,400,512,576,729,999,1024,1234,
		1365,1366,2000,2187,2836,2957,3856,3857,5678,6561,9274,100000,1000000,
		0.5,3.14,3.4,
		0x1234,},
	keywordstyle=[4]\color[HTML]{ff0bff},
	upquote=true,
	breaklines=true,
	showstringspaces=false,
}

\lstset{
	tabsize=4,
	columns=fixed,
	extendedchars=false,
}

\newcommand{\inlinePython}{\lstinline[style=pythonstyle]}



\begin{document}

\PreFirstFrame
\begin{frame} [fragile]
	\centerline{\fontsize{42}{42}\selectfont Reg Exp Talk 1}
\end{frame}
\PostFirstFrame

\begin{frame} [fragile]
	\frametitle{介绍}
	\linespread{1.5}
	  正则表达式,又称正规表示式、正规表示法、正规运算式、规则运算式、常规表示法(英语:Regular Expression,在代码中常简写为regex、regexp或RE),是计算机科学的一个概念。正则表达式使用单个字符串来描述、匹配一系列符合某个句法规则的字符串。在很多文本编辑器里,正则表达式通常被用来检索、替换那些符合某个模式的文本。
	
	  来源:\href{https://zh.wikipedia.org/wiki/\%E6\%AD\%A3\%E5\%88\%99\%E8\%A1\%A8\%E8\%BE\%BE\%E5\%BC\%8F}{维基百科}
\end{frame}

\begin{frame} [fragile]
	\frametitle{Python中的使用方法}
	\linespread{1.5}
	\begin{lstlisting}[style=pythonstyle, gobble=4, texcl]
	>>> import re
	>>> re.findall('[0-9]', '1234')
	['1', '2', '3', '4']
	>>> help(re)	# 探索其他函数,如search, match
	>>>
	\end{lstlisting}
\end{frame}

\begin{frame} [fragile]
	\frametitle{Bash中的使用方法}
	\linespread{1.5}
	\begin{lstlisting}[style=bashstyle, gobble=4, texcl]
	$ grep '[0-9]' << FOE	# grep命令,FOE表示结束
	> 1234					# stdin输入内容
	> HCC					 # 第二行
	> py2exe				 # 第三行
	> FOE					 # stdin结束
	1234					# grep命令的输出
	py2exe					 # 第二行
	$
	\end{lstlisting}
\end{frame}

\begin{frame} [fragile]
	\frametitle{Gedit中的使用方法}
	\linespread{1.5}
	\begin{itemize}
	\item 用\texttt{Ctrl-F}打开查找
	\item 点击弹出框体左边的放大镜,选择``用正则表达式匹配''
	\item 输入表达式
	\end{itemize}
\end{frame}

\begin{frame} [fragile]
	\frametitle{中括号}
	\linespread{1.5}
	\begin{itemize}
	\item 中括号内一般可以表示一些字符区间,匹配一个字符
		\begin{itemize}
		\item \inlineListing{[0-9a-zA\-]} 分别代表数字、小写字母、A和-
		\end{itemize}
	\item 例如
	\begin{lstlisting}[style=pythonstyle, gobble=4, texcl]
	>>> re.findall('[0-9A-Z]', '12abAB')
	['1', '2', 'A', 'B']
	>>> re.findall('[1-9a]', '12abAB')
	['1', '2', 'a']
	>>>
	\end{lstlisting}
	\end{itemize}
\end{frame}

\begin{frame} [fragile]
	\frametitle{加号和点}
	\linespread{1.25}
	\begin{itemize}
	\item 加号代表将前一个规则匹配多次,点表示任何字符
	\begin{lstlisting}[style=pythonstyle, gobble=4, texcl]
	>>> re.findall('[1-9]+', '12ab45AB')
	['12', '45']
	>>> re.findall('.+', '1ab\na2\n43\n')
	['1ab', 'a2', '43']
	>>>
	\end{lstlisting}
	\end{itemize}
\end{frame}

\begin{frame} [fragile]
	\frametitle{小括号}
	\begin{itemize}
	\item 小括号在\inlinePython{findall}时代表需要得到的结果
		\begin{itemize}
		\item 尝试:有多个小括号时会怎样?
		\end{itemize}
	\begin{lstlisting}[style=pythonstyle, gobble=4, texcl]
	>>> re.findall(r'http://[a-z\.]+/',
	... 'http://shiyiquan.net/club/hcc/')
	['http://shiyiquan.net/']
	>>> re.findall(r'http://([a-z\.]+)/',
	... 'http://shiyiquan.net/club/hcc/')
	['shiyiquan.net']
	>>>
	\end{lstlisting}
	\item 也可以指定加号运算的优先级
	\begin{lstlisting}[style=pythonstyle, gobble=4, texcl]
	>>> re.findall('((ab)+)', 'bababababa')
	[('abababab', 'ab')]
	>>> re.findall('(a(b)+)', 'babbbbbba')
	[('abbbbbb', 'b')]
	>>>
	\end{lstlisting}
	\end{itemize}
\end{frame}

\begin{frame} [fragile]
	\frametitle{大括号}
	\linespread{1.5}
	\begin{itemize}
	\item 大括号定义重复次数
	\begin{lstlisting}[style=pythonstyle, gobble=4, texcl]
	>>> re.findall(r'[0-9]{3,4}', 
	... 			'123a1234a12345')
	['123', '1234', '1234']
	>>> 
	\end{lstlisting}
	\end{itemize}
\end{frame}

\begin{frame} [fragile]
	\frametitle{\inlineBash{^} 和 \inlineBash{$}}
	\linespread{1.5}
	\begin{itemize}
	\item \inlineBash{^}表示匹配字符串的开始,\inlineBash{$}表示匹配结束
	\begin{lstlisting}[style=pythonstyle, gobble=4, texcl]
	>>> re.findall(r'^[0-9]', '123')
	['1']
	>>> re.findall(r'[0-9]$', '123')
	['3']
	\end{lstlisting}
	\end{itemize}
\end{frame}

\begin{frame} [fragile]
	\frametitle{实例 - urls.py}
	\linespread{1.5}
	\begin{itemize}
	\item 节选自 \href{https://github.com/lxylxy123456/shierquan/}
			{\inlineListing{shierquan}} 项目的
			\href{https://github.com/lxylxy123456/shierquan/blob/master/quan\_account/urls.py}
			{\inlineListing{quan\_account/urls.py}}
	\begin{lstlisting}[style=pythonstyle, gobble=4, texcl]
	urlpatterns = [
		url(r'^signup/$', user_signup),
		url(r'^login/$', user_login),
		url(r'^club/([A-Za-z\-]+)/follow/$', follow),
		url(r'^logout/$', user_logout),
		url(r'^create/$', club_create),
		url(r'^search/(user)/$', search),
	]
	\end{lstlisting}
	\end{itemize}
\end{frame}

\begin{frame} [fragile]
	\frametitle{练习}
	\linespread{1.5}
	\begin{itemize}
	\item 打开附带的\href{https://github.com/lxylxy123456/HCCTalks/blob/master/RegExpTalk01.html}
						{\inlineListing{RegExpTalk01.html}},完成练习。
	\item Exercise 19 提示:研究\href{https://zh.wikipedia.org/wiki/\%E7\%A1\%AE\%E5\%AE\%9A\%E6\%9C\%89\%E9\%99\%90\%E7\%8A\%B6\%E6\%80\%81\%E8\%87\%AA\%E5\%8A\%A8\%E6\%9C\%BA}
	{确定有限状态自动机(DFA)}。编写题目要求的相应DFA,并将其转换为正则表达式。
	\end{itemize}
\end{frame}

\PreLastFrame
\begin{frame}
	\centerline{\fontsize{32}{32}\selectfont 感谢参加此次活动}
\end{frame}

\newpage
\end{document}

